% Latex for Document prepared for Daniel Daigle, during summer 
% Work-Study Project 
% This file inludes other section files 
% Use command pdflatex algebraicFunctionFields.tex to compile 
% and create pdf

\documentclass{article} 
% \documentclass[12pt,a4paper]{amsbook}
% \usepackage[margin=0.5in, top=0.5in]{geometry}
\usepackage{
	amsmath,amsfonts,euscript,enumerate,mathrsfs,
	hyperref,amsthm,amssymb,upref,graphics,color
	}
\usepackage{blindtext}
% \usepackage[utf8]{inputenc}
\usepackage[all]{xy}
\hypersetup{breaklinks=true,%
colorlinks=true,%
linkcolor=black,%
urlcolor=MyDarkBlue}

% Theorems 
\newtheoremstyle{plain}% name of the style to be used
  	{\topsep}% measure of space to leave above the theorem. E.g.: 3pt
  	{\topsep}% measure of space to leave below the theorem. E.g.: 3pt
  	{\normalfont}% name of font to use in the body of the theorem
  	{0pt}% measure of space to indent
  	{\bfseries}% name of head font
  	{}% punctuation between head and body
  	{5pt plus 5pt minus 1pt}% space after theorem head; " " = normal interword space
  	{}% Manually specify head

	\newtheorem{theorem}[subsection]{Theorem}
	\newtheorem{proposition}[subsection]{Proposition}
	\newtheorem{lemma}[subsection]{Lemma}
	\newtheorem{corollary}[subsection]{Corollary}
	\newtheorem{subtheorem}{Theorem}[subsection]
	\newtheorem{subproposition}[subtheorem]{Proposition}
	\newtheorem{sublemma}[subtheorem]{Lemma}
	\newtheorem{subcorollary}[subtheorem]{Corollary}

\newtheoremstyle{definition}% name of the style to be used
  	{\topsep}% measure of space to leave above the theorem. E.g.: 3pt
  	{\topsep}% measure of space to leave below the theorem. E.g.: 3pt
  	{\normalfont}% name of font to use in the body of the theorem
  	{0pt}% measure of space to indent
  	{\bfseries}% name of head font
  	{}% punctuation between head and body
  	{5pt plus 5pt minus 1pt}% space after theorem head; " " = normal interword space
  	{}% Manually specify head

	\newtheorem{definition}[subsection]{Definition}
	\newtheorem{definitions}[subsection]{Definitions}
	\newtheorem*{definonumber}{Definition}
	\newtheorem{nothing*}[subsection]{}
	\newtheorem{example}[subsection]{Example}
	\newtheorem{examples}[subsection]{Examples}
	\newtheorem*{solution}{Solution}
	\newtheorem*{exnonumber}{Example}
	\newtheorem*{quesnonumber}{Question}
	\newtheorem{question}{Question}
	\newtheorem{problem}[subsection]{Problem}	
	\newtheorem{exercise}[subsection]{Exercise}	
	\newtheorem{notation}[subsection]{Notation}
	\newtheorem{notations}[subsection]{Notations}
	\newtheorem{step}{Step}
	\newtheorem*{claim}{Claim}
	\newtheorem{assumptions}[subsection]{Assumptions}
	\newtheorem{subdefinition}[subtheorem]{Definition}
	\newtheorem{subnotation}[subtheorem]{Notation}
	\newtheorem{subnotations}[subtheorem]{Notations}
	\newtheorem{subexample}[subtheorem]{Example}
	\newtheorem{subnothing*}[subtheorem]{}

\newtheoremstyle{remark}
 	{0.5\topsep}   % ABOVESPACE
  	{0.5\topsep}   % BELOWSPACE
  	{\normalfont}  % BODYFONT
  	{0pt}       % INDENT (empty value is the same as 0pt)
  	{\itshape} % HEADFONT
  	{}         % HEADPUNCT
  	{5pt plus 1pt minus 1pt} % HEADSPACE
  	{}          % CUSTOM-HEAD-SPEC

	\newtheorem*{remark}{Remark}
	\newtheorem*{remarks}{Remarks}
	\newtheorem*{warning}{Warning}
	\newtheorem*{smallexample}{Example}

\numberwithin{equation}{subsection}
\renewcommand{\qedsymbol}{$\blacksquare$}
% \swapnumbers %% numbering for theorems will be on the left


%% Macros
	\DeclareMathOperator*{\Oplus}{\oplus} 
	% \newcommand{\coker}{    	\operatorname{{\rm coker}}}
	% \newcommand{\Aut}{		\operatorname{{\rm Aut}}}
	% \newcommand{\Spec}{		\operatorname{{\rm Spec}}}
	% \newcommand{\Proj}{		\operatorname{{\rm Proj}}}
	% \newcommand{\rank}{		\operatorname{{\rm rank}}}
	% \newcommand{\Reg}{		\operatorname{{\rm Reg}}}
	% \newcommand{\haut}{		\operatorname{{\rm ht}}}
	% \newcommand{\supp}{		\operatorname{{\rm supp}}}
	% \newcommand{\image}{		\operatorname{{\rm im}}}
	% \newcommand{\ord}{		\operatorname{{\rm ord}}}
	% \newcommand{\bideg}{		\operatorname{{\rm bideg}}}
	% \newcommand{\trdeg}{		\operatorname{{\rm trdeg}}}
	\newcommand{\Frac}{		\operatorname{{\rm Frac}}}
	% \newcommand{\Char}{		\operatorname{{\rm char}}}
	% \newcommand{\Span}{		\operatorname{{\rm Span}}}
	% \newcommand{\Sing}{		\operatorname{{\rm Sing}}}
	% \newcommand{\Pic}{		\operatorname{{\rm Pic}}}
	% \newcommand{\Cl}{			\operatorname{{\rm Cl}}}
	% \newcommand{\dom}{		\operatorname{{\rm dom}}}
	% \newcommand{\codom}{		\operatorname{{\rm codom}}}
	% \renewcommand{\div}{		\operatorname{{\rm div}}}
	% \newcommand{\id}{			\operatorname{{\rm id}}}
	% \newcommand{\ob}{			\operatorname{{\rm ob}}}
	% \newcommand{\Hom}{		\operatorname{{\rm Hom}}}
	% \newcommand{\Set}{		\operatorname{{\rm\bf Set}}}
	% \newcommand{\Top}{		\operatorname{{\rm\bf Top}}}
	% \newcommand{\Grp}{		\operatorname{{\rm\bf Grp}}}
	% \newcommand{\CRng}{		\operatorname{{\rm\bf CRng}}}

	% \newcommand{\Mod}[1]{\mbox{\rm ${#1}$-\bf Mod}}
	\newcommand{\setspec}[2]{\big\{\,#1\, \mid \,#2\, \big\}}
	% \newcommand{\powerset}{\raisebox{\depth}{\Large $\wp$}}

	\newcommand{\notdiv}{\not\hspace{\mylength}\mid}
	\newcommand{\epi}{\twoheadrightarrow}
	\newcommand{\overepi}[1]{\overset{ #1 }{\epi}}
	\newcommand{\monic}{\rightarrowtail}
	\newcommand{\overmonic}[1]{\overset{ #1 }{\monic}}
	\newcommand{\Integ}{\ensuremath{\mathbb{Z}}}
	\newcommand{\Nat}{\ensuremath{\mathbb{N}}}
	\newcommand{\Rat}{\ensuremath{\mathbb{Q}}}
	\newcommand{\Comp}{\ensuremath{\mathbb{C}}}
	\newcommand{\Reals}{\ensuremath{\mathbb{R}}}
	\newcommand{\aff}{\ensuremath{\mathbb{A}}}
	\newcommand{\proj}{\ensuremath{\mathbb{P}}}
	\newcommand{\bk}{{\ensuremath{\rm \bf k}}}
	\newcommand{\ck}{{\bar{\bk}}}
	\newcommand{\kk}[1]{\bk^{[#1]}}

	\newcommand{\Aeul}{\EuScript{A}}
	\newcommand{\Beul}{\EuScript{B}}
	\newcommand{\Ceul}{\EuScript{C}}
	\newcommand{\Deul}{\EuScript{D}}
	\newcommand{\Eeul}{\EuScript{E}}
	\newcommand{\Feul}{\EuScript{F}}
	\newcommand{\Geul}{\EuScript{G}}
	\newcommand{\Heul}{\EuScript{H}}
	\newcommand{\Keul}{\EuScript{K}}
	\newcommand{\Oeul}{\EuScript{O}}
	\newcommand{\Peul}{\EuScript{P}}
	\newcommand{\Seul}{\EuScript{S}}

	\newcommand{\Acal}{\mathcal{A}}
	\newcommand{\Bcal}{\mathcal{B}}
	\newcommand{\Ccal}{\mathcal{C}}
	\newcommand{\Dcal}{\mathcal{D}}
	\newcommand{\Ecal}{\mathcal{E}}
	\newcommand{\Fcal}{\mathcal{F}}
	\newcommand{\Gcal}{\mathcal{G}}
	\newcommand{\OV}{\mathcal{O}}

	\newcommand{\pgoth}{\mathfrak{p}}
	\newcommand{\Pgoth}{\mathfrak{P}}
	\newcommand{\qgoth}{\mathfrak{q}}
	\newcommand{\Qgoth}{\mathfrak{Q}}
	\newcommand{\m}{\mathfrak{m}}
	\newcommand{\M}{\mathfrak{M}}

	\newcommand{\Q}{\mathbb{Q}}
	\newcommand{\R}{\mathbb{R}}
	\newcommand{\Z}{\mathbb{Z}}
	\newcommand{\C}{\mathbb{C}}
	\newcommand{\K}{\mathbb{K}}
	\newcommand{\N}{\mathbb{N}}
	\newcommand{\FP}{F_P}
	\newcommand{\PF}{\mathbb{P}_F}
	\newcommand{\LL}{\mathscr{L}} %% RRP 
	\newcommand{\Leul}{\mathscr{L}} %% Dimension  of the RRP
	\newcommand{\ldim}{\EuScript{l}}
	\newcommand{\A}{\mathcal{A}_F}
	\newcommand{\notzero}{\backslash \lbrace 0 \rbrace }
	\newcommand{\IDEALa}{\mathfrak{a}}

	\newcommand{\IDEALp}{\mathfrak{p}}
	\newcommand{\fX}{f(X) = a_0 + a_1X + ... + a_nX^n}
	\newcommand{\fx}{f(x) = a_0 + a_1x + ... + a_nx^n}

	\newcommand{\Abf}{\mathbf{A}}
	\newcommand{\Bbf}{\mathbf{B}}
	\newcommand{\Cbf}{\mathbf{C}}
	\newcommand{\Dbf}{\mathbf{D}}
	\newcommand{\Ebf}{\mathbf{E}}

	\newcommand{\PP}{\mathbb{P}}
	\newcommand{\PPP}{\mathbb{P}}
	\newcommand{\SSS}{\mathbb{S}}
	\newcommand{\VV}{\mathbb{V}}
	\newcommand{\VVV}{\mathbb{V}}

	\newcommand{\dirlim}{\varinjlim}
	\newcommand{\ssi}{\Leftrightarrow}
	\newcommand{\isom}{\cong}
	\renewcommand{\epsilon}{\varepsilon}
	\renewcommand{\phi}{\varphi}
	\renewcommand{\emptyset}{\varnothing}
	\newcommand{\rien}[1]{}
	\renewcommand{\baselinestretch}{1.07}
	\newcommand{\HRule}{\rule{\linewidth}{0.5mm}} 

% Pages sizing


	% \setlength{\textwidth}{15.5cm}
	% % \setlength{\mylength}{1.45\mylength}
	% % \settowidth{\mylength}{$\,$}
	% \addtolength{\oddsidemargin}{-1cm}
	% \addtolength{\evensidemargin}{-1cm}
	% \addtolength{\textheight}{14mm}

	% \raggedbottom
	% \CompileMatrices
	% \newlength{\mylength}

	\setlength\parindent{0pt}


	% Title and toc
	\begin{document}
	\begin{titlepage}
	\begin{center}

	\HRule \\[0.4cm] % Horizontal line
	{\huge \bfseries Algebraic Function Fields}\\[0.4cm] 
	\HRule \\[0.8cm] % Horizontal line
	{\large A Document Prepared for Daniel Daigle} \\[1.0cm]
	{\large Kevin Johnson} \\[0.5cm]
	{\large \today}\\

	\end{center}
	\end{titlepage}

	\tableofcontents
	\pagebreak


% main document 
	\section{Places}
		

\bigskip
For the purpose of these notes, 
we will let $k$ be any arbitrary field.

\begin{definition} \label{functionField}    
    An algebraic function field $F/k$ of one
    variable over $k$ is a field extension $k \subseteq F $ such 
    that $F$ is a finite field extension of $k (x)$ for 
    some $x \in F$ which is transcendental over $k$. For simplicity,
    we refer to $F/k$ as a function field.  
\end{definition}

\begin{example} \label{rational}
    Let $F = k(x)$ for some transcendental element $x$ 
    over $k$. Then $F$ is an function field over $k$ and is called
     the \textit{rational} function field over $k$. 
\end{example}

\begin{example}
    $\Rat(\sqrt{2})/\Rat$ is not a function field because $\sqrt{2}$ is 
    algebraic over $\Rat$.
\end{example}

\begin{example} \label{elliptic}
    Let $p = y^2 + x^3 - x \in \Comp [x,y]$. Since $p$ is 
    irreducible over $\Comp$ (exercise \ref{irriduciblePoverC}), 
    the ring $A = \Comp[x,y]/ (p)$ is an integral domain. 
    Therefore we may consider the field of fractions $F$ of $A$. 
    Then $F/\Comp$ is a function field in one variable 
    over $\Comp$ (exercise \ref{irriduciblePoverC}).
\end{example}

\begin{example}
    The field of fractions $k(x,y)$ of a polynomial ring $k[x,y]$ in two variables
    over $k$ is not a function field in one variable: If it were, then 
    $k(x,y)/k(x)$ would have to be a finite extension. This contradicts the 
    algebraic independence of $x,y$ in $k[x,y]$.
\end{example}

\begin{proposition} \label{transIFFfinite}
    Let $F/k$ be an algebraic function field. 
    Then $z \in F$ is transcendental over $k$ if and only if the 
    extension $F/k(z)$ is of finite degree. 
\end{proposition}

\begin{proof}
    By definiton $F$ is a finite extension of $k(x)$ for some transcendental 
    element $x\in F$. Let $z \in F $, then we have the following chain of 
    inclusions; $ k \subseteq k(z) \subseteq F$. Suppose $z \in F $ is 
    transcendental over $k$. Since $F$ is a finite extension of $k(x)$, 
    $z$ is algebraic over $k(x)$, so there exists 
    $f(T) = a_0(x) + a_1(x)T + ... + a_n(x)T^n \in k(x)[T] \notzero $ 
    such that $f(z)= 0$, that is $0 = a_0(x) + a_1(x)z + ... + a_n(x)z^n$. 
    Notice that if all coefficients $a_0(x),a_1(x),...,a_n(x)$ were only in
    $k$, then $z$ would not be transcendental over $k$. So there must be at 
    least one coefficient in $k(x)$ which is not in $k$. We may re-write $f$ 
    as a polynomial in one variable with coefficients in $k(z)$ with $f(x) = 0 $. 
    Hence $x$ is algebraic over $k(z)$ and $[k(x,z) : k(z) ] < \infty$. 
    By definition $[F : k(x,z)] < \infty $, so $[F : k(z)] = [F: k(x,z)][k(x,z) : k(z)] < \infty$. 
    Conversly, assume $z \in F $ is algebraic over $k$ and suppose that $F/k(z)$ 
    is an extension of finite degree, then $[F : k(z)] < \infty $ and thus 
    $[F:k ] = [F: k(z)][k(z) : k ] < \infty $, which would also imply 
    $[k(x) : k ] < \infty$, which is impossible since $x$ is transcendental over $k$.  
\end{proof}

\begin{definition} \label{valuationRing}
    A valuation ring of a function field $F/k$ is a ring 
    $\OV \subseteq F $ with the following properties:
    \begin{enumerate}[(i)]
        \item \label{strictContain} $k \subsetneqq \OV \subsetneqq F$
        \item \label{zorzinvers} For every $z \in F $, we have that $z \in \OV $ or $ z^{-1} \in \OV$
    \end{enumerate}
\end{definition}

\begin{example} \label{rationalValuationRing}
    Consider the rational function field $k(x)/k$. Let $p$ be an irreducible 
    polynomial in $k[x]$. Then the ideal $(p)$ is prime in $k[x]$ and thus we
    may consider the localiztion of $k[x]$ at $(p)$, denoted $\OV_p$. That is, 
    $\OV_p = \lbrace f/g \mid f,g\in k[x], g\notin (p)\rbrace$ 
    $\OV_p$ is a valution ring of the function field $k(x)/k$: 
    Clearly $k\subsetneqq \OV_p$. Since $1/p \notin \OV_p $ but $1/p \in k(x)$,
    $\OV_p \subsetneqq k(x)$. So $\OV_p$ satisfies condition \eqref{strictContain} 
    of definition \ref{valuationRing}. To verify condition \eqref{zorzinvers},
    let $z = f/g \in k(x)$. If $g \notin (p)$,
    then $z \in \OV_p$ by definition. If $g \in (p)$ and $f \notin (p)$, then $z^{-1}\in \OV_p$. 
    If Both $f,g\in (p)$. Then $f = p^nu$ and $g = p^mv$ for some $u,v \in k[x]$
    such that $u,v \notin (p)$. Suppose $n \geq m $, then $z = f/g = p^{n-m}u/v \in \OV_p$,
    since $v \notin (p)$. If $n < m$, then $z = f/g = u/p^{m-n}v$, 
    hence $z^{-1}\in \OV_p$, since $u \notin (p)$. Hence 
    for every $z \in k(x)$, $z \in \OV_p$ or $z^{-1} \in \OV_p$.
\end{example}

\begin{example}
    Let $F$ be the field of fractions of the integral domain $A$ as 
    in example \ref{elliptic}. Consider the prime ideal $(x,y) \in \Comp[x,y]$.
    Since $(x,y)$ contains the kernel of the projection 
    $\Comp[x,y] \stackrel{\pi}{\longrightarrow} A$, the ideal $\m = \pi((x,y))$ 
    is prime in $A$ (exercise \ref{primeontoprime}). 
    Hence we may consider the localization of $A$ at $\m$, denoted $A_\m$.
    Then $A_\m$ is a valution ring of the function field $F/\Comp$ 
    (exercise \ref{irriduciblePoverC}).   
\end{example}

\begin{proposition} \label{propAboutValuationRings}
    Let $\OV$ be a valuation ring of a function 
    field $F/k$. Then the following hold; 
    \begin{enumerate}[(a)]
        \item $\OV$ is a local ring where $P = \OV \backslash \OV^*$ 
        denotes the maximal ideal of $\OV$.
        \item Let $0 \neq x \in F$. Then $x \in P \Leftrightarrow x^{-1} \notin \OV $
        \item Let $\tilde{k}$ denote the algebraic closure of $k$ in $F$. 
        Then $\tilde{k} \subseteq \OV$ and $\tilde{k} \cap P = \lbrace 0 \rbrace $. 
    \end{enumerate}
\end{proposition}

\begin{proof} \label{valuationLocal}
    \begin{enumerate}[(a)]
        \item It suffices to show that $P$ is an ideal of 
        $\OV$, as any ideal that properly contains $P$ would also 
        contain a unit: Let $x \in P, z \in \OV$, then $x  \notin \OV^*$ 
        by definition. Hence $xz \notin \OV^*$, thus $xz \in P$. 
        Let $x,y \in P$. Then either $xy^{-1} \in \OV$ or 
        $x^{-1}y \in \OV$. Assume $xy^{-1} \in \OV $. 
        Then $1 + xy^{-1} \in \OV$, since $1 \in \OV$. Hence 
        $x + y = y(1 + xy^{-1}) \in P $.
        \item Notice that $x \in P \Leftrightarrow x \in \OV \backslash \OV^* 
        \Leftrightarrow x \notin \OV^* \Leftrightarrow x^{-1} \notin \OV$.   
        \item Let $z \in \tilde{k} \notzero $ and suppose that $z \notin \OV $, 
        then by the definition of a valuation ring, $z^{-1}  \in \OV $. 
        Since $z^{-1}$ is algebraic over $k$, there exists a $\fX \in k[X] \notzero$ 
        such that $f(z^{-1}) = 0 $. Then, $a_0 + a_1(z^{-1}) + ... + a_n(z^{-1})^n = 0 $. 
        Assume $f$ is one of the non-zero polynomials of minimal degree satisfying 
        $f(z^{-1}) = 0$. We may also assume that $a_0 = 1$; To see this, suppose $a_0 = 0 $, 
        then $0 = a_1(z^{-1}) + ... + a_n(z^{-1})^n = z^{-1}(a_1 + a_2(z^{-1}) 
        + ... + a_n(z^{-1})^{n-1}$. 
        Since $z^{-1} \neq 0$, we have found another polynomial 
        $g(X) = a_1 + a_2X + ... + a_nx^{n-1} \in k[X] \notzero $ such that 
        $g(z^{-1}) = 0$ and $deg(g) < deg(f) $. This contradicts the minimality of 
        $f$. If $a_0 \neq 1$ and $a_0 \neq 0$, then since $a_0 \in k$, there 
        exists $a_0^{-1} \in k$ such that $a_0a_0^{-1} = 1 $. Then 
        $a_0^{-1}f= 1 + a_0^{-1}a_1X + ... + a_0^{-1}a_nX^n $ is a new 
        polynomial that is still zero on $z^{-1}$ and has the same degree 
        as $f$. Thus we may write $f(X) = 1 + a_1X + ... + a_nX^n  $ and therefore
        $-1 = z^{-1}(a_1 + ... + a_n(z^{-1})^{n-1}) \Longrightarrow z = 
        -(a_1 + ... + a_n(z^{-1})^{n-1}) \in \OV $. 
        This is a contradiction to the assumption $z \notin \OV $. Hence 
        $\tilde{k} \subseteq \OV $. Lastly, the inverse of an algebraic 
        element is algebraic and $\tilde{k} \subseteq \OV^* $, 
        hence $\tilde{k} \cap P = \lbrace 0 \rbrace $. 
    \end{enumerate}
\end{proof} 

\begin{example}
    As in example \ref{rationalValuationRing}, consider the rational 
    function field $k(x)/k$.
    Let $p$ be an irreducible poylnomial in $k[x]$. 
    Then $p\OV_p$ is the unique maximal ideal
    of the valuation ring $\OV_p$. To verify this, 
    let $z = f/g \in p\OV_p$, then $g \notin (p)$ but $f \in (p)$. 
    Hence $z^{-1} \notin \OV_p$. Therefore $z \in \OV_p \backslash \OV_p^*$. Let 
    $z = f/g \in \OV_p \backslash \OV_p^*$, then $z^{-1} \notin \OV_p$, hence $f \in (p)$ and 
    $g \notin (p)$. Thus $z \in p\OV_p$. So $p\OV_p = \OV_p \backslash \OV_p^*$. 
\end{example}

\begin{definition}
    Let $\OV$ be a valution ring of a function field $F/k$ and 
    $P= \OV \backslash \OV^* $ be its maximal ideal. We call $P$ a \textit{place} of 
    the function field $F/k$. Since $\OV$ is uniquely determined by 
    $P$, that is, $\OV = \lbrace z \in F \mid z^{-1} \notin P \rbrace$, we 
    often denote it $\OV_p$, refered to as the valuation ring at place 
    $P$. We use $\PF$ to denote the set of all places of $F/k$ 
    and $\mathbb{V}_F$ to denote the set of all valuation rings of $F/k$.
\end{definition}

\begin{example}
	Let $F=k(x)$ be the rational function field over $k$ as in example \ref{rational}.
	Example \ref{rationalValuationRing} shows $ \lbrace k[x]_{(p)} \mid 
	\text{p is irreducible in } k[x] \rbrace \subseteq \mathbb{V}_F  $.
	Theorem \ref{noPlaceLikeHome} shows this set is all valuation 
	rings of $k(x)/k$ except one. Furthermore $
	\lbrace pk[x]_{(p)} \mid 
	\text{p is a irreducible polynomial in } k[x] \rbrace \subseteq \mathbb{P}_F
	$, where $pk[x]_{(p)}$ denotes the maximal ideal of the local ring $k[x]_{(p)}$
\end{example}

\begin{definition}
	Let $P$ be a place of a function field $F/k$ and $\OV_P$ the valuation ring 
	at place $P$. Since $P$ is a maximal ideal, the quotient ring $\OV_P/P$ is a 
	field. We call this the residue class field of $P$, denoted by $\FP$. The
	degree of a place $P$ is defined as $\deg P = [\FP : k]$ and we call a place
	of degree one a \textit{rational} place of $F/k$.
\end{definition}

\begin{example}
	For the rational function field $F = \Comp(x)$ over the complex numbers $\Comp$, 
	all places of the form $P :=p\Comp[x]_{(p)}$ for some irreducible $p \in \Comp[x]$, 
	have degree 1: Since $\Comp $ is algebraically closed, every irreducible
	polynomial $p$ is linear. Thus the degree of $P$ is equal 
	to the degree of the field extension $[F_P : \Comp]$. Theorem \ref{propaboutrationals}
	shows that $F_P \cong \Comp[x]/(p)$. Thus $\text{deg} P = [\Comp[x]/(p):\Comp] = 1$. 
	More generally, all places of this form, over an algebraically closed field are degree 1.
	This is not true in any arbitrary field: Consider the polynomial $f = x^2 +1 \in \R [x]$. The
	place $f\R[x]_{(f)}$ of the function field $\R(x)/\R$ has degree 2. 
\end{example}


\begin{definition}  \label{discreteValuation}
	Let $k$ be a field. Let $\infty$ denote any element that is not in $\Z$ 
	satisfying; $\infty + \infty = \infty + n = n+ \infty = \infty  $ and 
	$\infty > m $ For all $n,m \in \Z$.  A discrete valuation of $F/k$ is a 
	function $ v : F \longrightarrow \Z \cup \lbrace \infty \rbrace $ with the
	following properties: 
	\begin{enumerate}[i)]
		\item \label{cond1} $v(x) = \infty \Leftrightarrow x = 0 $.
		\item \label{cond2} $v(xy) = v(x) + v(y) $ for all $x,y \in F$.
		\item \label{cond3} $v(x + y) \geq min \lbrace  v(x),v(y) \rbrace  $ for all $x,y \in F$.
		\item \label{cond4} There exists an element $z \in F$ with $v(z)=1$.
		\item \label{cond5} $v(a)=0 $ for all $0 \neq a \in k$
	\end{enumerate}
\end{definition}

\begin{lemma} \label{strongtriange}
	Let $v$ discrete valuation on a function field $F/k$. Then;
	$$v(x + y) = min \lbrace v(x),v(y) \rbrace $$ for all $x,y\in  F$ such that $v(x) \neq v(y) $. 
\end{lemma}

\begin{proof}
	Asssume $v(x) < v(y)$ and suppose $v(x + y) \neq min \lbrace v(x),v(y) \rbrace $.
	Then $v(x+y) > v(x)$ by \eqref{cond3}. Therefore 
	$v(x) = v((x+y) - y) \geq \text{min} \lbrace v_(x+y),v(y) \rbrace > v(x)$, 
	which is impossible.  
\end{proof}

\begin{example} \label{infinityvaluation}
	Consider the rational function field $k(x)/k$. We define the map 
	$k(x) \stackrel{v_{\infty}}{\longrightarrow} \Integ \cup \lbrace \infty \rbrace $ by: 
	for all $z = f/g \in k(x) \backslash \lbrace 0 \rbrace $,
	$v_\infty(z)= \text{deg}(g) - \text{deg}(f)$ and $v_\infty(0) = \infty$.
	Then $v_\infty$ is a discrete valution of the $k(x)/k$. 
\end{example}

\begin{proof}
	Property \eqref{cond1} follows by definition of $v_\infty$. 
	Let $x = f/g, y = f^{\prime}/g^{\prime} \in k(x)$. Then 
	\begin{align*}
		v_\infty(xy) &= v_\infty(\frac{f}{g}  \frac{f^{\prime}}{g^{\prime}} ) \\
		&=\text{deg}(gg^{\prime}) - \text{deg}(ff^{\prime}) \\
		&= \text{deg}(g) + \text{deg}(g^{\prime}) - \text{deg}(f) - \text{deg}(f^{\prime}) \\
		&= v_\infty(x) + v_\infty(y)
	\end{align*}
	This shows property \eqref{cond2}. Assume $v_\infty(x) \geq v_\infty(y)$. Then 
	$\text{deg}(g) - \text{deg}(f) \geq \text{deg}(g^{\prime}) - \text{deg}(f^{\prime}) 
	\Longrightarrow \text{deg}(g) + \text{deg}(f^{\prime})  \geq \text{deg}(g^{\prime}) +  \text{deg}(f)
	\Longrightarrow \text{deg}(gf^{\prime}) \geq \text{deg}(g^{\prime}f)$. So  
	\begin{align*}
		v_\infty(x+ y) &= v_\infty(\frac{f}{g} + \frac{f^{\prime}}{g^{\prime}} ) \\
		&= v_\infty(\frac{fg^{\prime} + f^{\prime}g}{gg^{\prime}} ) \\
		&= \text{deg}(gg^{\prime}) - \text{deg} (fg^{\prime} + f^{\prime}g) \\
		&= \text{deg}(gg^{\prime})  - \text{max} \lbrace \text{deg}(fg^{\prime}),
		\text{deg}(f^{\prime}g) \rbrace \\
		&= \text{deg}(gg^{\prime})   - \text{deg}(f^{\prime}g) \\
		&= \text{deg}(g) + \text{deg}(g^{\prime}) - \text{deg}(g) - \text{deg}(f^{\prime}) \\
		&= \text{deg}(g^{\prime}) - \text{deg}(f^{\prime})  \\
		&= v_\infty(y) \\
		&= \text{min} 
		\lbrace v_\infty(x), v_\infty(y) \rbrace
	\end{align*} So $v_\infty$ satisfies \eqref{cond3}. Lastly $v_\infty(1/x) = 1$ 
	and clearly $v_\infty(a) = 0 $ for all $a \in k^* \backslash \lbrace 0 \rbrace$. 
	So $v_\infty $ satisfies conditions \eqref{cond4} and \eqref{cond5}. Thus $v_\infty$
	is a discrete valuation on the function field $k(x)/k$.
\end{proof}

\begin{theorem} \label{primeValuation}
	Let $\OV_P$ be a valution ring of a function field $F/k $ 
	with maximal ideal $P$.
	\begin{enumerate}[(a)]
		\item \label{Pisalwaysprinzipal} $P$ is a principal ideal.
		\item \label{existst} If $P = t\OV$, then each $0 \neq z \in F$ has a 
		unique representation of the form $z = t^n u$ for some
		$n \in \Z $ and $u \in \OV^*$.  
		\item \label{OVisprincipaldomain} $\OV$ is a principal ideal domain. More precisely, if 
		$P = t \OV$ and $ \lbrace 0 \rbrace \neq  I \subseteq \OV$ 
		is an ideal, then $I = t^n \OV$ for some $n \in \N$. 
		\item  \label{valutionringbyvaluations} To a place 
		$P \in \PF$ we associate a function $ F  \stackrel{v}{\longrightarrow} 
		\Z \cup \lbrace \infty \rbrace $  as follows; Choose a prime element 
		$t$ for $P$. Then every $0 \neq z \in F$ has a unique representation
		$z = t^nu$ with $u \in \OV_P^* $ and $n \in \Z \cup \lbrace \infty \rbrace  $.
		Define $v_P(z) = v_P(t^nu) := n $ and $v_P(0) := \infty$. 
		The function $v_P$ is a dicrete valuation of
		$F/k$. Moreover we have,
		$$  \OV_P = \lbrace z \in F \mid  v_P(z) \geq  0 \rbrace $$ 
		$$\OV_P^*  = \lbrace z \in F \mid  v_P(z) = 0   \rbrace  $$
		$$P = \lbrace z \in F \mid v_P(z) > 0  \rbrace  $$
		\item \label{primeiffvaleq1} An element $x \in F$ is a prime 
		element for $P$ if and only if 
		$v_P(x) = 1$
		\item \label{valuationdefinesring} Conversly, suppose that $v$ is 
		a discrete valuation of $F/k$. 
		Then the set $P = \lbrace z \in F \mid v_P(z) > 0  \rbrace  $ is a 
		place of $F/k$, and $\OV_P = \lbrace z \in F \mid  v_P(z) \geq  0  \rbrace $
		is the corresponding valuation ring. 
		\item \label{maxproperideal} Every valuation ring $\OV$ of $F/k$ 
		is a maximal proper 
		subring of $F$. 
	\end{enumerate}
\end{theorem}

\begin{proof}
	See Stichtenoth, theorem 1.1.6 for parts \ref{Pisalwaysprinzipal}, 
	\ref{existst} and 
	\ref{OVisprincipaldomain}. 
	\eqref{valutionringbyvaluations}:
	First, we verify the conditions of a discrete valaution. \eqref{cond1}  
	By definiton of $v_P$ we have $v_P(x)= \infty  \Leftrightarrow x = 0$. \eqref{cond2}
	Let $x,y \in F$ and write $x = t^nu, y = t^mv $ for $n,m \in \Z$ and 
	$u,v \in \OV_P^*$. 
	Then $v_P(xy) = v_P(t^nut^mv) = v_P(t^{n+m}uv) = n + m = v_P(x) = v_P(y) $.
	\eqref{cond3} We have $v_P(x + y) = v_P(t^nu+t^mv)$. If $n \geq m $, 
	then $v_P(t^nu+t^mv) = v_P(t^m(t^{n-m}u+v)) = m + (n - m) = n \geq 
	m = min \lbrace v_P(x), v_P(y) \rbrace$. Simularly, if $m \geq n$, then
	$v_P(t^nu+t^mv) \geq n = min \lbrace v_P(x), v_P(y) \rbrace$. \eqref{cond4}
	$v_P(t) = v_P(t^1) = 1 $. \eqref{cond5} Suppose $0 \neq a \in k $, 
	then $a \in \OV_P^*$, hence $a =t^0a $, thus $v_P(a) = v_P(p^0a) = 0$. 
	The remaining assertions in part $a)$ follow directly from the fact 
	that every $0 \neq z \in F$ can be written
	uniquely as $z = t^nu $ for some $n \in \Z $, $u \in \OV^*$ and condition
	\eqref{cond5}, which asserts that every $0 \neq z \in \OV_P$ such that $z \notin P$ 
	has discrete valuation $0$.
	\eqref{primeiffvaleq1}: 
	If $x$ is a prime element of $P$, then $x = x^1$, so 
	by definition $v_P(x) = 1$. 
	Let $x \in F$ such that $v_p(x) = 1$. Then $x = t^1u$ 
	for some $u \in \OV^*$, thus $t = xu^{-1}$. Given any $y \in P$, $y = t^mv $ 
	for some $v \in \OV^*$ and $m \in \Z$. Hence $y = (xu^{-1})^mv = x^mw $ 
	for $w = u^{-m}v \in \OV^*$. So $x$ is a prime element of $P$.
	\eqref{valuationdefinesring}:
	Let $z \in F$, write $z = t^nu$ for some $n \in Z$ and $u \in \OV_P^*$. 
	Suppose $n \geq 0$, then clearly 
	$z \in \OV_P = \lbrace z \in F \mid  v_P(z) \geq  0  \rbrace  $. 
	If $n < 0 $, then $z^{-1} = (t^{-n}u)^{-1} = t^nu^{-1}$, 
	hence $z^{-1} \in \OV_P$. So $\OV_P$ is a valuation ring of $F$. 
	The units of $\OV_P$ are precisely the elements with $v_P(x) \geq 0$ 
	and $v_P(x^{-1}) \geq 0$. Hence $x = t^nu$ and $x = t^{-n}u$ with 
	$n \in \Z$, $u \in \OV^*$ and $n \geq 0$ and $-n \geq 0$. Hence $n = 0$. 
	So  $P = \lbrace z \in F \mid v_P(z) > 0  \rbrace = \lbrace z \in F \mid  
	v_P(z) \geq  0  \rbrace \backslash \lbrace z \in F \mid  v_P(z) = 0  
	 \rbrace = \OV \backslash \OV^*$. 
	\eqref{maxproperideal}: 
	Let $z \in F \backslash \OV$. Claim: $F = \OV[z]$: Let $y \in F$,
	then $v_P(yz^{-k}) \geq 0$ for sufficiently large $k\geq 0 $. So
	$w = yz^{-k} \in \OV$ and $y=wz^k \in \OV[z]$.
\end{proof}

\begin{proposition} 
	If $P$ is a place of a function field $F/k$ and 
	$0 \neq x \in P$, then $ \deg P \leq [F:k(x)] < \infty $.
\end{proposition}

\begin{proof}
	Let $P \in \PF $ and $0\neq  x \in P$. There are two inequalities 
	to show; 
	\begin{enumerate}[(i)]
		\item \label{lessthaninfinity} $[F:k(x)] < \infty $ 
		\item \label{lessthandegextension} $ \deg P \leq [F:k(x)]$
	\end{enumerate}
	\eqref{lessthaninfinity} Since $0 \neq x \in P$ is 
	transcendental, by proposition \ref{transIFFfinite}, $[F:k(x)]$ is finite.
	\eqref{lessthandegextension} Suppose $a_1(x)x_1 + a_2(x)x_2 + ... + a_n(x)x_n = 0$ 
	is some non-trivial linear combination of elements in $F$ where
	$a_1(x),...,a_n(x) \in k(x)$. Assume that $x$ does not divide each
	$a_i(x)$, hence each $a_i(x)$ may be expressed as $a_i(x) = a_i + g_i(x)x$ 
	for some $g_i(x) \in k[x]$ and $a_i(x) \in k$, with not all $a_i = 0$. 
	Since $x \in P$ and $g_i(x ) \in \OV$, we have $a_i(x) \equiv a_i$ 
	mod $P$. If we apply the residue class map to $a_1(x)x_1 + a_2(x)x_2 + ... + a_n(x)x_n = 0$ 
	we get $0 + P = a_1(x_1 + P) + a_2(x_2 + P)^2 + ... + a_n(x_n + P)^n $ 
	where not all $a_i = 0$. Hence $x_1 + P,x_2 + P, ..., x_n + P $ are 
	linearly dependent over $k$. Thus, any elements $x_1,...,x_n \in \OV$, 
	who's residue classes $x_1 + P,...,x_n + P$ are linearly independent 
	over $k$, are linearly independent over $k(x)$.
\end{proof}

\begin{definition}
	Let $F/k $ be an function field. 
	Let $z \in F $ and $P \in \PF$. We say that $P$ is 
	a zero of $z$ if $v_P(z) > 0$; $P $ is a pole of $z $ 
	if $v_P(z) < 0$. If $v_P(z) = m > 0$, $P$ is a zero of $z$ 
	of order $m$; if $v_P(z) = -m < 0 $, $P$ is a pole of $z$ 
	of order $m$.
\end{definition}

\begin{example} \label{firstencounterwithpolesanddivisors}
	Let $F = \mathbb{C}(x)/\mathbb{C}$ be the rational function field
	and consider the polynomial $f = x^3(x+1) \in F$. Let $P_{x}$ denote
	the maximal ideal of 
	$\OV_x = \lbrace f/g \mid f,g \in \Comp[x] \text{ and } g \notin (x) \rbrace $
	The prime element for $P_x$ is the polynomial $x$. Let $v_x$ be 
	the discrete valution corresponding to the polynomial $x$, 
	as in theorem \ref{primeValuation}. 
	Then $f = x^3(x+1)$. To assert that $v_x(f) = 3$, 
	we need to show that $x+1$ is a unit in $\OV_x$.
	Notice that $x + 1 \in  \mathcal{O}_P$ 
	since $x \nmid 1$. Similarly $(x+1)^{-1} \in \mathcal{O}_P$ 
	since $x \nmid x + 1 $. Hence $x+1 \in \mathcal{O}_P^*$. 
	Therefore the valuation of $f$ at place $P_x$ is $v_x(f)= 3$. 
	Let $P_{x+1}$ denote the maximal ideal of the valuation ring
	$\OV_{x+1} = \lbrace f/g \mid f,g \in \Comp[x] \text{ and } g \notin (x+1) \rbrace $
	The prime element for $P_{x+1}$ is the polynomial $x+1$. Let $v_{x+1}$ be 
	the discrete valution corresponding to the polynomial $x+1$. 
	A similar argument shows that $v_{x+1}(f) = 1$. So $f$ 
	has zeros at $P_x$ and $P_{x+1}$. 
	Let $v_\infty$be the discrete valution defined as in example \ref{infinityvaluation}. 
	Part \eqref{valuationdefinesring} of theorem \ref{primeValuation}
	shows that we may obtain a valution ring 
	$
	\OV_\infty := \lbrace f/g \in \Comp(x) \mid \deg (g) - \deg (f) \geq 0 \rbrace
	$ 
	with corresponding place 
	$
	P_\infty := \lbrace f/g \in \Comp(x) \mid \deg (g) - \deg (f) > 0 \rbrace
	$.
	Let $z = x^{-1}$. Then part \eqref{primeiffvaleq1} asserts that 
	since $v_\infty(z)=1$, the element $z$ is a prime element of $P_\infty$. 
	So $v_\infty$ may be redefined the same way as in theorem \ref{primeValuation}, 
	where $z$ is the prime element of the place $P_\infty$.
	Notice that $f = z^{-4}(x+1)x^{-1}$. To conclude that $v_\infty (f) = -4$,
	it suffices to show that $(x+1)x^{-1}$ is a unit in $\OV_\infty$. 
	We know that the units in $\OV$ are exactly those which have valuation $0$, 
	by part \eqref{valutionringbyvaluations} of theorem \ref{primeValuation}.
	So we compute $v_\infty((x+1)x^{-1}) = \deg(x) - \deg(x+1) = 1-1=0$. 
	Hence $v_\infty(f) = -4$. This means the place $P_\infty$ is a pole 
	of $f$ in $\Comp(x)/\Comp$. 
\end{example}

\begin{theorem} \label{subringIsValuation}
	Let $F/k$ be a function field and let $R$ be a subring 
	of $F$ with $k \subseteq R \subseteq F $. Suppose 
	that $\lbrace 0 \rbrace \neq I \subsetneq R $ is a 
	proper ideal of $R$. Then there is a place $P \in \PF$ 
	such that $I \subseteq P $ and $ R \subseteq \OV_P$.  
\end{theorem}

\begin{proof}
	See proof of theorem 1.1.13 in Stinchtenoth. 
\end{proof}

\begin{corollary} \label{everyonesgotapole}
	Let $F/k$ be a function field, $z \in F$ transcendental 
	over $k$. Then $z$ has at least one zero and one pole. 
	In particular $\PF \neq \emptyset $.
\end{corollary}

\begin{proof}
	Let $R=k[x]$ and consider the ideals $I= zR$ and $J=z^{-1}R$.
	By theorem \ref{subringIsValuation} there exists a places
	$P,Q\in \PF$ with $z \in P$ and $z^{-1} \in Q$, so both
	$z$ and $z^{-1}$ have zeros in $F$. Thus both $z$ and $z^{-1}$ 
	both have poles in $Q$ and $P$ respectively.
\end{proof}

	\section{The Rational Function Field}
		
\begin{proposition} \label{propaboutrationals}
	Let $F = k(x)$ be a ration function field, where $k$ is
	any field. Then the following hold;
	\begin{enumerate}[(a)]
		\item \label{isoofresidueraitonal} 
		Let $p(x)$ be an irreducible polynomial in $k[x]$ 
		and  $P = P_{p(x)}$ a place of $F$, then $F_P \cong k[x]/(p(x))$.
		\item \label{intinitydegree1}The infinity place defined in example 
		\ref{firstencounterwithpolesanddivisors} is rational.
	\end{enumerate}
\end{proposition}

\begin{proof}
	\eqref{isoofresidueraitonal}: The map $f(x) \mapsto f(x) + P$ 
	is homomorphism from $k[x]$ onto $F_P$ with kernel $(p(x))$. 
	\eqref{intinitydegree1}: Consider the place $P_x$, corresponding
	to the polynomial $p = x$. From \eqref{isoofresidueraitonal},
	we know that $F_P \cong k[x]/(x)$. 
	Hence $\deg P_x = [F_p : k] = [k[x]/(x): k] = 1$. Make the change of 
	coordinate $t = x^{-1}$, then $P_\infty = P_t$. Hence $\deg P_\infty = 1$. 
\end{proof}


\begin{theorem} \label{noPlaceLikeHome}
	There are no places of the rational function field
	$k(x)/k$ other than the places $P_{p(x)}$ and $P_{\infty}$
	where $p(x)$ is a monic irreducible polynomial in $k[x]$. 
\end{theorem}

\begin{proof}
	Let $P$ be a place of $k(x)/k$. There are two cases; 
	$x \in \OV_P$ or $x \notin \OV_P$. 
	Suppose the former. Then $k[x] \subset \OV_P$. Let 
	$I = k[x] \cap P$. I is a prime ideal of $k[x]$. Thus 
	$k[x]/I$ embeds into the field $k(x)_P$ through the 
	residue class map. Hence $I \neq 0$ by proposition 
	\ref{propAboutValuationRings}. So there exists a 
	\textit{uniquely determined} irreducible monic polynomial $p(x) \in k[x]$
	such that $I = p(x)k[x]$. Every $g(x) \in k[x]$ with $p(x) \nmid g(x)$ 
	is not in $I$, so $g(x) \notin P$ and $1/g(x) \in \OV_P$ by proposition 
	\ref{propAboutValuationRings}. 
	Therefore $\OV_P= \lbrace \frac{f(x)}{g(x)} \mid f(x), g(x) 
	\in k[x] \text{ and } p(x) \nmid g(x) \rbrace \subseteq \OV_P$.
	By theorem \ref{primeValuation}, all valuation rings 
	are maximal proper subrings, thus $\OV_P = \OV_{p(x)}$.
	For the second case; $x \notin \OV_P$, we must have
	$k[x^{-1}] \in \OV_P$ because $\OV_P$ is a valuation ring.
	So, as in case $1$, $x^{-1} \in P \cap k[x^{-1}] $ and 
	$P\cap k[x^{-1}]= x^{-1} k[x^{1}]$. 
	So 
	\begin{align*}
		\OV_P 
		&\subseteq 
		\lbrace 
		\frac{f(x^{-1})}{g(x^{-1})} \mid f(x^{-1}), g(x^{-1}) \in k[x^{-1}] \text{ and } x^{-1} \nmid g(x^{-1}) 
		\rbrace \\
		&= 
		\lbrace 
		\frac{a_0 + a_1x^{-1} + ... + a_nx^{-n}}{b_0 + b_1x^{-1} + ... + b_mx^{-m}} \mid b_0 \neq 0 
		\rbrace \\
		&=
		\lbrace 
		\frac{a_0x^{n+m} + ... + a_nx^{m}}{b_0x^{n+m} + ... + b_mx^{n}} \mid b_0 \neq 0 
		\rbrace \\
		&=
		\lbrace 
		\frac{u(x)}{v(x)} \mid u(x) v(x) \in k[x] \text{ and } \deg u(x) \leq \deg v(x)
		\rbrace  \\
		&= \OV_\infty
	\end{align*} 
\end{proof}


	\section{Divisors}
		\begin{definition} \label{Divisor}
	Let $F/k $ be a function field over a 
	field $k$. The divisor group of $F/k$ is defined 
	as the free abelian group which is generated by 
	the places of $F/k$, denoted $\text{Div}(F)$. The 
	elements of $\text{Div}(F)$ are called divisors of $F/k$. 
	In other words; a divisor is a formal sum;
	$$ D = \sum_{P \in \PF } n_P P $$ with $n_P \in \Z$, 
	almost all $n_P = 0$. Two divisors $D = \sum n_P P$ and
	$D' = \sum n'_P P $ are added coefficientwise;
	$$ D + D' = \sum_{p \in \PF } (n_p + n'_p) P$$ The 
	zero element of the divisor group $\text{Div}(F)$ is the divisor;
	$$ 0 := \sum_{p \in \PF} r_P P$$ where all $r_P =  0$. 
	For all $Q \in \PF $ and $D \in \text{Div} (F)$ we define
	$v_Q(D) := n_Q$.
\end{definition}

\begin{example}
	Consider the rational function field $\Comp(x)/\Comp$. Since 
	$\Comp $ is algebraically closed, we may identify the places
	of $\Comp(x)/\Comp$ with $\Comp \cup \lbrace \infty \rbrace $
	as follows: Let $P$ be a place of $\Comp(x)/\Comp$.
	By theorem \ref{noPlaceLikeHome}, if $P$ is not the infinity 
	place, then it can be indentified with 
	a irreducible polynomial $p$ in $\C[x]$. 
	Since $\C$ is algebraically closed,
	$p = x - a$ for some $a \in \Comp$. Through this, 
	all non-infinity places maybe be identified with some 
	$a \in \Comp$. We identify $P_\infty$ with $\infty$.
	In this view, $3(i) + \sqrt{3}i(\infty) \in \text{Div}(\Comp(x))$
\end{example}

\begin{example}
	Let $F/k$ be a function field and $D = \sum_Pn_PP $ where 
	$n_P = 1 $ for all $P\in \PF$. Then $D  \notin \text{Div}(F)$
	since it has infinitely many nonzero coefficients.  
\end{example}

\begin{lemma} \label{finZeroPole}
	Let $F/k$ be a function field. Every $z \in F$ has finitely
	many zeros and finitely many poles.
\end{lemma}

\begin{proof}
	See Stichtenoth corollary 1.3.4. for proof. 
\end{proof}

\begin{definition} \label{divValuation}
	Let $F/k $ be a function field. 
	A partial ordering on $\text{Div}(F)$ is defined by;
	$$ D_1 \leq D_2 \Leftrightarrow v_P(D_1) \leq v_P(D_2) \text{ for all } P\in \PF $$ 
	A divisor $D \geq 0 $ is called possitive (or effective). 
	The degree of a divisor is defined as;
	$$ \text{deg}(D) := \sum _{P \in \PF}v_P (D) \cdot \deg P  $$ 
	which is a homomorphism $\text{Div}(F) \stackrel{\text{deg}}{\longrightarrow} \Z $. 
	By Lemma \ref{finZeroPole} any nonezero element $x \in F$ 
	has finitely many zeros and poles in $\PF$. Thus this 
	definition makes sense.
\end{definition}

\begin{definition} \label{princeDiv}
	Let $F/k $ be a function field. 
	Let $ 0 \neq x \in F$ and denote by $Z$ (respectively $N$) 
	the set of all zeros (repectively poles) of $x$ in $\PF$. 
	Then we define
	$$ (x)_0 := \sum _{P \in Z} v_p(x) P$$ $$ (x)_{\infty} :=  
	\sum_{P\in N} (-v_P(x))P $$ $$ (x) := (x)_0  - (x)_{\infty} $$ 
	where $(x)_0,(x)_{\infty} $, and $(x)$ are called the 
	zero divisor of $x$, the pole divisor of $x$ and the 
	principal divisor of $x$ respectively. 
\end{definition}

\begin{example}
	Let $F = \mathbb{C}(x)/\mathbb{C}$ be the rational function field
	and consider the polynomial $f = x^3(x+1) \in F$. Recall from example
	\ref{firstencounterwithpolesanddivisors} that the
	valutions of $f$ at places $P_x$,$P_{x+1}$ and $P_\infty$ were $3,1$ 
	and $-4$ respectively. Let $p$ be a monic irreducible polynomial in
	$\mathbb{C}[x]$ other than $x$ and $x + 1$. Consider the place $P= P_p$. 
	Suppose $p \mid f $, then $f = ph$ for some monic $h \in \mathbb{C}[x]$. 
	So $p = x^n(x+1)^m$ and $h=x^r(x+1)^s$ for some $n,m,r,s \in \mathbb{Z}^+$ 
	such that $n+r = 3$ and $m+s = 1$. Since $p \neq x$, $p \neq x+1$ and $p$ is 
	irreducible, it follows that $n = m = 0$ and $h =f$. Therefore
	$f = p^0x^3(x+1)$ and $p \nmid f$, so $(x^3(x+1))^{-1} \in \mathcal{O}_P$. 
	Furthermore $x^3(x+1) \in \mathcal{O}_P$, since $p \nmid 1$. 
	Which implies $ x^3(x+1) \in \mathcal{O}_P^*$ and $v_P(f) = 0$. This imples, by theorem
	\ref{noPlaceLikeHome}, that the only zeros of $f$ in $\Comp(x)/ \Comp$
	are the places $P_{x}$ 
	and $P_{x+1}$, while the only pole of $f$ in $\Comp(x)/ \Comp$
	is $P_\infty$. Hence for 
	the element $f \in \Comp(x) / \Comp$ we obtain divisors; 
	$$(f)_0 = 3P_x + P_{x+1}$$
	$$(f)_{\infty} = 4P_{1/x}$$ 
	$$(f) = 3P_x + P_{x+1} - 4P_{1/x}$$
\end{example}

\begin{definition} \label{divPrinc}
	Let $F/k $ be a function field. 
	The set of divisors; $$ Princ(F) : = \lbrace (x) \mid 0 \neq x \in F \rbrace  $$ 
	is called the group of principal divisors of $F/k$.
\end{definition}

\begin{example} 
	Again, consider the rational function field $ F = \Comp(x)/\Comp $. 
	Let $f \in \mathbb{C}[x] \backslash \lbrace 0 \rbrace  $ 
	and suppose we know the prime factorization of
	$f = a p_1^{e_1}p_2^{e_2}...p_n^{e_n}$ for
	$e_1,e_2,...,e_n \in \mathbb{N} \text{, } a\in \mathbb{C}$ and $p_1,p_2,...,p_n$ 
	distinct monic irreducible polynomials in
	$\mathbb{C}[x]$. Denote the place of $\mathbb{C}(x)$ at prime
	$p_i$ by $P_i = P_{p_i}$ for $i = 1,2,...n$. Then at places
	$P_1,P_2,...,P_n \in \mathbb{P}_F$, $f$ has
	valuation $v_{P_i}(f) = e_i $ for $i=1,2,...,n$. To verify 
	this claim it suffices
	to show that $fp_i^{-e_i} \in \mathcal{O}_{P_i}^*$ for 
	all $i = 1,2,...n$. Since $p_1,p_2,...,p_n$ are distinct
	irreducible polynomials in $\mathbb{C}[x]$, it follows that
	$p_i \nmid p_j $ for all $j \neq i$, thus
	$v_{p_i}(fp_i^{-e_{i}}) \leq 0$ for all $i = 1,2,...,n$.
	Since $e_1,e_2,...,e_n \geq 0$, it follows that
	$v_{p_i}(fp_i^{-e_{i}}) \geq 0$, and thus
	$v_{p_i}(fp_i^{-e_{i}}) = 0$ for all $i=1,2,...,n$. 
	By Theorem \ref{primeValuation} part \ref{valutionringbyvaluations},
	$fp_i^{-e_i} \in \mathcal{O}_{P_i}^*$ 
	for all $i=1,2,...,n$. From theorem \ref{noPlaceLikeHome}, 
	we get that besides the infinity place, 
	these are the only zeros of $f$: for any other 
	zero would have to be at a place corresponding to a 
	irreducible polynomial not in the representation of $f$ 
	and thus would have valuation $0$. Hence the $f$ has zero 
	divisor $$ (f)_0 = \sum^n_{i=1} e_i P_i $$ We calculate 
	the valutation at the infinity place $P_{\infty}$;
	$$ v_{\infty}(f) = \deg (1) - \deg(a p_1^{e_1}p_2^{e_2}...p_n^{e_n}) = 0 - \sum^n_{i=1} e_i \cdot \deg(p_i)$$ 
	Since $p_1,p_2,...,p_n \in \mathbb{C}[x]$, 
	they all have degree 1,  $v_{\infty}(f) = \sum^n_{i =1 } e_i$. 
	So
	$$ (f)_{\infty} = (\sum^n_{i=1} e_i )P_{\infty}$$
	$$(f)= \sum^n_{i=1} e_i P_i -   (\sum^n_{i=1} e_i) P_{\infty} $$ 
	To calculate the degree of $(f)_0,(f)_{\infty},(f)$, we need to find the
	degrees of the places $P_1,P_2,...,P_n$ and $P_{\infty}$. That is,
	calculate $\deg P_i = [F_{P_i} : \mathbb{C}] = [\mathcal{O}_{P_i}/ P_i] : \mathbb{C}]$ 
	for $i = 1,2,...,n $ and $\deg P_{\infty}$. 
	By proposition \ref{propaboutrationals} part \eqref{isoofresidueraitonal},
	$F_{P_i} = \mathcal{O} _ {P_i} / P_i \cong \mathbb{C}[x]/(p_i) $ for all
	$i = 1,2,...,n$. Since each $p_i$ is linear,
	$[\mathbb{C}[x]/(p_i) : \mathbb{C}] = 1$ for all
	$i = 1,2,...n $. Part \eqref{intinitydegree1} of proposition
	\ref{propaboutrationals} states
	that $degP_{\infty} = 1$, hence;
	$$\deg (f)_0 = \sum^n_{i=1} e_ i \cdot \deg P_i =  
	\sum^n_{i=1} e_i = \deg _{\mathbb{C}[x]}(f)$$
	$$\deg (f)_{\infty} = (\sum^n_{i=1} e_i ) \cdot P_{\infty} = 
	\sum^n_{i=1} e_i = \deg _{\mathbb{C}[x]}(f) $$
	$$ \deg (f) =  \sum^n_{i=1} e_i - \sum^n_{i=1} e_i  = 0$$
	So the degree of every principal divisor of a polynomial in $\Comp[x]$ has
	degree 0. Theorem \ref{Princ0} will generalize this result.
\end{example} 

\begin{definition} \label{RRdef}
	Let $F/k $ be a function field. 
	For a divisor $A \in Div(F)$ we define the Riemann-Roch space 
	associated to A by
	$$ \LL(A) := \lbrace x \in F \mid (x) \geq - A \rbrace  $$
\end{definition}

\begin{lemma} \label{RRisVectorSpace}
	Let $F/k $ be a function field.
	Let $A\in Div(F)$. Then 
	$\LL(A)$ is a vector space over $k$.
\end{lemma}

\begin{proof}
	Let $x,y \in \LL(A)$. Then $v_P(x) \geq - v_P(A) $ and 
	$v_P(y) \geq - v_P(y)  $ for all $P\in \PF$. Suppse $v_P(x) < v_P(y)$
	for all $P \in \PF$. Then
	$v_P(x+y) = min \lbrace v_P(x),v_P(y) \rbrace  = v_P(x) \geq v_P(A)$. 
	So $x + y \in \LL(A)$. Let $a\in k$. 
	Then $v_P(ax) = v_P(a) + v_P(x) = 0 + v_P(x) = v_P(x) \geq - v_P(A)$
	for all $P\in \PF$.
\end{proof}

\begin{definition}
	Let $F/k $ be a function field. 
	For a divisor $A \in Div(F)$ the integer $ \ell (A) := dim \LL (A) $ 
	is called the dimension of the divisor $A$.
\end{definition}

\begin{example} \label{ellipticRR}
	Let $p = y^2 + x^3 - x$ and consider the integral domain
	$A$ as in example
	\ref{elliptic}. Let $\pi : \Comp[x,y] \to A$ be the 
	canonical homomorphism of the quotient ring.
	For simplicity, define $x := \pi(x)$ and $y := \pi(y)$. 
	Then $A = \Comp [x,y]$ where $x,y \in A$ satisfy $y^2+x^3-x=0$.
	Exercise \ref{irriduciblePoverC} shows that 
	$A$ is a free $\Comp[x]$-module, with basis $\{ 1, y \}$. 
	So each element of $A$ has a unique expression of the form
	$p(x)y + q(x)$ where $p(x), q(x)$ are polynomials in $x$. 
	From exercise \ref{irriduciblePoverC} we also know that $F/\Comp$ 
	is a function field over $\Comp$. Let $\VVV = \VVV( F/ \Comp )$ 
	be the set of valuation rings of $F/ \Comp$ 
	and $\PPP = \PPP( F/ \Comp )$ the set of places. 
	If we make the following assumptions;
	\begin{itemize}
		\item There is exactly one element $\Oeul \in \VVV$ such that
		$A \nsubseteq \Oeul$. Denote it by $\Oeul_\infty$,
		let $P_\infty$ be its maximal ideal, and let
		$v_\infty : F^* \to \Integ$ be its valuation.
		\item $v_\infty(x) = -2$
		\item \label{Aisintersection}$A$ is equal to the intersection
		 of all rings $\Oeul \in \VVV \setminus \{ \Oeul_\infty \}$.
	\end{itemize}
	Then we have the following;
	\begin{enumerate}[(a)]
		\item If $f \in F^*$ then $v_P(f) \ge 0$ for all
		 $P \in \PPP \setminus \{ P_\infty \}$ $\iff$ $f \in A$.
		\item $v_\infty(y)=-3$
		\item For any $f = p(x)y + q(x) \in A$, let $m=\deg_x( p(x) )$ and $n = \deg_x( q(x) )$. 
		\begin{enumerate}[(i)]
			\item $v_\infty( p(x)y ) = -2n - 3$
			\item $v_\infty( q(x) ) = -2m$
			\item $v_\infty( p(x)y ) \neq  v_\infty( q(x) )$
			\item $v_\infty(f) = - \max \lbrace 2n + 3, 2m \rbrace $
		\end{enumerate}
		\item Let $N \ge 2$. Then $\Leul( 2N P_\infty )$ has basis 
		$\lbrace y, xy, x^2y,...,x^{N-2}y,1,x,x^2,...,x^N \rbrace $
		and dimension $\ell( 2N P_\infty ) = 2N$.
	\end{enumerate}
\end{example}

\begin{proof}
\begin{enumerate}[(a)]
	\item \label{inAiff} Let $f\in F^*$. $v_P(f) \geq 0 $ 
	for all $P\in \mathbb{P}_F \backslash \lbrace P_\infty \rbrace $ 
	if and only if
	$f \in O_P$ for all
	$P\in \mathbb{P}_F \backslash \lbrace P_\infty \rbrace $ 
	if and only if $f \in A$ by assumption \ref{Aisintersection}.

	\item Recall lemma \ref{strongtriange}. Then 

	\begin{align*}
		2v_\infty(y) &= v_\infty(y^2) \\
		&= v_\infty(x - x^3) \\
		&= v_\infty(x) + v_\infty(1+x) + v_\infty (1-x) \\
		&= v_\infty(x) + \min \lbrace v_\infty(1), v_\infty(-x) \rbrace + \min \lbrace v_\infty(1) , v_\infty(x) \rbrace \\
		&= (-2) + (-2) + (-2) \\
		&= -6
	\end{align*}

	Hence $v_\infty(y) = -3 $

	\item Since $p,q$ are polynomials in $\Comp [x]$, we may 
	write them as $p_1^{e_1}p_2^{e_2}...p_s^{e_s}$ and
	$q_1^{f_1}q_2^{f_2}...q_t^{f_t}$ respectively, 
	where $p_1,...,p_s,q_1,...q_t$ are linear polynomials 
	in $\Comp[x]$ and $e_1,...,e_s,f_1,...,f_t \in \Integ$. Hence 

	\begin{align*}
		v_\infty(py) &=v_\infty(p_1^{e_1}p_2^{e_2}...p_s^{e_s}y) \\
		&= e_1v_\infty(p_1) + ... + e_sv_\infty(p_s) + v_\infty (y)  \\
		&= e_1(-2) + ... + e_s(-2) - 3 \\
		&= -2n - 3
	\end{align*}

	\begin{align*}
		v_\infty(q) &=v_\infty(q_1^{f_1}q_2^{f_2}...q_t^{f_t}) \\
		&= f_1v_\infty(q_1) + ... + f_tv_\infty(p_t)  \\
		&= f_1(-2) + ... + f_t(-2) \\
		&= -2m
	\end{align*}

	Notice that 

	\begin{align*}
		v_\infty(py)=v_\infty(q) & \Longrightarrow -2n - 3 =  -2m  \\
		& \Longrightarrow m = n + 3/2
	\end{align*}

	Which is impossible since $q \in \Comp [x]$. 
	Hence $v_\infty(py) \neq v_\infty (q) $. Therefore 


	\begin{align*}
		v_\infty (f) &= v_\infty( p(x)y + q(x) ) \\
		&= \min \lbrace  v_\infty( p(x)y),  v_\infty( q(x) ) \rbrace \\
		&= \min \lbrace -2n - 3, -2m \rbrace  \\
		&= - \max \lbrace 2n + 3, 2m \rbrace 
	\end{align*}

	\item Let $N\geq 2$ and write $f = py + q$. $f \in \Leul (2N P_\infty ) $ 
	if and only if $v_P(f) + v_P(2N P_\infty) \geq 0 $ 
	for all $P\in \mathbb{P}_F$ by definition. 
	By part \eqref{inAiff}, we know that $v_P(f) \geq 0 $ 
	for all $P\in \mathbb{P}_F \backslash P_\infty$ if and 
	only if $f \in A$. Since $v_P(2NP_\infty) = 0 $ for 
	all $P \in \mathbb{P}_F \backslash P_\infty$, we 
	require $v_P(f) \geq 0 $ for 
	all $P \in \mathbb{P}_F \backslash P_\infty$. 
	Thus $f \in A$. Hence

	\begin{align*} 
		\Leul (2N P_\infty ) &= \lbrace f \in A \mid \max \lbrace 2\deg(q), 3 + 2\deg(p) \rbrace \leq 2N \rbrace \\
		 &= \lbrace  py + q \in A \mid \deg(p) \leq N-3/2, \deg(q) \leq N \rbrace \\
		 &= \lbrace  py + q \in A \mid \deg(p) \leq N - 2, \deg(q) \leq N \rbrace 
	\end{align*}

	\noindent Hence $\lbrace y, xy, x^2y,...,x^{N-2}y,1,x,x^2,...,x^N \rbrace $ 
	is a basis for $\Leul (2N P_\infty )$ 
	over $\Comp$ and $\ell( 2N P_\infty )= (N - 1) + N + 1 = 2N$. 

\end{enumerate}
\end{proof}

\begin{example} \label{RRis0}
	Let $F/k$ be a function field over $k$ and $A \in \text{Div}(F)$.
	We have $\LL(0) = k$ and if $A < 0$ then $\LL(A) = \lbrace 0 \rbrace$.
\end{example}


\begin{proof}
	To show the first assertion, let $0 \neq x \in k$, then $(x)=0$. So 
	$x \in \LL(0)$. Let $0 \neq x \in \LL(0)$.
	Then $(x) \geq 0$, but then $x$ has no pole, so by corollary \ref{everyonesgotapole},
	so $x \in k$. Suppose $A < 0$ and let $ 0 \neq x \in \LL(A)$. Then $(x) \geq - A > 0$, 
	but then $x $ has at least one zero and no pole. This is impossible. Hence $x = 0$.
\end{proof}


\begin{proposition} \label{Dimofthequotient}
	Let $A,B$ be two divisors of $F/k$ with $A \leq B$. 
	Then we have $\LL(A) \subseteq \LL(B)$ and $\text{dim}(\LL(B)/\LL(B)) \leq \deg B - \deg A$. 
\end{proposition}

\begin{proof}
	Assume that $A,B$ be two divisors of $F/k$ with $A \leq B$.  
	We show $\LL(A) \subseteq \LL(B)$. Let $x \in \LL(A)$, then 
	$v_P(x) + v_P(x) \geq v_P(x) + v_P(B) \geq 0 $, so $x \in \LL(B)$.
	Hence $\LL(A) \subseteq \LL(B)$. 
	To vertify the second claim, assume that 
	$B = A + P $ for some $P\in \PF$. This is possible since 
	the general case follows by induction. Let $t \in F $ such that 
	$v_P(t) = v_P(B) = v_P(A) + 1$. For $x \in \LL (B)$ we have 
	$v_P(x ) \geq - v_P(B) = - v_P(t)$, so $xt\in \OV_P$. So we obtain
	a $k-linear$ map $ \phi : \LL(B) \Longrightarrow F_P$, $x \mapsto xt + P$. 
	An element $x$ is in the the kernel of $\phi $ if and only if 
	$v_P(xt) > 0$, that is $v_P(x) \geq - v_P(A)$. So $\text{ker} \phi = \LL(A)$.
	Thus $\phi $ induces an \textit{injective} $k$-linear map from $\LL(B)/ \LL(A)$ to $\FP$. 
	Therefore $\text{dim} \LL(B)/ \LL(A) \leq \text{dim} F_P = \deg B - \deg A$.
\end{proof}

\begin{lemma} \label{lessthanextensiondeg}
	Let $F/k$ be a function field and let $P_1,...,P_r$ be 
	zeros of the element $x \in F$. Then $ \sum^r_{i=1} v_{P_i}(x) \leq [F:k(x)]$.
\end{lemma}

\begin{proof}
	See Stichtenoth Proposition 1.3.3.
\end{proof}

\begin{theorem} \label{Princ0}
	All principal divisors have degree zero. More precisely, 
	let $x \in F \backslash k$ and $(a)_0$ resp. $(a)_{\infty}$ 
	denote the zero resp. pole divisor of $x$. 
	Then $$deg(x)_0=deg(x)_{\infty}=[F:k(x)]$$. 
\end{theorem}


\begin{proof}
	Let $n := [F:k(x)]$. Then $\deg (x)_\infty  \leq n$ by 
	\ref{lessthanextensiondeg}, we have 
	$\sum^r_{i=1 }v_{P_i}(x) \leq [F:k(x)]$.
	Thus it remains to show that $n \geq \deg (x)_\infty $. 
	Let $v_1,...v_n$ as a basis for $F/k(x)$. Let $A \geq 0$ 
	be a divisor such that $(v_i) \geq A$ for $i = 1,..,n$.
	Then we have $\Leul(r(x)_\infty + A ) \geq n(r +1 )$ for 
	all $r \geq 0$, since $x^iv_j \in \LL (r (x)_\infty + 1)$ 
	for $0 \leq i \leq r$, $1 \leq j \leq n$. Letting $c: = \deg A$
	we get $n(r + 1 ) \leq \Leul (r(x)_\infty +A) \leq r \cdot (x)_\infty + c + 1 $. 
	Thus $r(\deg (x)_\infty - n )  \geq n - c - 1 $ 
	for all $r \in \Nat$. Hence $\deg (x)_\infty \geq n$.
\end{proof}

\begin{proposition}
	There is a constant $\gamma \in \Z $ such that for all 
	divisors $A \in Div(F)$ the following holds:
	$$ degA - l(A) \leq \gamma $$
\end{proposition}

\begin{proof}
	First observe that by proposition \ref{Dimofthequotient}, 
	$A_1 \leq A_2  \Rightarrow \deg A_1 - \Leul(A_1) \leq \deg A_2 - \Leul(A_2)$.
	Let $x \in F \backslash k$ and consider the divisor $(x)_\infty$. 
	There exists a divisor $C \geq 0 $ such that
	$\Leul (r(x)_\infty + C ) \geq (r + 1) \cdot \deg (x)_\infty$ for all 
	$r \geq 0$. We also have $\Leul (r(x)_\infty + C) \leq \Leul(r(x)_\infty) + \deg C$
	from proposition \ref{Dimofthequotient}. 
	Hence $\Leul(r(x)_\infty) \geq (r+1) \cdot \deg (x)_\infty 
	- \deg C = \deg (r(x)_\infty ) + ([F : k(x)] - \deg C) $. 
	Hence $\deg (r(x)_\infty ) - \Leul (r(x)_\infty ) \leq \gamma$ for all $r > 0$
	with some $\gamma \in \Integ$. 
	\textit{Claim}: For all $A \in \text{Div}(F)$, there exists divisors 
	$A_1,D$ and a integer $r \geq 0$ such that $A \leq A_1$, $A = D + P$ 
	for some $P \in \PF$ and $D \leq r(x)_\infty$. \textit{Proof of claim:}
	Let $A_1 \geq A $ such that $A_1 > 0 $. 
	Then $\Leul (r (x)_\infty - A_1) \geq \Leul (r(x)_\infty ) - 
	\deg A_1 \geq \deg (r(x)_\infty ) - \gamma - \deg A_1 > 0 $
	for sufficiently large $r$. Thus there exists some nonzero element 
	$z \in \LL ( r(x)_\infty - A_1)$. Letting $D:= A_1 - (z)$, we 
	obtain $A = D + P $ where $P + -(z)$ and
	$D \leq A_1 - (A_1 - r(x)_\infty ) = r(x)_\infty$ as desired.
	Thus the claim is verified. 
	From this, observe that
	$\deg A - \Leul (A) \leq \deg A_1 - \Leul (A_1) = \deg D 
	- \Leul (D) \leq \deg(r(x)_\infty ) - \Leul (r(x)_\infty ) \leq \gamma $.
\end{proof}



\begin{definition}
	Let $F/k $ be a function field.
	The genus of $g$ of $F/k$ is defined 
	by $$ g := max \lbrace degA - l(A) + 1 \mid A \in Div(F) \rbrace $$
\end{definition}


\begin{theorem}[Riemann's Theorem] \label{Riemannstheorem}
	Let $F/k$ be a function field of genus $g$. 
	Then there exists an integer $c$, depending only on the
	function field $F/k$, such that $l(A) = degA + 1 - g $ whenever $degA \geq c$.
\end{theorem}


\begin{proof}
	Let $A_0$ such that $g =	 \deg A_0 \Leul (A_0) + 1$ and 
	set $c = \deg A_0 + g$. If $\deg A \geq c$ then 
	$\Leul (A-A_0 ) \geq \deg (A-A_0) + 1  - g \geq c - \deg A_0 + 1 - g = 1$.
	So there is an element $0 \neq z \in \LL(A - A_0)$. 
	Consider the divisor $A^{\prime} =  A + (z) \geq A_0$. 
	We have
	$\deg A^{\prime } - \Leul (A^{\prime}) \geq \deg A_0 - \Leul (A_0) = g -1$.
	Hence $ \Leul (A) \leq \deg A + 1 - g$.
\end{proof}


\begin{example}
	Recall the setup of example \ref{ellipticRR}. Let $N > 0$ be 
	arbitrarly large. We know that 
	$\Leul (2NP_\infty) = 2N$. Assume $\deg P_\infty = 1$.
	Then by Riemann's Theorem, $g =\deg(2NP_\infty)  - \Leul (2NP_\infty) + 1 = 1$. 
	Then we may conclude that $F/\Comp$ has genus $1$, 
	that is the curve $p = y^2 + x^3 - x$ has genus $1$. Similary, 
	from exercise \ref{infinity}, we know that $\Leul (NP_\infty) = N + 1$, 
	where $P_\infty$ denotest the infinity place of the function field $k(x)/k$.
	From proposition \ref{propaboutrationals}, $\deg P)\infty = 1$. Thus by 
	Riemann's theorem $g = \deg(NP_\infty ) - \Leul (NP_\infty) + 1 =N - N-1  + 1= 0$
	Hence the rational function field has genus $0$.
\end{example}



	\section{Dicriticals}
		\begin{lemma} \label {i9283hd01bxdo912ep01}
	Let $E/k$ be a field extension and $u,v \in E$. Then we have field extensions: $$
	\xymatrix@R=0pt@C=6pt{
	& E  \\
	\\
	& k(u,v) \ar @{-} [dl] \ar @{-} [dr] \ar @{-} [uu] \\
	k(u) && k(v) \\
	& k \ar @{-} [ul] \ar @{-} [ur]
	}
	$$
	Assume that each of $u,v$ is transcendental over $k$. 
	Then the following are equivalent:
\begin{itemize}
	\item \label{con1} $(u,v)$ is algebraically dependent over $k$;
	\item \label{con2} $v$ is algebraic over $k(u)$;
	\item \label{con3} $u$ is algebraic over $k(v)$.
\end{itemize}
\end{lemma}

\begin{proof}
	$\eqref{con1}  \Longrightarrow \eqref{con2} $ Assume $(u,v)$ 
	are algebraically dependent over $k$. Then there 
	exists $$ f(X,Y) = \sum_{i,j \in \N} a_{ij} X^iY^j \in 
	k[X,Y] \backslash \lbrace 0 \rbrace $$ such that
	$f(u,v) = \sum_{i,j \in \N} a_{ij} u^iv^j = 0 $. 
	Consider $$g(Y) = \sum_{i,j \in \N} a_{ij} u^iY^j 
	\in k[u][Y] \backslash \lbrace 0 \rbrace $$ We still 
	have $g(v) = \sum_{i,j} a_{ij} u^iv^j = 0 $. 
	Hence $v$ is algebraic over $k[u] \subset k(u)$. 


	$\eqref{con2} \Longrightarrow \eqref{con3} $; Assume $v$ is 
	algebraic over $k(u)$. Then there exists
	$$ f(Y) = \sum_{i\in \N} a_i(u)Y^i \in k(u)[Y] \backslash \lbrace 0 \rbrace $$ 
	such that $f(v) = 0$. Since $a_i(u) \in k(u)$ for 
	all $i$, $a_i(u) = \frac{f_i(u)}{g_i(u)} $ 
	where $f_i(u),g_i(u) \in k[u]$ and $g_i(u) \neq 0 $. 
	Let $$a(u) := \prod_i g_i(u) \in k[u] \backslash \lbrace 0 \rbrace $$ 
	Then consider $$ g(Y) := a(u)f(Y) = a(u)\sum_{i\in \N}
	a_i(u)Y^i = \sum_{i \in \N}a(u) a_i(u)Y^i $$ 
	Note that $h_i(u):=a(u)a_i(u) \in k[u] $ for all $i$ 
	and we still have $g(v) = 0 $. We can rewrite $g(Y)$ 
	as $$ g(X) = \sum_{i \in \N} h_i(X)v^i \in k[v][X] \backslash \lbrace 0 \rbrace $$ 
	and we still have  $g(u) = \sum_{i \in \N} h_i(u) v^i = 0 $. 
	Hence $u$ is algebraic over $k[v] \subset k(v)$ 

	$\eqref{con3} \Longrightarrow \eqref{con1}$; 
	Assume $u$ is algebraic over $k(v)$. Then there 
	exists $$ f(X) = \sum_{i \in \N } a_i(v)X^i \in k(v)[X] 
	\backslash \lbrace 0 \rbrace $$ such that $f(u) = 0$. 
	Since $a_i(v) \in k(v) $ for all $i$, $a_i(v)= \frac{f_i(v)}{g_i(v)}$ 
	where $f_i(v),g_i(v) \in k[Y]$ and $g_i(v) \neq 0 $. 
	Let $$a(v) := \prod_i g_i(v) \in k[v] \backslash \lbrace 0 \rbrace  $$ 
	Then consider $$ g(X) = a(v)f(X) = a(v)\sum_{i \in \N} a_i(v)X^i 
	= \sum_{i \in \N} a(v)a_i(v)X^i $$ Note that
	$h_i(v):=a(v)a_i(v) \in k[v] $ for all $i$ and we still 
	have $g(u) = 0 $. We can rewrite $g(Y)$ as
	$$ g(X,Y) = \sum_{i \in \N} h_i(Y)X^i \in k[X,Y] 
	\backslash \lbrace 0 \rbrace  $$ We still have
	$g(X,Y) = \sum_{i \in \N} h_i(v) u^i = 0 $. 
	Hence $(u,v)$ is algebraically independent over $k$.
\end{proof}

\begin{lemma} \label {z0cb9182wbxi81f6}
	Let $k$ be a field and let $k(x_1, \dots, x_n)$ be 
	the field of fractions of the
	polynomial ring  $k[x_1, \dots, x_n]$ in $n$ variables 
	over $k$ (where $n \ge 1$).
	Then $k$ is algebraically closed in $k(x_1, \dots, x_n)$.
\end{lemma}

\begin{proof}
	Induction on $n$. When $n= 1$, then we are considering 
	the rational function field in one variable over $k $.
	Let $P$ be a place of $k(x)/k$ of degree $1$. That is
	$P:=P_{x-a}$ for $a \in k$. The algebraic closure of 
	$k$, denoted $\tilde{k}$, embeds into $k(x)_P$, since 
	$P \cap \tilde{k} = \lbrace 0 \rbrace$. Then we have 
	$k \subseteq \tilde{k} \subseteq k(x)_P = k$. 
	Hence $k$ is algebraically closed in $k(x)$. 
	Assume that $n>1$.
	Inductive hypothesis: $k$ is algebraically closed 
	in $k(x_1, \dots, x_{n-1})$.
	To prove that $k$ is algebraically closed in $k(x_1, \dots, x_n)$, we consider
	an element $w$ of $k(x_1, \dots, x_n)$ that is algebraic 
	over $k$. We have to show that $w \in k$.
	Observe that $w$ is algebraic over $k(x_1, \dots, x_{n-1})$.
	Write $F = k(x_1, \dots, x_{n})$ and $K = k(x_1, \dots, x_{n-1})$.
	Then $F/K$ is the rational function field of one variable, 
	so $K$ is algebraically closed in $F$, so $w \in K$.
	As $w \in k(x_1, \dots, x_{n-1})$ is algebraic over $k$, 
	the inductive hypothesis implies that $w \in k$.
\end{proof}



For this section, let $ k$ be a field, $A =  k[x,y]$ 
the polynomial ring in two variables over $ k$,
and $L= \Frac A =  k(x,y)$, the field of rational functions in two variables.
The objects $ k$, $A$ and $L$ are fixed throughout.
For each choice of $F \in A \setminus  k$, 
we may consider the subfield $K= k(F)$ of $L$
(since $F$ is an element of the field $L =  k(x,y)$, 
it follows that $ k(F)$ is a subfield of $L$).
We have $ k \subset K \subset L$ where $ k$ 
and $L$ are always the same but $K$ depends on the choice of $F$.
We are particularly interested in the field extension $L/K$.
Our first objective is to show that $L/K$ is a function field 
of one variable.
There are several steps in the proof of this.

\begin{remark}
	By lemma~\ref{z0cb9182wbxi81f6}, $ k$ is algebraically closed in $L$.
	However, whether or not $K$ is algebraically closed in $L$ depends on the choice of $F$.
	For instance, if $F=x$ then $L/K$ is the rational function field of one variable, so $K$ is algebraically closed 
	in $L$ in this case.  But if $F = x^2$ then $x$ is an element of $L$ that is algebraic over $K$ but that does not 
	belong to $K$, so $K$ is not algebraically closed in $L$ in this case.
\end{remark}

\begin{proposition} \label {90cdk2938db129}
	Show that the following are equivalent:
	\begin{enumerate}[(i)]
	\item \label{fact1} $(F,x)$ is algebraically dependent over $ k$;
	\item \label{fact2} $x$ is algebraic over $K$;
	\item \label{fact3} $F$ is algebraic over $ k(x)$;
	\item \label{fact4} $F \in  k(x)$;
	\item \label{fact5} $F \in  k[x]$.
	\end{enumerate}
\end{proposition}

\begin{proof}
	Since $A$ is a polynomial ring in two variables $x,y$, by definition $(x,y)$ 
	is algebraically independent over $ k$, so we have that $x$ is transcendental 
	over $ k$. Suppose $F$ is algebraic over $ k$, 
	then $F \in  k$ by \ref{z0cb9182wbxi81f6}, 
	and this contradicts the hypothesis $F \in A \setminus k$.
	Therefore we may use lemma \ref{i9283hd01bxdo912ep01} 
	to show that \ref{fact1}, \ref{fact2} and \ref{fact3} are all equivalent.
	(\eqref{fact3} $\Rightarrow $ \eqref{fact4}): 
	Since $L/ k(x)$ is the rational function field 
	in one variable, it follows that $ k(x)$ is algebraically closed
	in $L$.
	(\eqref{fact4} $\Rightarrow $ \eqref{fact5}): 
	$ F \in  k[x,y] \backslash  k$ by definition. 
	Hence if $F\in  k(x) $. We want to show that 
	$ k(x) \cap  k[x,y] =  k[x]$. 
	It is clear that $ k[x] \subseteq k(x) \cap  k[x,y]$.
	Consider an element $\xi \in  k(x) \cap  k[x,y]$.
	Since $\xi \in  k(x)$, we may write $\xi = fg^{-1}$ for
	$f,g\in  k[x]$ and $g \neq 0$. Thus $g \in  k[x,y]$. 
	So $g$ must belong to $\Comp$. Hence $ k(x) \cap  k[x,y] =  k[x]$.
	(\eqref{fact5} $\Rightarrow $ \eqref{fact3}):
	If $F \in  k[x]$, then $F$ is algebraic over $ k[x] \subset  k(x)$.
\end{proof}


\begin{remark}
	The result of proposition \ref{90cdk2938db129} remains valid 
	if one replaces all `$x$' by `$y$' in the statement.
	In particular, if $y$ is algebraic over $K$ then $F \in  k[y]$.
\end{remark}

\begin{corollary} \label {9fo13d912o9o0}
	At least one of $x,y$ is transcendental over $K$.
\end{corollary}

\begin{proof}
	Let $F \in A \backslash  k$. Then if $F \notin  k(x)$, then $x$ is 
	transcendental over $K$ by proposition \ref{90cdk2938db129}. Similarly, 
	if $F\notin  k(y)$, then $y$ is transcendental over $K$. 
\end{proof}

\begin{proposition}
	For some $t \in \{x,y\}$, $L/K(t)$ is a finite extension and therefore
	$L/K$ is a function field of one variable.
	Furthermore, $F \in  k[x]$.
\end{proposition}

\begin{proof}
	Suppose that both $L/K(x)$ and $L/K(y)$ are not finite. Then we get the following
	chain of inclusions $ k(x) \subseteq K(x) \subset L$ and $ k(y) \subseteq K(y) \subset L$. 
	Observe that both $L/ k(x)$ and $L/ k(y)$ are function fields in one 
	variable. So the fact that both $L/K(x)$ and $l/K(y)$ are not finite implies 
	that $K(x)/ k(x)$ and $K(y)/ k(y)$ are both algebraic. In particular,
	$F$ is algebraic over $ k(x)$ and $ k(y)$. Hence by proposition \ref{90cdk2938db129}
	$F\in  k[x]$ and $F\in  k[y]$, which is impossible. So for some $t \in \{x,y\}$, $L/K(t)$ is a finite extension.
\end{proof}

\begin{remark}
	So, for any choice of $F \in A \setminus  k$, $L/K$ is a function field of one variable
	(it is important that $F \notin  k$ here; 
	if $F \in  k$, then $ k(F)= k$ and the field extension $L/ k$ has transcendental degree 2).
	The properties of the function field $L/K$ depend on the choice of $F$:
	whether or not $K$ is algebraically closed in $L$ depends on the choice of $F$;
	whether or not $L/K$ is the rational function field depends on the choice of $F$.
\end{remark}

\begin{example} \label {ue0982rhr329r23jew}
	Let $k = \Comp$. Then for each of
	the following values of $F$, the function 
	field $L/K$ is the rational 
	function field.

	\begin{enumerate}
		\item $F = xy^2$
		\item $F = x^2y^3$
		\item $F = x(y+x^3)$ 
		\item $F = y^2 + x^2 - 1$
	\end{enumerate}
	To prove this, it suffices to find $G \in L$ such that $L = K(G)$.
	\begin{enumerate}
		\item $K(y)=\Comp (xy^2,y) = \Comp (xy^2y^{-2},y)=\Comp (x,y) = L$
		\item $K(xy)=\Comp(x^2y^3,xy)=\Comp(x^2y^3x^{-2}y^{-2},xy)=\Comp(y,xy)=\Comp(x,y)=L$
		\item $K(x)= \Comp(x(y+x^3),x)=\Comp(y+x^3,x)=\Comp(y+x^3 - x^3,x)=\Comp(x,y)=L$ 
		\item Let $u = x+iy$ and $v=x-iy$ of $L=\Comp(x,y)$. 
		Notice that $uv=x^2+y^2 = F + 1$.  
		So $K = \Comp(x^2+y^2-1) = \Comp(x^2+y^2) = \Comp(uv)$ and
		consequently $K(v) = \Comp(uv,v) = \Comp(u,v) = L$.
	\end{enumerate} 
\end{example}

\begin{notation}
	Let $F \in A \setminus  k$.
	\begin{enumerate}[(a)]
		\item Let $\VV(F)$ be the set of all valuation rings of the function field $L/K$.
		The notation `$\VV(F)$' reminds us that this set of rings depends on the choice of $F$.
		\item Let $\PP(F)$ be the set of places of $L/K$.
		\item Let $\VV^\infty(F) = \setspec{ R \in \VV(F) }{ A \nsubseteq R }$.
	\end{enumerate}
	Note that $\VV^\infty(F) = \setspec{ R \in \VV(F) }{ \{x,y\} \nsubseteq R }$.
\end{notation}


\begin{proposition} \label {3o9jdb9182so}
	Let $F \in A \setminus  k$.
	\begin{enumerate}[(a)]
	\item \label{notmempty} $\VV^\infty(F)$ is a nonempty set.
	\item \label{finite} $\VV^\infty(F)$ is a finite set.
	\end{enumerate}
\end{proposition}

\begin{proof}
	\eqref{notmempty}: By \ref{9fo13d912o9o0}, 
	we know that at least one of $x,y$ is transcendental 
	over $K$. Assume $x$ is transcendental over $K$, then by 
	corollary \ref{everyonesgotapole}, $x$ has at least one pole. 
	That is, there exists 
	a place $P$ such that $v_P(x)<0$. Let $\OV_P$ be 
	the valuation ring of $L$ corresponding to $P$. Hence $V^{\infty}(F) \neq \emptyset$. 
	\eqref{finite}:
\end{proof}

\begin{definition}
	Let $F \in A \setminus  k$.
	The elements of the nonempty finite set $\VV^\infty(F)$ are called the {\it dicriticals\/} of $F$.
	We define the {\it degree\/} of a dicritical $R$ of $F$ to be $\deg P$, where
	$P$ is the place of $R$.
\end{definition}

\begin{example} \label {982b8d12vxowe9f}
	Let $F=x$. In this case, we have $K= k(x)$; 
	so $L=K(y)$ is the rational function field.
	The place at infinity of $K(y)$ is $R = K[ z ]_{(z)}$ where $z = y^{-1}$;
	since $y \notin R$, we have $R \in \VV^\infty(F)$.  
	The degree of this dicritical is $1$,
	because we know that the place at infinity of $K(y)$ has degree $1$.
	If $R'$ is any valuation ring of $L/K$ other than $R$
	then $R' = K[y]_{(p)}$ for some irreducible polynomial $p \in K[y]$;
	then $x \in K \subseteq K[y]_{(p)}$ and $y \in K[y]_{(p)}$, 
	so $\{x,y\} \subseteq R'$ and $R' \notin  \VV^\infty(F)$.
	Hence  $\VV^\infty(x) = \lbrace K[z]_{(z)} \rbrace$. Since 
	$\deg K[z]_{(z)} = 1$. We say that $F$ has only 1 rational 
	dicritical. 
\end{example}

\begin{example} \label {0ndvx54120cmrnh}
	The polynomial $F=xy$ has two dicriticals, $k[z_1]_{(z_1)}$ 
	and $K[z_2]_{(z_2)}$, where $z_1=x^{-1}$ and $z_2=y^{-1}$. 
\end{example}


\begin{definition}
	Use square brackets to represent unordered lists of positive integers.
	For instance, $[1,1,2] = [1,2,1] = [2,1,1] \neq [1,2,2]$. Let $F \in A \setminus  k$.
	Let $R_1, \dots, R_s$ be the distinct dicriticals of $F$, where $R_i$ is a dicritical of degree $d_i$.
	Then we write $\Delta(F) = [ d_1, \dots, d_s ]$.
\end{definition}

\begin{examples}
	From \ref{982b8d12vxowe9f}, $\Delta(x) = \Delta(y) = [1]$.
	By \ref{0ndvx54120cmrnh}, $\Delta(xy)=[1,1]$.
\end{examples}

	\section{Field Generators}
		\begin{definition}
An integral domain is said to be {\it normal\/} if it is 
integrally closed in its field of fractions.
\end{definition}

\begin{proposition} \label {i12biud012oww}
Every UFD is normal.
\end{proposition}

\begin{proof}
Let $A $ be a UFD and let $F$ be the field of 
fractions of $A$. Let $z \in F$, writen $z = a/b$ such that
$0 \neq b,a \in A$ and $a,b$ share no commmon
primes in their factorizations. If $z$ is integral over $A$. Then 
$z^n + a_{n-1}z^{n-1} + ... + a_1z + a_0 = 0$ for some 
$a_0,a_1,...,a_{n-1} \in A$. That is,
$(a/b)^n + a_{n-1}(a/b)^{n-1} + ... + a_1 (a/b) + a_0 = 0$. 
We may clear denominators to obtain 
$a^n + a_{n-1}a^{n-1}b + ... + a_1 ab^{n-1} + a_0b^n = 0$.
Then $a^n = -b(a_{n-1}a^{n-1} + ... + a_1 ab^{n-2} + a_0b^{n-1})$. 
This means $b$ divides $a$. If $b$ is not a unit,
then this imples that any prime elements
in the factorization of $b$ appear in the factorization of $a^n$, 
and thus also in $a$. This is a contradiction to the supposition
that $a,b$ share no common primes in their factorization. So $b$
must be a unit. Thus $z \in A$. 
\end{proof}

% \begin{lemma}
% Let $A$ be a 
% \end{lemma}


\begin{example}
The ring $A = \Comp[x,y]/(y-x^2)$ is normal: 
Let $\phi : \Comp[x,y] \longrightarrow \Comp[t] $ 
be given by $ f(x,y) \mapsto f(t,t^2)$. 
Then $\phi $ is an onto homomorphism with kernel $(y-x^2)$. 
Thus $\Comp[x,y]/(y-x^2) \cong \Comp[t]$, which is a UFD. 
\end{example}

\begin{example}
The converse is not true in general. If we 
let $A=\Comp[x,y,z]/(xy-z^2)$.
Then $A$ is normal but not a UFD.
To prove this, let $\phi : \Comp[z,y,z] \longrightarrow \Comp[s,t]$ 
given as $f(x,y,z) \mapsto f(s^2,t^2,st)$. Then $\phi$ is homomorphism
an onto the ring $B = \Comp[s^2,t^2,st]$ with kernel $(xy-z^2)$. Hence 
$A \cong B$. We prove that $B $ is normal by 
showing that the integral closure of $\Comp[s,t]$ in 
$\Comp(s,t,\sqrt{st})$ is $\Comp[s,t,\sqrt{st}]$. Let 
$u + v\sqrt{st} \in \Comp(s,t,\sqrt{st}) $ be integral over 
$\Comp[s,t]$ for some $u,v \in \Comp(s,t)$. 
Then since the integral closure of a integral
domain is an integral domain, $u-v \sqrt{st}$ is in the integral
closure of $\Comp[s,t]$ in $\Comp(s,t,\sqrt{st})$ as well. 
Thus their sum, $2u $ belongs to this closure. Since $\Comp[s,t]$ is 
normal, $u \in \Comp[s,t]$. Similarily, $v\sqrt{st} \in \Comp[s,t]$.
Hence $v^2st \in \Comp[s,t], v \in \Comp(s,t)$. Clearly, then 
$v$ can have no denominator, thus $v \in \Comp[s,t]$. Hence 
$u + v \sqrt{st} \in \Comp[s,t]$. To see that $A$ is not a 
UFD, notice that $z^2 = xy$. 
\end{example}

\begin{remark}
Recall that if $R$ is an integral domain and 
$S \subseteq R \setminus \{0\}$ is a multiplicative set
then $S^{-1}R$ is an integral domain and
$R \subseteq S^{-1}R \subseteq \Frac(R)$, so $R$ and $S^{-1}R$ 
have the same field of fractions.
\end{remark}


\begin{lemma} \label {poifj012woid2q9}
Let $R$ be an integral domain and $S \subseteq R \setminus \{0\}$ 
a multiplicative set.
If $R$ is normal then so is $S^{-1}R$.
\end{lemma}

\begin{proof}
Let $R$ be an integral domain (not necessarily normal),
$S \subseteq R \setminus \{0\}$ a multiplicative
set, $K = \Frac R$, and consider 
$ R \subseteq \tilde R \subseteq K $
where $\tilde R$ is the integral closure of $R$ in $K$.
Then Prop.~5.12 of Atiyah-McDonald implies that
$S^{-1} \tilde R$ is the integral closure of $S^{-1}R$ in $S^{-1}K=K$. 
If we now assume that $R$ is normal then $\tilde R = R$, so
$S^{-1} R$ is the integral closure of $S^{-1}R$ in $K$, i.e., 
$S^{-1} R$ is normal.
\end{proof}


\begin{proposition}  \label {0923d091sd0192s}
Let $F/K$ be a rational function field of one variable, 
let $P$ be a place of $F/K$ of degree $1$,
and let $\Oeul_{P}$ be the corresponding valuation ring 
of $F/K$.  Then there exists $t \in F$ satisfying
$F = K(t) $ and $\Oeul_P =  k[ t^{-1} ]_{(t^{-1})} $.
Moreover, for any such $t$,
$K[t] $ is the intersection of all valuation rings that 
belong to the set $\VV(F/K) \setminus \{ \Oeul_P \}$.

\end{proposition}

\begin{proof}
Omitted
\end{proof}


let $ k$ be a field, $A =  k[x,y]$ the polynomial ring in 
two variables over $ k$,
$L= \Frac A =  k(x,y)$ the field of rational functions in 
two variables, and $K =  k(F)$.
Let $\VV(F)$ be the set of all valuation rings of the 
function field $L/K$.
Let $\VV^\infty(F) = \setspec{ R \in \VV(F) }{ A \nsubseteq R }$
be the set of dicriticals of $F$.


\begin{definition}
Let $A =  k[x,y]$ be a polynomial ring in two variables 
over a field $ k$.
A {\it field generator\/} of $A= k[x,y]$ is an element
$F \in A$ that satisfies:
$\text{$ k(x,y) =  k(F,G)$ for some $G \in  k(x,y)$.}$
If $F \in A$ satisfies the stronger condition
$\text{$ k(x,y) =  k(F,G)$ for some $G \in  k[x,y]$}
$ we call $F$ a {\it good\/} field generator of $A$.
A field generator that is not good is said to be bad.
\end{definition}

\begin{remark}
Given any $F \in  k[x,y] \setminus  k$, we know that
$ k(x,y) /  k(F)$ is a function field
of one variable. Observe that $F$ is a field generator 
of $ k[x,y]$ if and only if 
$ k(x,y) /  k(F)$ is the rational function field.
\end{remark}

\begin{proposition}
Suppose that $F$ is a field generator of $A= k[x,y]$ 
such that $1$ occurs in the list $\Delta(F)$, then $F$ is a 
good field generator.
\end{proposition}

\begin{proof}[Proof by Daniel Daigle:]
Let $F \in A$ be a field generator of $A$ such that 
`$1$' occurs in $\Delta(F)$.
Write $L =  k(x,y)$ and $K =  k(F)$, then $L/K$ is the 
rational function field of one variable.
Let $\VV(F)$ be the set of all valuation rings of the 
function field $L/K$.
Let 
$$
\VV^\infty(F) = \setspec{ R \in \VV(F) }{ A \nsubseteq R } = \{ R_1, \dots, R_s \}
$$
be the set of dicriticals of $F$. Since `$1$' occurs in
$\Delta(F)$, one of $R_1, \dots, R_s$
is a dicritical of degree $1$; relabelling  $R_1, \dots, R_s$ 
if necessary, we may arrange that
$R_1$ is a dicritical of degree $1$. Let  $P$ be the maximal 
ideal of $R_1$; then
$$
\text{$P$ is a place of degree $1$ of the rational function field $L/K$.}
$$
Moreover, $R_1$ is the valuation ring of $P$, i.e., $R_1 = \Oeul_P$.
By \ref{0923d091sd0192s}, there exists $t \in L$ satisfying
$L = K(t)$, $\Oeul_P =  k[ t^{-1} ]_{(t^{-1})}$, and
\begin{equation} \label {uqid7623erih9we8}
K[t] = \bigcap_{\Oeul \in E} \Oeul
\end{equation}
where $E = \VV(F) \setminus \{ \Oeul_P \}$.

Consider the ring $\Aeul = S^{-1}A$ where $S =  k[F] \setminus \{0\} \subset A \setminus \{0\}$. 
Then
$$
\text{$A$ is a UFD} \overset{\ref{i12biud012oww}}{\implies} \text{$A$ is a normal}
\overset{\ref{poifj012woid2q9}}{\implies} \text{$\Aeul$ is a normal} .
$$
Since $\Frac( \Aeul ) = L$, it follows that $\Aeul$ is integrally closed in $L$. Thus, 
by Cor.~5.22 of Atiyah-McDonald,
$\Aeul$ is equal to the intersection of all valuation rings $\Oeul$ 
of $L$ that satisfy $\Aeul \subseteq \Oeul$.
\begin{equation} \label {bcdkoi23w9eje}
\Aeul = \bigcap_{\Oeul \in E'} \Oeul
\end{equation}
where $E'=$ set of all valuation rings $\Oeul$ of $L$ that satisfy $\Aeul \subseteq \Oeul$.
Note that $L \in E'$; let us prove that 
\begin{equation} \label {h9jccnodi1ds}
E' \subseteq E \cup \{ L \} .
\end{equation}
Indeed, consider $\Oeul \in E'$ such that $\Oeul \neq L$, and let us prove that  $\Oeul  \in E$.
Since $\Oeul$ is a valuation ring of $L$ such that $\Oeul \neq L$ and
$$
 k(F) = S^{-1}  k[F] \subseteq S^{-1}A = \Aeul \subseteq \Oeul,
$$ 
it follows that $\Oeul$ is a valuation ring of $L/K$, i.e., $\Oeul \in \VV(F)$. Since
$A \subseteq \Aeul \subseteq \Oeul$, we have $\Oeul \notin \{ R_1, \dots, R_s \}$,
so $\Oeul \in \VV(F) \setminus \{R_1\} = E$. This proves \eqref{h9jccnodi1ds}.

It follows that
\begin{equation} \label {lo923idewo}
\text{for each $\Oeul \in E'$,\quad  $K[t] \subseteq \Oeul$.}
\end{equation}

Indeed, if $\Oeul \in E'$ then \eqref{h9jccnodi1ds} implies that $\Oeul \in E$ or $\Oeul=L$;
in the first case we have $K[t] \subseteq \Oeul$ by \eqref{uqid7623erih9we8},
and in the second case we have $K[t] \subseteq L = \Oeul$. So \eqref{lo923idewo} is true.
It follows from \eqref{lo923idewo} that
$$
K[t] \subseteq \bigcap_{\Oeul \in E'} \Oeul\ =\ \Aeul,
$$
so in particular $t \in \Aeul = S^{-1}A$; then $t = G/s$ for 
some $G \in A$ and $s \in S =  k[F] \setminus \{0\}$.
Since $s \in K^*$, we have $K[t] = K[st] = K[G]$, so 
$$
 k(F,G) = K(G) = K(t) = L,
$$
showing that $F$ is a good field generator of $A$.
\end{proof}





	\section{Exercises}
		\begin{exercise} \label{irriduciblePoverC}
	Consider the poylnomial ring $\Comp [x,y]$. Let $p =y^2 +x^3 - x \in \Comp[x,y]$ 
	and $\pi : \Comp[x,y] \longrightarrow A = \Comp[x,y]/(p)$ by defined by
	$\pi(f) = f + (p)$ for all $f \in \Comp[x,y]$. Let $F$ be the field of 
	fractions of $A$ and $\m \subset A$ be image of the prime ideal $(x,y)$ under $\pi$. 
	Show 
	\begin{enumerate}[(a)]
		\item \label{pisirreducible}$p$ is irreducible in $\Comp[x,y]$
		\item \label{basisisis} $A$ is a free $\Comp[x]$-module, with basis $\{ 1, y \}$.
		\item \label{ellipticfunctionfield} $F/ \Comp $ is a function field in one variable.
		\item \label{localizationValuation} $A_\m$ is a valution ring of the function field $F/ \Comp$.
	\end{enumerate}
\end{exercise}

\begin{solution}
	\eqref{pisirreducible}: Suppose $p = gh $ for $g,h \in \C[x,y] \notzero $. 
	We may view $p$ as a polynomial in one variable $y$ over $\C[x]$. 
	Hence $deg_y(p)= deg_y(gh) = deg(g) + deg(h)$. Since $deg_y(p)=2$, 
	there are three cases for the $y$-degrees of $g$ and $h$: 
	\begin{enumerate}[(i)]
		\item $\deg_y(g) = 1$ and $\deg_y(h) = 1$ 
		\item $\deg_y(g) = 2$ and $\deg_y(h) = 0$ 
		\item $\deg_y(h) = 2$ and $\deg_y(g) = 0$
	\end{enumerate}
	Suppose $deg_y(g)=deg_y(h)=1$. 
	Write $g= a_1y + a_2$ and $h= b_1y + b_2$ where
	$a_1,a_2,b_1,b_2 \in \C[x]$. 
	Then $p = gh = (a_1y + a_2)(b_1y + b_2) = a_1b_1y^2 + (a_1b_2 + a_2b_1)y + a_2b_2 $.
	Thus we have the equations;
	\begin{align*}
		&a_1b_1 = 1  \\
		&a_1b_2 + a_2b_1 = 0 \\
		&a_2b_2 = x^3 - x 
	\end{align*} 
	$a_1b_1 = 1 \Longrightarrow a_1 = b_1^{-1}$. 
	Then $a_1b_2 + a_2b_1 = 0$ becomes
	$b_2=- b_1^2a_2$. Then in $a_2b_2 = x^3 - x $ we have
	$a_2b_2 = - a_2^2b_1^2 = - (a_2b_1)^2 = x^3 - x$ which is 
	impossible since $x^3 - x = x(x-1)(x+1)$ is not a 
	square. So either $f$ or $g$ is a unit. 
	If $\deg_y(g) = 2$ and $\deg_y(h) = 0$, then we 
	may write $g = a_1y^2 + a_2y + a_3$ and $h = b$ for some 
	$a_1,a_2,a_3,b \in \Comp[x]$. 
	Then $$y^2 +x^3 - x = g = a_1by^2 + a_2by + a_3b$$
	This would mean that $a_1b= 1$. Hence $b \in \Comp[x]^* $.
	So $b$ must be a unit of $\Comp[x,y]$ as well. For the last
	case (where $\deg_y(h) = 2$ and $\deg_y(g) = 0$), the same 
	argument will show that $g \in \Comp[x,y]^*$.
	Thus $p$ is irreducibe in $\C[x,y]$.  \\

	\eqref{basisisis}: Suppose $a+by = 0$ for nonzero $a,b \in \Comp[x]$. 
	Recall the ``re-definition'' $x = \pi(x)$ and $y = \pi(y)$. To 
	avoid confusion, let $X,Y$ be used to represent variables in 
	$\Comp[X,Y]$ and $x,y$ be used to represent variables in $A$. 
	Then $a+bY \in \text{ker}(\pi) = (p)$ - but this is impossible since 
	$\deg_Y(p) = 2 \geq deg_Y(a+bY) = 1$. This means that $\lbrace 1,y \rbrace$
	is an independent set in $A$. Clearly $\text{span} \lbrace 1,y \rbrace \subseteq A$.
	Let $f \in A$. Then $$f = a_0 + a_ 1x + a_2y + a_3xy + a_4 x^2 
	+ a_5y^2 + a_6x^2y + a_7xy^2 + a_8x^2y^2 + ... + a_nx^ny^m$$
	for some $a_0,a_1,...,a_n \in \Comp$ where $y^2+x^3 - x = 0$. 
	In each term of $f$ divisible by $y^2$, substitute $y^2$ with $x - x^3$. 
	Then $$f = ... + a_3xy + a_4 x^2 + a_5(x - x^3) + a_6x^2y + a_7x(x - x^3) 
	+ a_8x^2(x - x^3) + ... + a_nx^n(x - x^3)^ky^l$$
	where $k = m/2, l = 0$ if $m$ is even and $k = (m-1)/2, l = 1$ if $m$ is odd.
	Factoring out the $y$ in some of terms, we may rearrange $f$ as;
	$$f =  a_0 + a_ 1x +  a_4 x^2  + a_5(x - x^3)  + a_7x(x - x^3) + ... + 
	(a_2 +  a_3x  + a_6x^2 + ... +a_nx^n(x - x^3)^k)y $$
	Hence $f \in \text{span}_{\Comp[x]} \lbrace 1,y \rbrace$. 
	So $A$ is a free $\Comp[x]$-module, with basis $\{ 1, y \}$.\\

	\eqref{ellipticfunctionfield}: Since $\Comp \cap (p) = \lbrace 0 \rbrace $, 
	the composition of the cannonical projection map
	of the quotient ring $A$ with the inclusion homomorphism $f \mapsto f/1 $ of 
	$A$ to $F$ embeds $\Comp $ in $F$. So we may view $\Comp $ as a subfield of
	$F$. To show that $F/ \Comp$ is a transcendental extension of 
	fields, condsider $y\in F$. Suppose $y$ is algebraic over 
	$\Comp $, then $a_0 + a_1y + ... + a_ny^n \in (p)$ 
	for some $a_0,...,a_n \in \Comp$. That would mean 
	that $a_0 + a_1y + ... + a_ny^n = g(x,y)(p) = g(x,y)(y^2 + x - x)$ 
	for some nonzero $g(x,y) \in \Comp[x,y]$.
	But $\deg_x(a_0 + a_1y + ... + a_ny^n) = 0$ and 
	$\deg_x (y^2 +x^3 -x) = 3$. So there does not exists such a $g$.
	Hence $y$ is transcendental over $\Comp$.
	We want to show that $F/\Comp(x)$ is finite.
	Let $0 \neq z \in F$. Write $z = f/g$ for $f,g\in A$ with $g \neq 0$.
	By \eqref{basisisis}, we may write $f = a + by$ for 
	$a,b \in \Comp[x]$. 
	Since $0 \neq z$, we may assume $a\neq 0$ or  $b \neq 0$. 
	\textit{Claim}: If $m(T) \in \Comp(x) [T] \backslash \lbrace 0 \rbrace$ is 
	some polynomial satisfying $m(f) = 0$. Then we may find another 
	nonzero polynomail $m^{\prime}(T)$, that depends on $g$, such 
	that $m(z) = 0$. \\

	Proof of claim:
	Assume there exists a nonzero polynomial $m(T) \in \Comp(x)[T]$ 
	such that $m(f) = 0$. Write $m(T) = a_0 + a_1T + ... + a_nT^n$ 
	for $a_1,...,a_n \in \Comp(x)$. Consider the terms in the polynomial $g$.
	Write $g(x,y) = \sum_{i,j\in \Nat } b_{ij} x^iy^j$. 
	If each term of $g$ is either in $\Comp(x)$ or divisible by $y^2$,
	then we may multiply $m(T)$ by $h(x) = (\sum_{i,j\in \Nat } b_{ij} x^i(x-x^3)^j)^n$
	Then all the denominators of $m(z)$ will be cleared by $h(x)$, which 
	belongs to $\Comp(x)$. So if we define $m^{\prime}(T) = h(x)m(T)$,
	then $m^{\prime} (z) = 0$. Alternatively, suppose any of the terms
	of $g$ are of the form $uy^n$ for $u\in \Comp(x)$ and $n$ odd.
	We may assume that only one of the terms of $g$ is of this form 
	and that $n = 1$: If more of the terms are of this form then 
	repeat the following process again. 
	The case $n > 1 $, follows by induction. So we may write
	$g = a + by$ for some $a,b \in \Comp(x)$. let $h(x) = (a^2 - b^2(x-x^3)^2)^n$ and 
	let $m^{\prime}(T) = h(x)m(T)$. 
	Thus
	\begin{align*}
		m^{\prime}(z) &= (a^2 - b^2(x-x^3)^2)^n(a_0 + a_1(\frac{f}{a+by}) + ... + a_n(\frac{f}{a+by})^n) \\
		&= (a+by)^n(a-by)^n (a_0 + a_1(\frac{f}{a+by}) + ... + a_n(\frac{f^n}{(a+by)^n})) \\
		&= (a+by)^n(a-by)^na_0 + a_1(a+by)^{n-1}(a-by)^nf + ... + a_nf^n(a-by)^n \\
		&= 0  
	\end{align*}

	So it suffices to find a nonzero polynomial $m(T) \in \Comp (x)[T]$
	such that $m(f) = 0$. Let $m(T) = (T-a)^{2} - b^2(x-x^3)$. Then 
	clearly $m(a+by) = (a+by - a)^2 - b^2y^2 = 0$. Since $a \neq 0 $ 
	or $b \neq 0$, it follows that $m(T)$ is nonzero. Hence
	$F/\Comp(x)$ is finite. Thus $F/\Comp$ is a function field \\

	\eqref{localizationValuation}: By definition, 
	$A_\m = \lbrace f/g \mid f,g \in A , g \neq 0 \text{ and } g \notin \m \rbrace $.
	So $\Comp \subsetneqq A_\m \subsetneqq F$ is clear. Let $z \in F$. 
	Write $z = f/g$ for $f,g \in A$ and $g \neq 0$. Suppose $f,g \in \m$. 
	Then we may write $f = ax+by,g = cx + dy$ for $a,b,c,d \in \C[x,y]$.
	Thus
	\begin{align*}
		z &= f/g \\
		&= \frac{ax+by}{cx + dy} \\
		&= \frac{(ax+by)(cx - dy)}{(cx + dy)(cx - dy)} \\
		&= \frac{acx^2 + (cb-ad)xy - bdy^2}{c^2x^2 -d^2y^2} \\
		&= \frac{acx^2 + (cb-ad)xy - bd(x-x^3)}{c^2x^2 -d^2(x-x^3)} \\
		&= \frac{acx + (cb-ad)y - bd(1-x^2)}{c^2x -d^2(1-x^2)} \\
		&= \frac{acx + (cb-ad)y - bd + bdx^2}{c^2x -d^2 + d^2x^2} 
	\end{align*}

	Notice, if $b\neq 0, d \neq 0$, then both the numerator and denominator 
	do not belong to
	$\m$ since $\m \cap \Comp = \lbrace 0 \rbrace$. So $z \in A_\m^*$.
	If $d = 0$ then $z = a/c$. If both $a,c\in \m$, then we
	may repeat the same process again. If both $b = 0$ and $d \neq 0$. 
	Then only $z \in A_\m^*$. Thus $A_\m $ is a valuation ring of $F$.
\end{solution}

\begin{exercise} \label{primeontoprime}
	Let $A \stackrel{\phi}{\longrightarrow} B$ be a surjective homomorphism of rings. 
	Let $\pgoth $ be a prime ideal in $R$ containing the kernel of $\phi$. 
	Then $\phi(\pgoth)$ is prime in $S$.
\end{exercise}

\begin{solution} 
	Let $x,y \in B$ and suppose $xy \in \phi(\mathfrak{p})$. Since $\phi$ is 
	surjectivity, there exists $a, b \in A$ such that $\phi(a)= x$, $\phi(b)=y$. 
	Choose $c \in \mathfrak{p}$ such that $\phi(c) = xy$.  
	Then $ab -c \in ker(\phi)$, so $ab \in \mathfrak{p}$.  
	Thus, either $a$ or $b$ is in $\mathfrak{p}$, which means 
	either $x$ or $y$ is in $\phi(\mathfrak{p})$.
\end{solution}

    
\begin{exercise}(Stichtenoth, Exercise 1.1)
	Consider the rational function field $K(x)/K$ and a non-constant element 
	$z = f(x)/g(x) \in K(x) \backslash K$, where $f(x),g(x) \in K[x]$ are 
	relatively prime. We call $deg(z) = max \lbrace deg(f) , deg(g) \rbrace $ 
	the degree of $z$.
	\begin{enumerate}[(i)]
		\item Show that $[K(x): K(z)] = deg(z)$, and write down the minimal 
		polynomial of $x$ over $K(z)$
		\item Show that $K(x)=K(z) $ if and only if $z = (ax+b)/(cx+d)$ with
		$a,b,c,d \in K$ and $ad - bc \neq 0$.
	\end{enumerate}
\end{exercise}

\begin{solution} 
	We find the minimal polynomial of the field extension $K(x)/K(z)$.
	\begin{enumerate}[(i)]

		\item Consider the polynomial $m(t) = zg(t) - f(t) \in K(z)[t]$. 
		Notice that $0 \neq z = f(x)/g(x) \Longrightarrow f(x) \neq 0 \Longrightarrow m(t) \neq 0$
		and that $m(x) = 0$. Also, if $deg(g(x)) \geq deg(f(x))$, 
		then $deg(m(t)) = deg(g(t)) = deg(g(x))$.
		Otherwise $deg(m(t)) = deg(f(t)) = deg(f(x))$. So $deg(m(t)) = deg(z)$,
		as required. Lastly we need to show that $m(t)$ is irreducible over
		$K(z)$. By Gauss's lemma, it is sufficient to check that $m(t)$ is 
		irreducible over $K[z]$ but $K[z][t]=K[t][z]$, in which $m(t)$ is linear. 
		Hence $m$ is the minimal polynomial of the field extension $K(x)/K(z)$ 
		and $[K(x):K(z)] = deg(m(t)) = deg(z)$ as required.

		\item Assume $z = (ax+b)/(cx + d)$ with $a,b,c,d \in K$ and 
		$ad - cd \neq 0$. The condition $ad-bc \neq 0 $ implies that 
		$z \notin K$. By part $i)$, $[K(x):K(z)] = deg(z) = 1$, Hence 
		$K(x) = K(z)$. Assume $K(x) = K(z)$ and suppose $z \neq (ax+b)/(cx + d) $ 
		for any $a,b,c,d \in K$ satisfying $ad - bc \neq 0$. Then 
		either $z \in K$ or $deg(z) \geq 2$. If $z \in K$, then $K(z) = K$, 
		which would imply $K(x) = K$, a contradiction to the fact 
		that $x$ is transcendental over $K$. If $deg(z) \geq 2$, 
		then by part $i)$, $[K(x):K(z)] \geq 2$. This contradicts the fact 
		that every field is a one dimensional vector space over itself. 
		So our supposition must be false, thus $z = (ax+b)/(cx+d)$ with 
		$a,b,c,d \in K$ and $ad - bc \neq 0$.
	\end{enumerate}
\end{solution}

\begin{exercise}(Stichtenoth, Exercise 1.2) \label{GLP2}
	For a field extension $L/M$ we denote by $Aut(K(x)/K)$ the group 
	of automorphisms of $L/M$ (i.e., automorphisms of $L$ which are 
	the identity on $M$). Let $K(x)/K$ be the rational function field 
	over $K$. Show:

	\begin{enumerate}[(i)]
		\item \label{existsabcd} For every $\sigma \in Aut(K(x)/K)$ there 
		exists $a,b,c,d \in K$ such that $ad-bc \neq 0$ and $\sigma (x) = (ax+b)/(cx+d)$. 

		\item Given $a,b,c,d \in K $ with $ad - bc \neq 0$, there 
		is a unique automorphism $\sigma \in Aut(K(x)/K)$ with $\sigma(x) = (ax+b)/(cx+d)$.

		\item Denote by $GL_2(K)$ the group of invertible $2 \times 2$ 
		- matrices over $K$. For $A= 
		\begin{pmatrix} a & c \\ b & d \\ \end{pmatrix} \in GL_2(K)$ 
		denote by $\sigma_A$ the automophism of $K(x)/K$ with $\sigma_A(x) = (ax+b)/(cx+d) $. 
		Show that the map that sends $A$ to $\sigma_A$, is a homomorphism of $GL_2(K) $ 
		onto $Aut(K(x)/K)$. Its kernel is the set of diagonal matrices of the 
		form $\begin{pmatrix} a & 0 \\ 0 & a \\ \end{pmatrix}$ with $a \in K^{\times}$, 
		hence $$Aut(K(x)/K) \cong GL_2(K)/K^{\times}$$ (The group $GL_2(K)/K^{\times}$ 
		is called the projective linear group and is denoted by $PGL_2(K)$.)
	\end{enumerate}
\end{exercise}

\begin{solution} 
	Any given tuple $(a,b,c,d) \in K^4$ will satisfy $ad - bc = 0$ 
	unless otherwise specified.
	\begin{enumerate}[(i)]
		\item Let $\sigma \in Aut(K(x)/K)$. We show that $K(x) = K(\sigma(x)) $. 
		Let $f \in K(\sigma(x))$, then 
		$$
		f(\sigma(x)) = \frac{a_n\sigma(x)^n + a_{n-1}\sigma(x)^{n-1} + ... + a_0 }{b_m\sigma(x)^m + b_{m-1}\sigma(x)^{m-1} + ... + b_0 } 
		$$ 
		for $a_0,...,a_n,b_0,...,b_m \in K$. Since $\sigma(a) = a $ 
		for all $a \in K$ and $\sigma$ is a homomorphism, we may 
		rewrite $f$ as $$\sigma (\frac{a_nx^n + a_{n-1}x^n + ... + a _0}{b_mx^m + b_{m-1}x^{m-1} + ... + b_0}) \in im(\sigma)$$ 
		Since $\sigma$ is surjective, $f \in im(\sigma) = K(x)$. \\

		Let $f \in K(x)$, then $f \in im(\sigma)$ by surjectivity of $\sigma$. 
		Hence there exists some $g\in K(x)$ such that $\sigma(g(x)) = f(x)$.
		Write $$g(x) = \frac{a_nx^n + a_{n-1}x^n + ... + a _0}{b_mx^m + b_{m-1}x^{m-1} + ... + b_0}$$ 
		then $$
		f(x)= \sigma(\frac{a_nx^n + a_{n-1}x^n + ... + a _0}{b_mx^m + b_{m-1}x^{m-1} + ... + b_0}) 
		= \frac{a_n\sigma(x)^n + a_{n-1}\sigma(x)^{n-1} + ... + a_0 }{b_m\sigma(x)^m + b_{m-1}\sigma(x)^{m-1} + ... + b_0 } 
		$$ 
		Hence $f \in K(\sigma(x))$. That is, $K(x) = K(\sigma(x))$ as required.
		By part (ii) of exercise 1, $\sigma(x) = (ax+b)/(cx+d)$.

		\item Let $a,b,c,d \in K$, define $\sigma : K(x) \longrightarrow K(x) $ by $\sigma(f(x)/g(x))$ 

		\item Let $\Phi : GL_2(K) \longrightarrow Aut(K(x)/K) $ be defined as
		$\Phi (A) = \sigma_A $ for all $A \in GL_2(K)$. 
		\begin{remark}
			To verify that two automorphisms $\sigma_2,\sigma_1 \in Aut(K(x)/K)$ 
			are equivalent, it suffices so show that $\sigma_1(x)=\sigma_2(x)$. 
			This is because every automorphism of $Aut(K(x)/K)$ is uniquely 
			determined by $x$. 
		\end{remark}
			First, $\Phi$ preserves
			identity: $\Phi(\begin{pmatrix} 1 & 0 \\ 0 & 1 \\ \end{pmatrix})(x) = (1x + 0)/(0x +1) = x = i(x)$
			where $i$ is the indentiy automorphism of $Aut(K(x)/K)$. Let $A,B \in GL_2(K)$.
			Then $A = \begin{pmatrix} a & c \\ b & d \\ \end{pmatrix} , B = \begin{pmatrix} e & g \\ f & h \\ \end{pmatrix} $ 
			for $a,b,c,d,e,f,g,h \in K$. Then
			\begin{align*}
				\Phi (\begin{pmatrix} a & c \\ b & d \\ \end{pmatrix}
				\begin{pmatrix} e & g \\ f & h \\ \end{pmatrix})(x) &= 
				\Phi (\begin{pmatrix} ae + cf & ag + ch \\ be + df & bg + dh \\ \end{pmatrix})(x) \\
				&= \frac{(ae+cf)x + be+df}{(ag+ch)x+bg+dh} (x)\\
				&= \frac{e(ax+b) + (cx+d)f}{g(ax+b)+ (cx+d)h} \\
				&= \frac{[e(ax+b) + (cx+d)f](cx+d)}{[g(ax+b)+ (cx+d)h](cx+d)} \\
				&= \frac{\frac{e(ax+b) + (cx+d)f}{(cx+d)}}{\frac{g(ax+b) + (cx+d)h}{(cx+d)}} \\
				&= \frac{\frac{e(ax+b)}{(cx+d)}+f}{\frac{g(ax+b)}{(cx+d)}+h} \\
				&= \frac{e\sigma_A(x)+f}{g\sigma_A(x)+h} \\
				&= \sigma_A(\frac{ex+f}{gx+h})(x) \\
				&= \sigma_A \circ \sigma_B (x) \\
				&= \Phi(\begin{pmatrix} a & c \\ b & d \\ \end{pmatrix}) \Phi(\begin{pmatrix} e & g \\ f & h \\ \end{pmatrix}) (x)
				\end{align*}
			\noindent Hence $\Phi$ is a group homomorphism. Clearly
			$\Phi(\begin{pmatrix} a & 0 \\ 0 & a \\ \end{pmatrix})(x) = \frac{ax + 0}{0x +a} = x = i(x)$. 
			So $\begin{pmatrix} a & 0 \\ 0 & a \\ \end{pmatrix} \in ker(\Phi)$. 
			Let $A = \begin{pmatrix} a & c \\ b & d \\ \end{pmatrix} \in ker(\Phi)$, 
			then $\frac{ax+b}{cx + d} = x \Rightarrow 0 = cx^2 + (d-a)x - b \Rightarrow a = d, b = c = 0$. 
			Notice that $a=d = 0$ implies $ad - bc = 0$, which is not allowed here. 
			Hence $ker(\Phi) = \begin{pmatrix} a & 0 \\ 0 & a \\ \end{pmatrix} $
			where $a \in K^{\times}$. Let $\sigma \in Aut(K(x)/K)$, 
			then by part \eqref{existsabcd}, there exists $a,b,c,d \in K$ such that 
			$\sigma(x) = \frac{ax+b}{cx+d}$, that is there exists 
			$A = \begin{pmatrix} a & 0 \\ 0 & a \\ \end{pmatrix} \in GL_2(K)$ such that 
			$\Phi(A) = \sigma$. Hence $\Phi$ is surjective and $Aut(K(x)/K) \cong GL_2(K)/\mathfrak{K}$,
			where $\mathfrak{K} = \begin{pmatrix} a & 0 \\ 0 & a \\ \end{pmatrix}$ for $a \in K^{\times}$.
	\end{enumerate}
\end{solution}
    

\begin{exercise}(Stichtenoth, Exercise 1.4)
	Let $K(x)$ be the rational function field over $K$. Find 
	bases for the following Riemann-Roch spaces: 
	\begin{enumerate}[(i)]
		\item \label{infinity} 		$\mathscr{L}(rP_{\infty})$
		\item \label{linear}   		$\mathscr{L}(rP_{\alpha})$
		\item \label{irreducible11} 	$\mathscr{L}(P_{p(x)})$
	\end{enumerate}
	where $R \geq 0$, and the places $P_{\infty}, P_{\alpha}$ and $P_{p(x)}$
	are as in section $1.2$ of Stichtenoth. 
\end{exercise}

\begin{solution}
	\eqref{infinity}: Let $r \geq 0$. $\mathscr{L}(rP_{\infty})$: Write $z \in K(x)^*$ as $z = f/g$ 
	where $f,g$ are relatively prime in $K[x]$. Then $z \in \mathscr{L}(rP_\infty)$ 
	if and only if $v_P(z) + v_P(rP_\infty) \geq 0 $ for all 
	$P \in \mathbb{P}_{K(x)}$ if and only if $v_P(z) + 0 \geq 0 $ 
	for all $P\in \mathbb{P}_{K(x)} \backslash \lbrace P_\infty \rbrace $ 
	and $v_\infty(z) + r \geq 0$ if and only if $z \in K[x]$ and 
	$r \geq \deg(f)$. Hence $\mathscr{L}(r P_\infty)$ is the vector space 
	of all polynomial in $K[x]$ with degree less than or equal to 
	$r$. Transcedance of $x$ over $K$ implies $1,x,x^2,...,x^r$ 
	are linearly independent over $K$ and clearly they span 
	$\mathscr{L}(rP_\infty)$. Hence $\ell(rP_\infty) = r+1$. 

	\eqref{linear}: Since $P_a$ corresponds to the linear polynomial
	$p = x - a$, so $\deg P = 1$. By proposition \ref{0923d091sd0192s}, there 
	exists $t \in K(x)$ such that $\OV_{P_a} = K[t^{-1}]_{(t^{-1})}$. Then
	from the solution to \eqref{infinity}, $\lbrace 1,t,t^2,...,t^r \rbrace$ is a $K$-basis
	for $\mathscr{L}(rP_{\alpha})$ with dimension $r + 1$.

	\eqref{irreducible11}: Let $n = \deg p(x)$. Then since 
	$p(x)$ is irriducible over $K$, all other places of 
	the form $P_{q(x)}$ for some irreducible polynomial $q(x) \in K[x]$
	will have valuation $0$. Thus by theorem \ref{Princ0}, $v_\infty(p) = -n$.
	Hence if $z \in \LL(P_{p(x)})$, then $v_\infty(z) \geq -n$. 
	Similarly, $v_P(z) \geq 0$ for all $P \in \PF \backslash \lbrace P_{p(x)},P_\infty \rbrace$.
	So $z \in K[x]$. This implies that $z$ is a polynomial in $K[x]$ with degree 
	less than or equal to $n$. Then clearly $\LL(P_{p(x)})$ has $K$-basis $\lbrace 1,x,...,x^{n} \rbrace$,
	which implies $\ell(P_{p(x)}) = n+1$.
\end{solution}
  
\begin{exercise}(Stichtenoth, Exercise 1.5) \label{PForPartial}
    
(\textit{Representation of rational funcitons by partial fractions})
    
\begin{enumerate}[(i)]
	\item Show that every $z \in K(x)$ can be written as 
	$$z = \sum^r_{i = 1} \sum^{k_i}_{j=1} \frac{c_{ij}(x)}{p_i(x)^j} + h(x)$$ where 

	\begin{enumerate}
		\item $p_1(x),...,p_r(x)$ are distinc monic irreducible polynomials in $K[x]$,
		\item $k_1,...,k_r \geq 1$,
		\item $c_{ij}(x) \in K[x]$ and $deg(c_{ij}(x)) < deg(p_i(x))$,
		\item $c_{ik_i}(x) \neq 0$ for $1 \leq i \leq r$,
		\item $h(x) \in K[x]$
	\end{enumerate}
	\item Show that the above representation of $z$ is unique.
\end{enumerate}
\end{exercise}

\begin{solution}
	\begin{enumerate}[(i)] 
		\item Let $z = f(x)/g(x) \in K(x)$. If $\deg(f) \geq \deg(g)$, 
		then we may write $f = q(x)g(x) + r(x)$ with $q(x),r(x) \in K[x]$ and
		$\deg(r(x)) < \deg(g(x))$. Then $f(x)/g(x) = q(x) + r(x)/g(x)$ and we may use 
		the following argument for $r(x)$ instead of $f(x)$.
		Hence it suffices to consider the case 
		where $\deg(g(x)) > \deg(f(x))$. Write $g(x) = p_1(x)^{e_1}p_2(x)^{e_2}...p_s(x)^{e_s}$ 
		where $e_1,e_2,...,e_s \in \Integ^{+}$ and $p_1,p_2,...,p_s$ are distinct 
		prime factors of $g(x)$. Suppose $deg(f) < deg(g)$. Then 
		$$\text{gcd}(p_1(x)^{e_1},p_2(x)^{e_2}...p_s(x)^{e_s}) = 1$$ hence there exists 
		$h_1(x),h_2(x) \in K[x]$ such that $$1 = h_1(x)p_1(x)^{e_1} + h_2(x)p_2(x)^{e_2}...p_s(x)^{e_s}$$ 
		Thus $$f(x) = f(x)h_1(x)p_1(x)^{e_1} + f(x)h_2(x)p_2(x)^{e_2}...p_s(x)^{e_s}$$ 
		We may divide and write $h_1(x)f(x) = p_1(x)^{e_1}q(x) + r_1(x)$ 
		for some $q(x),r_1(x) \in K[x]$ with $deg(r(x)) < deg(q(x))$. 
		Let $$r_2(x) = h_2(x)p_1(x)^{e_1}p_2(x)^{e_2}...p_s(x)^{e_s} + q(x)p_2(x)^{e_2}...p_s(x)^{e_s}$$ 

		\begin{align*} 
			f(x) &= p_1(x)^{e_1}r_2(x) - p_1(x)^{e_1}p_2(x)^{e_2}...p_s(x)^{e_s}q(x) \\
			&+ p_1(x)^{e_1}p_2(x)^{e_2}...p_s(x)^{e_s}f(x) + p_2(x)^{e_2}...p_s(x)^{e_s}r_1(x) \\
			&= p_1(x)^{e_1}r_2(x) + p_2(x)^{e_2}...p_s(x)^{e_s}r_1(x)
		\end{align*}
		Hence $\frac{f(x)}{g(x)} = \frac{r_1(x)}{p_1(x)^{e_1}} + \frac{r_2(x)}{p_2(x)^{e_2}...p_s(x)^{e_s}} $. 
		Claim: $deg(r_2(x)) < deg(p_2(x)^{e_2}...p_s(x)^{e_s})$: Suppose the opposite, 
		then $deg(p_1(x)^{e_1}r_2(x)) \geq deg(p_1(x)^{e_1}p_2(x)^{e_2}...p_s(x)^{e_s})$ 
		but we also have 
		$$deg(p_1(x)^{e_1}p_2(x)^{e_2}...p_s(x)^{e_s}) >  
		deg(deg(p_1(x)^{e_1}p_2(x)^{e_2}...p_s(x)^{e_s}r_1(x))$$
		Then
		    
		\begin{align*}
			deg(f(x)) &= deg(p_1(x)^{e_1}r_2(x) + p_2(x)^{e_2}...p_s(x)^{e_s}r_1(x)) \\
			&= deg(p_1(x)^{e_1}p_2(x)^{e_2}...p_s(x)^{e_s}r_2(x)) \\
			&\geq deg(p_1(x)^{e_1}p_2(x)^{e_2}...p_s(x)^{e_s}) \\
			&= deg(p_1(x)^{e_1}) + deg(p_2(x)^{e_2}...p_s(x)^{e_s}) \\
			&< deg(f(x))
		\end{align*}

		This is a contradiction, so the claim is verified. 

		Hence we may 
		repeat this processess $s-1$ times to obtain the expression
		$$\frac{f(x)}{g(x)} = \sum^s_{i=1}\frac{r_i(x)}{p_i(x)^{e_i}}$$ 
		We now need to expand the powers of $p_i(x)$ for $i=1,...,s$. 
		For $i=1,...,s$, let $r_{i0}(x) = p_i(x)$  and for $j=1,...,e_s$ 
		we can use the division algorithm to find $q_{ij}(x),r_{ij}(x) \in K[x]$ 
		such that $$q_{i(j-1)}(x) = q_ij(x)p_i(x) + r_{ij}(x)$$ where $\deg(r_{ij}(x)) < \deg(q_{ij}(x))$.
		Using back substitution, we find 
		\begin{align*}
			r_i(x)_i(x) &= q_{i0}(x) \\
			&= q_{i1}(x)p_i(x) + r_{i1}(x) \\
			&= (q_{i2}(x)p_i(x)+r_{i2}(x))p_i(x) + r_{i1}(x) \\
			&= ... \\
			&= q_{ie_{s}}(x)p_i(x)^{e_s} + r_{ie_{s}}(x)p_i(x)^{e_s -1} \\
			&+ r_{i(e_s -1)}(x)p_i(x)^{e_s -2} + ... + r_{i2}(x)p_i(x) + r_{i1}(x) \\
			\Longrightarrow \frac{r_i(x)}{p_i(x)^{e_s}} &= q_{ie_s}(x) + 
			\frac{r_{ie_s}(x)}{p_i(x)}+ \frac{r_{i(e_s -1)}(x)}{p_i(x)^2} + 
			...+ \frac{r_{i2}(x)}{p_i(x)^{e_s - 1}} + \frac{r_{i1}(x)}{p_i(x)^{e_s}}
		\end{align*}
		Hence $$\frac{f(x)}{g(x)} = \sum^s_{i=1}(q_{ie_s}(x) + \sum^i_{j=1}  \frac{r_{ij}(x)}{p_i(x)^j})$$
		Let $h(x) = \sum^s_{i=1}q_{ie_s}(x)$, $c_{ij}(x)=r_{if(x)}$, $e=k$ and $r = s$. Then 
		$$z = \sum^r_{i = 1} \sum^{k_i}_{j=1} \frac{c_{ij}(x)}{p_i(x)^j} + h(x)$$ as required.
	\end{enumerate}
\end{solution}

\begin{exercise}(Stichtenoth, Exercise 1.7)
		A valuation ring of  field $L$ is a subring $\OV \subsetneq L$ 
		such that for all $z \in L$ one has $z \in \OV$ or $z^{-1} \in \OV$.
		\begin{enumerate}[(i)]
			\item Show that every valuation ring is a local ring 
			(i.e., it has a unique maximal ideal)
			\item Now we consider the field $L = \Rat$. Show that for every 
			prime number $p\in \Integ$, the set 
			$\Integ_{(p)} := \lbrace a/b \in \Rat \mid a,b \in \Integ , b \notin (p) \rbrace $ 
			is a valuation ring of $\Rat$. What is the maximal ideal of $\Integ_{(p)}$?  
			\item Let $\OV$ be a valuation ring of $\Rat$. 
			Show that $\OV = \Integ_{(p)}$ for some prime number $p$.
	\end{enumerate}
\end{exercise}

\begin{solution}
	We use the definition of a ``valuation ring'' as given above. 
	\begin{enumerate}[(i)]
		\item Let $\OV$ be a valuation ring of a field $L$. 
		We claim that $\m = \OV \backslash \OV^*$ is the only 
		maximal ideal of $\OV$. Let $x \in \m$ and $y \in \OV$. 
		Notice $xy \in \OV^* \Longrightarrow x \in \OV^*$, which 
		is impossible since $x \in \m = \OV \backslash \OV^*$. 
		Hence $xy \in \m$. Let both $x,y \in \m$. Since $\OV$ 
		is a valuation ring, either $xy^{-1} \in \OV$ or $x^{-1}y \in \OV$. 
		Assume $xy^{-1} \in \OV$, then $1 + xy^{-1} \in \OV$. 
		Hence $y + x = y(1+xy^{-1}) \in \m$, since $y \in \m$. 
		So $\m$ is an ideal of $\OV$. Suppose $I$ is another 
		ideal of $\OV$ with $\m \subsetneq I \subset \OV$. 
		Then $I$ must contain a 
		unit of $\OV$, which would imply $I = \OV$. lastly 
		any other maximal ideal $M$ of $\OV$ would be properly 
		contained in $\m$. So $\m$ the only maximal ideal of the 
		local ring $\OV$.

		\item Let $p$ be a prime number and $z = a/b \in \raggedbottom$
		with $gcd(a,b)=1$. If $p \nmid b$, then $z \in \Integ_{(p)}$. 
		If $p \mid b $ but $p \nmid a$, then $z^{-1} = b/a \in \Integ_{(p)}$. 
		If $p \mid a$ and $p \mid b$, then $gcd(a,b) \neq 1$, 
		which contradicts our assumption that $gcd(a,b) =1$. 
		Hence $\Integ_{(p)}$ is a valutation ring of $\raggedbottom$. We claim 
		that the maximal ideal of $\Integ_{(p)}$ is 
		$(p)\Integ_{(p)} := \lbrace a/b \in \raggedbottom 
		\mid a,b \in \Integ, b \notin (p), a \in (p) \rbrace $. 
		We want to show that $(p)\Integ_{(p)} = \Integ_{(p)} \backslash \Integ_{(p)}^*$. 
		Let $z = a/b \in \OV^*$ such that $gcd(a,b)=1$, then 
		$z^{-1} = b/a \in \OV$. That is $p \nmid a$ and $p \nmid b$. 

		\item Let $P$ be the maximal ideal of $\OV$, by theorem
		\ref{primeValuation} there exists $t \in P$ such that 
		$(t) = P$. Write $t = p/q$ for $p,q \in \Integ \backslash \lbrace  0  \rbrace$. 
		Suppose $q \neq 1$. Then $(t) \subsetneqq (p) \subsetneqq \Rat$. 
		Which would contradict the minimality of $P$. Hence we may assume 
		that $t = p$. Suppose $p$ is not prime, then there exists 
		$n,m \in \Integ $ with $n,m >1$ such that $p=nm$. 
		But then $(p) \subsetneqq (n) \subsetneqq \Rat$, again a 
		contradiction. Thus $p$ must be prime.  
	\end{enumerate}
\end{solution}


\end{document}
