\begin{definition}
	An integral domain is said to be {\it normal\/} if it is 
	integrally closed in its field of fractions.
\end{definition}

\begin{proposition} \label {i12biud012oww}
	Every UFD is normal.
\end{proposition}

\begin{proof}
	Let $A $ be a UFD and let $F$ be the field of 
	fractions of $A$. Let $z \in F$, writen $z = a/b$ such that
	$0 \neq b,a \in A$ and $a,b$ share no commmon
	primes in their factorizations. If $z$ is integral over $A$. Then 
	$z^n + a_{n-1}z^{n-1} + ... + a_1z + a_0 = 0$ for some 
	$a_0,a_1,...,a_{n-1} \in A$. That is,
	$(a/b)^n + a_{n-1}(a/b)^{n-1} + ... + a_1 (a/b) + a_0 = 0$. 
	We may clear denominators to obtain 
	$a^n + a_{n-1}a^{n-1}b + ... + a_1 ab^{n-1} + a_0b^n = 0$.
	Then $a^n = -b(a_{n-1}a^{n-1} + ... + a_1 ab^{n-2} + a_0b^{n-1})$. 
	This means $b$ divides $a$. If $b$ is not a unit,
	then this imples that any prime elements
	in the factorization of $b$ appear in the factorization of $a^n$, 
	and thus also in $a$. This is a contradiction to the supposition
	that $a,b$ share no common primes in their factorization. So $b$
	must be a unit. Thus $z \in A$. 
\end{proof}

\begin{example}
	The ring $A = \Comp[x,y]/(y-x^2)$ is normal: 
	Let $\phi : \Comp[x,y] \longrightarrow \Comp[t] $ 
	be given by $ f(x,y) \mapsto f(t,t^2)$. 
	Then $\phi $ is an onto homomorphism with kernel $(y-x^2)$. 
	Thus $\Comp[x,y]/(y-x^2) \cong \Comp[t]$, which is a UFD. 
\end{example}

\begin{example}
	The converse is not true in general. If we 
	let $A=\Comp[x,y,z]/(xy-z^2)$.
	Then $A$ is normal but not a UFD.
	To prove this, let $\phi : \Comp[z,y,z] \longrightarrow \Comp[s,t]$ 
	given as $f(x,y,z) \mapsto f(s^2,t^2,st)$. Then $\phi$ is homomorphism
	an onto the ring $B = \Comp[s^2,t^2,st]$ with kernel $(xy-z^2)$. Hence 
	$A \cong B$. We prove that $B $ is normal by 
	showing that the integral closure of $\Comp[s,t]$ in 
	$\Comp(s,t,\sqrt{st})$ is $\Comp[s,t,\sqrt{st}]$. Let 
	$u + v\sqrt{st} \in \Comp(s,t,\sqrt{st}) $ be integral over 
	$\Comp[s,t]$ for some $u,v \in \Comp(s,t)$. 
	Then since the integral closure of a integral
	domain is an integral domain, $u-v \sqrt{st}$ is in the integral
	closure of $\Comp[s,t]$ in $\Comp(s,t,\sqrt{st})$ as well. 
	Thus their sum, $2u $ belongs to this closure. Since $\Comp[s,t]$ is 
	normal, $u \in \Comp[s,t]$. Similarily, $v\sqrt{st} \in \Comp[s,t]$.
	Hence $v^2st \in \Comp[s,t], v \in \Comp(s,t)$. Clearly, then 
	$v$ can have no denominator, thus $v \in \Comp[s,t]$. Hence 
	$u + v \sqrt{st} \in \Comp[s,t]$. To see that $A$ is not a 
	UFD, notice that $z^2 = xy$. 
\end{example}

\begin{remark}
	Recall that if $R$ is an integral domain and 
	$S \subseteq R \setminus \{0\}$ is a multiplicative set
	then $S^{-1}R$ is an integral domain and
	$R \subseteq S^{-1}R \subseteq \Frac(R)$, so $R$ and $S^{-1}R$ 
	have the same field of fractions.
\end{remark}


\begin{lemma} \label {poifj012woid2q9}
	Let $R$ be an integral domain and $S \subseteq R \setminus \{0\}$ 
	a multiplicative set.
	If $R$ is normal then so is $S^{-1}R$.
\end{lemma}

\begin{proof}
	Let $R$ be an integral domain (not necessarily normal),
	$S \subseteq R \setminus \{0\}$ a multiplicative
	set, $K = \Frac R$, and consider 
	$ R \subseteq \tilde R \subseteq K $
	where $\tilde R$ is the integral closure of $R$ in $K$.
	Then Prop.~5.12 of Atiyah-McDonald implies that
	$S^{-1} \tilde R$ is the integral closure of $S^{-1}R$ in $S^{-1}K=K$. 
	If we now assume that $R$ is normal then $\tilde R = R$, so
	$S^{-1} R$ is the integral closure of $S^{-1}R$ in $K$, i.e., 
	$S^{-1} R$ is normal.
\end{proof}


\begin{proposition}  \label {0923d091sd0192s}
	Let $F/K$ be a rational function field of one variable, 
	let $P$ be a place of $F/K$ of degree $1$,
	and let $\Oeul_{P}$ be the corresponding valuation ring 
	of $F/K$.  Then there exists $t \in F$ satisfying
	$F = K(t) $ and $\Oeul_P =  k[ t^{-1} ]_{(t^{-1})} $.
	Moreover, for any such $t$,
	$K[t] $ is the intersection of all valuation rings that 
	belong to the set $\VV(F/K) \setminus \{ \Oeul_P \}$.
\end{proposition}

\begin{proof}
	Omitted
\end{proof}

let $ k$ be a field, $A =  k[x,y]$ the polynomial ring in 
two variables over $ k$,
$L= \Frac A =  k(x,y)$ the field of rational functions in 
two variables, and $K =  k(F)$.
Let $\VV(F)$ be the set of all valuation rings of the 
function field $L/K$.
Let $\VV^\infty(F) = \setspec{ R \in \VV(F) }{ A \nsubseteq R }$
be the set of dicriticals of $F$.


\begin{definition}
	Let $A =  k[x,y]$ be a polynomial ring in two variables 
	over a field $ k$.
	A {\it field generator\/} of $A= k[x,y]$ is an element
	$F \in A$ that satisfies:
	$\text{$ k(x,y) =  k(F,G)$ for some $G \in  k(x,y)$.}$
	If $F \in A$ satisfies the stronger condition
	$\text{$ k(x,y) =  k(F,G)$ for some $G \in  k[x,y]$}
	$ we call $F$ a {\it good\/} field generator of $A$.
	A field generator that is not good is said to be bad.
\end{definition}

\begin{remark}
	Given any $F \in  k[x,y] \setminus  k$, we know that
	$ k(x,y) /  k(F)$ is a function field
	of one variable. Observe that $F$ is a field generator 
	of $ k[x,y]$ if and only if 
	$ k(x,y) /  k(F)$ is the rational function field.
\end{remark}

\begin{proposition}
	Suppose that $F$ is a field generator of $A= k[x,y]$ 
	such that $1$ occurs in the list $\Delta(F)$, then $F$ is a 
	good field generator.
\end{proposition}

\begin{proof}[Proof by Daniel Daigle:]
	Let $F \in A$ be a field generator of $A$ such that 
	`$1$' occurs in $\Delta(F)$.
	Write $L =  k(x,y)$ and $K =  k(F)$, then $L/K$ is the 
	rational function field of one variable.
	Let $\VV(F)$ be the set of all valuation rings of the 
	function field $L/K$.
	Let 
	$$
	\VV^\infty(F) = \setspec{ R \in \VV(F) }{ A \nsubseteq R } = \{ R_1, \dots, R_s \}
	$$
	be the set of dicriticals of $F$. Since `$1$' occurs in
	$\Delta(F)$, one of $R_1, \dots, R_s$
	is a dicritical of degree $1$; relabelling  $R_1, \dots, R_s$ 
	if necessary, we may arrange that
	$R_1$ is a dicritical of degree $1$. Let  $P$ be the maximal 
	ideal of $R_1$; then
	$$
	\text{$P$ is a place of degree $1$ of the rational function field $L/K$.}
	$$
	Moreover, $R_1$ is the valuation ring of $P$, i.e., $R_1 = \Oeul_P$.
	By \ref{0923d091sd0192s}, there exists $t \in L$ satisfying
	$L = K(t)$, $\Oeul_P =  k[ t^{-1} ]_{(t^{-1})}$, and
	\begin{equation} \label {uqid7623erih9we8}
	K[t] = \bigcap_{\Oeul \in E} \Oeul
	\end{equation}
	where $E = \VV(F) \setminus \{ \Oeul_P \}$.

	Consider the ring $\Aeul = S^{-1}A$ where $S =  k[F] \setminus \{0\} \subset A \setminus \{0\}$. 
	Then
	$$
	\text{$A$ is a UFD} \overset{\ref{i12biud012oww}}{\implies} \text{$A$ is a normal}
	\overset{\ref{poifj012woid2q9}}{\implies} \text{$\Aeul$ is a normal} .
	$$
	Since $\Frac( \Aeul ) = L$, it follows that $\Aeul$ is integrally closed in $L$. Thus, 
	by Cor.~5.22 of Atiyah-McDonald,
	$\Aeul$ is equal to the intersection of all valuation rings $\Oeul$ 
	of $L$ that satisfy $\Aeul \subseteq \Oeul$.
	\begin{equation} \label {bcdkoi23w9eje}
		\Aeul = \bigcap_{\Oeul \in E'} \Oeul
	\end{equation}
	where $E'=$ set of all valuation rings $\Oeul$ of $L$ that satisfy $\Aeul \subseteq \Oeul$.
	Note that $L \in E'$; let us prove that 
	\begin{equation} \label {h9jccnodi1ds}
		E' \subseteq E \cup \{ L \} .
	\end{equation}
	Indeed, consider $\Oeul \in E'$ such that $\Oeul \neq L$, and let us prove that  $\Oeul  \in E$.
	Since $\Oeul$ is a valuation ring of $L$ such that $\Oeul \neq L$ and
	$$
	 k(F) = S^{-1}  k[F] \subseteq S^{-1}A = \Aeul \subseteq \Oeul,
	$$ 
	it follows that $\Oeul$ is a valuation ring of $L/K$, i.e., $\Oeul \in \VV(F)$. Since
	$A \subseteq \Aeul \subseteq \Oeul$, we have $\Oeul \notin \{ R_1, \dots, R_s \}$,
	so $\Oeul \in \VV(F) \setminus \{R_1\} = E$. This proves \eqref{h9jccnodi1ds}.

	It follows that
	\begin{equation} \label {lo923idewo}
		\text{for each $\Oeul \in E'$,\quad  $K[t] \subseteq \Oeul$.}
	\end{equation}

	Indeed, if $\Oeul \in E'$ then \eqref{h9jccnodi1ds} implies that $\Oeul \in E$ or $\Oeul=L$;
	in the first case we have $K[t] \subseteq \Oeul$ by \eqref{uqid7623erih9we8},
	and in the second case we have $K[t] \subseteq L = \Oeul$. So \eqref{lo923idewo} is true.
	It follows from \eqref{lo923idewo} that
	$$
	K[t] \subseteq \bigcap_{\Oeul \in E'} \Oeul\ =\ \Aeul,
	$$
	so in particular $t \in \Aeul = S^{-1}A$; then $t = G/s$ for 
	some $G \in A$ and $s \in S =  k[F] \setminus \{0\}$.
	Since $s \in K^*$, we have $K[t] = K[st] = K[G]$, so 
	$$
	 k(F,G) = K(G) = K(t) = L,
	$$
	showing that $F$ is a good field generator of $A$.
\end{proof}




