\begin{exercise} \label{irriduciblePoverC}
	Consider the poylnomial ring $\Comp [x,y]$. Let $p =y^2 +x^3 - x \in \Comp[x,y]$ 
	and $\pi : \Comp[x,y] \longrightarrow A = \Comp[x,y]/(p)$ by defined by
	$\pi(f) = f + (p)$ for all $f \in \Comp[x,y]$. Let $F$ be the field of 
	fractions of $A$ and $\m \subset A$ be image of the prime ideal $(x,y)$ under $\pi$. 
	Show 
	\begin{enumerate}[(a)]
		\item \label{pisirreducible}$p$ is irreducible in $\Comp[x,y]$
		\item \label{basisisis} $A$ is a free $\Comp[x]$-module, with basis $\{ 1, y \}$.
		\item \label{ellipticfunctionfield} $F/ \Comp $ is a function field in one variable.
		\item \label{localizationValuation} $A_\m$ is a valution ring of the function field $F/ \Comp$.
	\end{enumerate}
\end{exercise}

\begin{solution}
	\eqref{pisirreducible}: Suppose $p = gh $ for $g,h \in \C[x,y] \notzero $. 
	We may view $p$ as a polynomial in one variable $y$ over $\C[x]$. 
	Hence $deg_y(p)= deg_y(gh) = deg(g) + deg(h)$. Since $deg_y(p)=2$, 
	there are three cases for the $y$-degrees of $g$ and $h$: 
	\begin{enumerate}[(i)]
		\item $\deg_y(g) = 1$ and $\deg_y(h) = 1$ 
		\item $\deg_y(g) = 2$ and $\deg_y(h) = 0$ 
		\item $\deg_y(h) = 2$ and $\deg_y(g) = 0$
	\end{enumerate}
	Suppose $deg_y(g)=deg_y(h)=1$. 
	Write $g= a_1y + a_2$ and $h= b_1y + b_2$ where
	$a_1,a_2,b_1,b_2 \in \C[x]$. 
	Then $p = gh = (a_1y + a_2)(b_1y + b_2) = a_1b_1y^2 + (a_1b_2 + a_2b_1)y + a_2b_2 $.
	Thus we have the equations;
	\begin{align*}
		&a_1b_1 = 1  \\
		&a_1b_2 + a_2b_1 = 0 \\
		&a_2b_2 = x^3 - x 
	\end{align*} 
	$a_1b_1 = 1 \Longrightarrow a_1 = b_1^{-1}$. 
	Then $a_1b_2 + a_2b_1 = 0$ becomes
	$b_2=- b_1^2a_2$. Then in $a_2b_2 = x^3 - x $ we have
	$a_2b_2 = - a_2^2b_1^2 = - (a_2b_1)^2 = x^3 - x$ which is 
	impossible since $x^3 - x = x(x-1)(x+1)$ is not a 
	square. So either $f$ or $g$ is a unit. 
	If $\deg_y(g) = 2$ and $\deg_y(h) = 0$, then we 
	may write $g = a_1y^2 + a_2y + a_3$ and $h = b$ for some 
	$a_1,a_2,a_3,b \in \Comp[x]$. 
	Then $$y^2 +x^3 - x = g = a_1by^2 + a_2by + a_3b$$
	This would mean that $a_1b= 1$. Hence $b \in \Comp[x]^* $.
	So $b$ must be a unit of $\Comp[x,y]$ as well. For the last
	case (where $\deg_y(h) = 2$ and $\deg_y(g) = 0$), the same 
	argument will show that $g \in \Comp[x,y]^*$.
	Thus $p$ is irreducibe in $\C[x,y]$.  \\

	\eqref{basisisis}: Suppose $a+by = 0$ for nonzero $a,b \in \Comp[x]$. 
	Recall the ``re-definition'' $x = \pi(x)$ and $y = \pi(y)$. To 
	avoid confusion, let $X,Y$ be used to represent variables in 
	$\Comp[X,Y]$ and $x,y$ be used to represent variables in $A$. 
	Then $a+bY \in \text{ker}(\pi) = (p)$ - but this is impossible since 
	$\deg_Y(p) = 2 \geq deg_Y(a+bY) = 1$. This means that $\lbrace 1,y \rbrace$
	is an independent set in $A$. Clearly $\text{span} \lbrace 1,y \rbrace \subseteq A$.
	Let $f \in A$. Then $$f = a_0 + a_ 1x + a_2y + a_3xy + a_4 x^2 
	+ a_5y^2 + a_6x^2y + a_7xy^2 + a_8x^2y^2 + ... + a_nx^ny^m$$
	for some $a_0,a_1,...,a_n \in \Comp$ where $y^2+x^3 - x = 0$. 
	In each term of $f$ divisible by $y^2$, substitute $y^2$ with $x - x^3$. 
	Then $$f = ... + a_3xy + a_4 x^2 + a_5(x - x^3) + a_6x^2y + a_7x(x - x^3) 
	+ a_8x^2(x - x^3) + ... + a_nx^n(x - x^3)^ky^l$$
	where $k = m/2, l = 0$ if $m$ is even and $k = (m-1)/2, l = 1$ if $m$ is odd.
	Factoring out the $y$ in some of terms, we may rearrange $f$ as;
	$$f =  a_0 + a_ 1x +  a_4 x^2  + a_5(x - x^3)  + a_7x(x - x^3) + ... + 
	(a_2 +  a_3x  + a_6x^2 + ... +a_nx^n(x - x^3)^k)y $$
	Hence $f \in \text{span}_{\Comp[x]} \lbrace 1,y \rbrace$. 
	So $A$ is a free $\Comp[x]$-module, with basis $\{ 1, y \}$.\\

	\eqref{ellipticfunctionfield}: Since $\Comp \cap (p) = \lbrace 0 \rbrace $, 
	the composition of the cannonical projection map
	of the quotient ring $A$ with the inclusion homomorphism $f \mapsto f/1 $ of 
	$A$ to $F$ embeds $\Comp $ in $F$. So we may view $\Comp $ as a subfield of
	$F$. To show that $F/ \Comp$ is a transcendental extension of 
	fields, condsider $y\in F$. Suppose $y$ is algebraic over 
	$\Comp $, then $a_0 + a_1y + ... + a_ny^n \in (p)$ 
	for some $a_0,...,a_n \in \Comp$. That would mean 
	that $a_0 + a_1y + ... + a_ny^n = g(x,y)(p) = g(x,y)(y^2 + x - x)$ 
	for some nonzero $g(x,y) \in \Comp[x,y]$.
	But $\deg_x(a_0 + a_1y + ... + a_ny^n) = 0$ and 
	$\deg_x (y^2 +x^3 -x) = 3$. So there does not exists such a $g$.
	Hence $y$ is transcendental over $\Comp$.
	We want to show that $F/\Comp(x)$ is finite.
	Let $0 \neq z \in F$. Write $z = f/g$ for $f,g\in A$ with $g \neq 0$.
	By \eqref{basisisis}, we may write $f = a + by$ for 
	$a,b \in \Comp[x]$. 
	Since $0 \neq z$, we may assume $a\neq 0$ or  $b \neq 0$. 
	\textit{Claim}: If $m(T) \in \Comp(x) [T] \backslash \lbrace 0 \rbrace$ is 
	some polynomial satisfying $m(f) = 0$. Then we may find another 
	nonzero polynomail $m^{\prime}(T)$, that depends on $g$, such 
	that $m(z) = 0$. \\

	Proof of claim:
	Assume there exists a nonzero polynomial $m(T) \in \Comp(x)[T]$ 
	such that $m(f) = 0$. Write $m(T) = a_0 + a_1T + ... + a_nT^n$ 
	for $a_1,...,a_n \in \Comp(x)$. Consider the terms in the polynomial $g$.
	Write $g(x,y) = \sum_{i,j\in \Nat } b_{ij} x^iy^j$. 
	If each term of $g$ is either in $\Comp(x)$ or divisible by $y^2$,
	then we may multiply $m(T)$ by $h(x) = (\sum_{i,j\in \Nat } b_{ij} x^i(x-x^3)^j)^n$
	Then all the denominators of $m(z)$ will be cleared by $h(x)$, which 
	belongs to $\Comp(x)$. So if we define $m^{\prime}(T) = h(x)m(T)$,
	then $m^{\prime} (z) = 0$. Alternatively, suppose any of the terms
	of $g$ are of the form $uy^n$ for $u\in \Comp(x)$ and $n$ odd.
	We may assume that only one of the terms of $g$ is of this form 
	and that $n = 1$: If more of the terms are of this form then 
	repeat the following process again. 
	The case $n > 1 $, follows by induction. So we may write
	$g = a + by$ for some $a,b \in \Comp(x)$. let $h(x) = (a^2 - b^2(x-x^3)^2)^n$ and 
	let $m^{\prime}(T) = h(x)m(T)$. 
	Thus
	\begin{align*}
		m^{\prime}(z) &= (a^2 - b^2(x-x^3)^2)^n(a_0 + a_1(\frac{f}{a+by}) + ... + a_n(\frac{f}{a+by})^n) \\
		&= (a+by)^n(a-by)^n (a_0 + a_1(\frac{f}{a+by}) + ... + a_n(\frac{f^n}{(a+by)^n})) \\
		&= (a+by)^n(a-by)^na_0 + a_1(a+by)^{n-1}(a-by)^nf + ... + a_nf^n(a-by)^n \\
		&= 0  
	\end{align*}

	So it suffices to find a nonzero polynomial $m(T) \in \Comp (x)[T]$
	such that $m(f) = 0$. Let $m(T) = (T-a)^{2} - b^2(x-x^3)$. Then 
	clearly $m(a+by) = (a+by - a)^2 - b^2y^2 = 0$. Since $a \neq 0 $ 
	or $b \neq 0$, it follows that $m(T)$ is nonzero. Hence
	$F/\Comp(x)$ is finite. Thus $F/\Comp$ is a function field \\

	\eqref{localizationValuation}: By definition, 
	$A_\m = \lbrace f/g \mid f,g \in A , g \neq 0 \text{ and } g \notin \m \rbrace $.
	So $\Comp \subsetneqq A_\m \subsetneqq F$ is clear. Let $z \in F$. 
	Write $z = f/g$ for $f,g \in A$ and $g \neq 0$. Suppose $f,g \in \m$. 
	Then we may write $f = ax+by,g = cx + dy$ for $a,b,c,d \in \C[x,y]$.
	Thus
	\begin{align*}
		z &= f/g \\
		&= \frac{ax+by}{cx + dy} \\
		&= \frac{(ax+by)(cx - dy)}{(cx + dy)(cx - dy)} \\
		&= \frac{acx^2 + (cb-ad)xy - bdy^2}{c^2x^2 -d^2y^2} \\
		&= \frac{acx^2 + (cb-ad)xy - bd(x-x^3)}{c^2x^2 -d^2(x-x^3)} \\
		&= \frac{acx + (cb-ad)y - bd(1-x^2)}{c^2x -d^2(1-x^2)} \\
		&= \frac{acx + (cb-ad)y - bd + bdx^2}{c^2x -d^2 + d^2x^2} 
	\end{align*}

	Notice, if $b\neq 0, d \neq 0$, then both the numerator and denominator 
	do not belong to
	$\m$ since $\m \cap \Comp = \lbrace 0 \rbrace$. So $z \in A_\m^*$.
	If $d = 0$ then $z = a/c$. If both $a,c\in \m$, then we
	may repeat the same process again. If both $b = 0$ and $d \neq 0$. 
	Then only $z \in A_\m^*$. Thus $A_\m $ is a valuation ring of $F$.
\end{solution}

\begin{exercise} \label{primeontoprime}
	Let $A \stackrel{\phi}{\longrightarrow} B$ be a surjective homomorphism of rings. 
	Let $\pgoth $ be a prime ideal in $R$ containing the kernel of $\phi$. 
	Then $\phi(\pgoth)$ is prime in $S$.
\end{exercise}

\begin{solution} 
	Let $x,y \in B$ and suppose $xy \in \phi(\mathfrak{p})$. Since $\phi$ is 
	surjectivity, there exists $a, b \in A$ such that $\phi(a)= x$, $\phi(b)=y$. 
	Choose $c \in \mathfrak{p}$ such that $\phi(c) = xy$.  
	Then $ab -c \in ker(\phi)$, so $ab \in \mathfrak{p}$.  
	Thus, either $a$ or $b$ is in $\mathfrak{p}$, which means 
	either $x$ or $y$ is in $\phi(\mathfrak{p})$.
\end{solution}

    
\begin{exercise}(Stichtenoth, Exercise 1.1)
	Consider the rational function field $K(x)/K$ and a non-constant element 
	$z = f(x)/g(x) \in K(x) \backslash K$, where $f(x),g(x) \in K[x]$ are 
	relatively prime. We call $deg(z) = max \lbrace deg(f) , deg(g) \rbrace $ 
	the degree of $z$.
	\begin{enumerate}[(i)]
		\item Show that $[K(x): K(z)] = deg(z)$, and write down the minimal 
		polynomial of $x$ over $K(z)$
		\item Show that $K(x)=K(z) $ if and only if $z = (ax+b)/(cx+d)$ with
		$a,b,c,d \in K$ and $ad - bc \neq 0$.
	\end{enumerate}
\end{exercise}

\begin{solution} 
	We find the minimal polynomial of the field extension $K(x)/K(z)$.
	\begin{enumerate}[(i)]

		\item Consider the polynomial $m(t) = zg(t) - f(t) \in K(z)[t]$. 
		Notice that $0 \neq z = f(x)/g(x) \Longrightarrow f(x) \neq 0 \Longrightarrow m(t) \neq 0$
		and that $m(x) = 0$. Also, if $deg(g(x)) \geq deg(f(x))$, 
		then $deg(m(t)) = deg(g(t)) = deg(g(x))$.
		Otherwise $deg(m(t)) = deg(f(t)) = deg(f(x))$. So $deg(m(t)) = deg(z)$,
		as required. Lastly we need to show that $m(t)$ is irreducible over
		$K(z)$. By Gauss's lemma, it is sufficient to check that $m(t)$ is 
		irreducible over $K[z]$ but $K[z][t]=K[t][z]$, in which $m(t)$ is linear. 
		Hence $m$ is the minimal polynomial of the field extension $K(x)/K(z)$ 
		and $[K(x):K(z)] = deg(m(t)) = deg(z)$ as required.

		\item Assume $z = (ax+b)/(cx + d)$ with $a,b,c,d \in K$ and 
		$ad - cd \neq 0$. The condition $ad-bc \neq 0 $ implies that 
		$z \notin K$. By part $i)$, $[K(x):K(z)] = deg(z) = 1$, Hence 
		$K(x) = K(z)$. Assume $K(x) = K(z)$ and suppose $z \neq (ax+b)/(cx + d) $ 
		for any $a,b,c,d \in K$ satisfying $ad - bc \neq 0$. Then 
		either $z \in K$ or $deg(z) \geq 2$. If $z \in K$, then $K(z) = K$, 
		which would imply $K(x) = K$, a contradiction to the fact 
		that $x$ is transcendental over $K$. If $deg(z) \geq 2$, 
		then by part $i)$, $[K(x):K(z)] \geq 2$. This contradicts the fact 
		that every field is a one dimensional vector space over itself. 
		So our supposition must be false, thus $z = (ax+b)/(cx+d)$ with 
		$a,b,c,d \in K$ and $ad - bc \neq 0$.
	\end{enumerate}
\end{solution}

\begin{exercise}(Stichtenoth, Exercise 1.2) \label{GLP2}
	For a field extension $L/M$ we denote by $Aut(K(x)/K)$ the group 
	of automorphisms of $L/M$ (i.e., automorphisms of $L$ which are 
	the identity on $M$). Let $K(x)/K$ be the rational function field 
	over $K$. Show:

	\begin{enumerate}[(i)]
		\item \label{existsabcd} For every $\sigma \in Aut(K(x)/K)$ there 
		exists $a,b,c,d \in K$ such that $ad-bc \neq 0$ and $\sigma (x) = (ax+b)/(cx+d)$. 

		\item Given $a,b,c,d \in K $ with $ad - bc \neq 0$, there 
		is a unique automorphism $\sigma \in Aut(K(x)/K)$ with $\sigma(x) = (ax+b)/(cx+d)$.

		\item Denote by $GL_2(K)$ the group of invertible $2 \times 2$ 
		- matrices over $K$. For $A= 
		\begin{pmatrix} a & c \\ b & d \\ \end{pmatrix} \in GL_2(K)$ 
		denote by $\sigma_A$ the automophism of $K(x)/K$ with $\sigma_A(x) = (ax+b)/(cx+d) $. 
		Show that the map that sends $A$ to $\sigma_A$, is a homomorphism of $GL_2(K) $ 
		onto $Aut(K(x)/K)$. Its kernel is the set of diagonal matrices of the 
		form $\begin{pmatrix} a & 0 \\ 0 & a \\ \end{pmatrix}$ with $a \in K^{\times}$, 
		hence $$Aut(K(x)/K) \cong GL_2(K)/K^{\times}$$ (The group $GL_2(K)/K^{\times}$ 
		is called the projective linear group and is denoted by $PGL_2(K)$.)
	\end{enumerate}
\end{exercise}

\begin{solution} 
	Any given tuple $(a,b,c,d) \in K^4$ will satisfy $ad - bc = 0$ 
	unless otherwise specified.
	\begin{enumerate}[(i)]
		\item Let $\sigma \in Aut(K(x)/K)$. We show that $K(x) = K(\sigma(x)) $. 
		Let $f \in K(\sigma(x))$, then 
		$$
		f(\sigma(x)) = \frac{a_n\sigma(x)^n + a_{n-1}\sigma(x)^{n-1} + ... + a_0 }{b_m\sigma(x)^m + b_{m-1}\sigma(x)^{m-1} + ... + b_0 } 
		$$ 
		for $a_0,...,a_n,b_0,...,b_m \in K$. Since $\sigma(a) = a $ 
		for all $a \in K$ and $\sigma$ is a homomorphism, we may 
		rewrite $f$ as $$\sigma (\frac{a_nx^n + a_{n-1}x^n + ... + a _0}{b_mx^m + b_{m-1}x^{m-1} + ... + b_0}) \in im(\sigma)$$ 
		Since $\sigma$ is surjective, $f \in im(\sigma) = K(x)$. \\

		Let $f \in K(x)$, then $f \in im(\sigma)$ by surjectivity of $\sigma$. 
		Hence there exists some $g\in K(x)$ such that $\sigma(g(x)) = f(x)$.
		Write $$g(x) = \frac{a_nx^n + a_{n-1}x^n + ... + a _0}{b_mx^m + b_{m-1}x^{m-1} + ... + b_0}$$ 
		then $$
		f(x)= \sigma(\frac{a_nx^n + a_{n-1}x^n + ... + a _0}{b_mx^m + b_{m-1}x^{m-1} + ... + b_0}) 
		= \frac{a_n\sigma(x)^n + a_{n-1}\sigma(x)^{n-1} + ... + a_0 }{b_m\sigma(x)^m + b_{m-1}\sigma(x)^{m-1} + ... + b_0 } 
		$$ 
		Hence $f \in K(\sigma(x))$. That is, $K(x) = K(\sigma(x))$ as required.
		By part (ii) of exercise 1, $\sigma(x) = (ax+b)/(cx+d)$.

		\item Let $a,b,c,d \in K$, define $\sigma : K(x) \longrightarrow K(x) $ by $\sigma(f(x)/g(x))$ 

		\item Let $\Phi : GL_2(K) \longrightarrow Aut(K(x)/K) $ be defined as
		$\Phi (A) = \sigma_A $ for all $A \in GL_2(K)$. 
		\begin{remark}
			To verify that two automorphisms $\sigma_2,\sigma_1 \in Aut(K(x)/K)$ 
			are equivalent, it suffices so show that $\sigma_1(x)=\sigma_2(x)$. 
			This is because every automorphism of $Aut(K(x)/K)$ is uniquely 
			determined by $x$. 
		\end{remark}
			First, $\Phi$ preserves
			identity: $\Phi(\begin{pmatrix} 1 & 0 \\ 0 & 1 \\ \end{pmatrix})(x) = (1x + 0)/(0x +1) = x = i(x)$
			where $i$ is the indentiy automorphism of $Aut(K(x)/K)$. Let $A,B \in GL_2(K)$.
			Then $A = \begin{pmatrix} a & c \\ b & d \\ \end{pmatrix} , B = \begin{pmatrix} e & g \\ f & h \\ \end{pmatrix} $ 
			for $a,b,c,d,e,f,g,h \in K$. Then
			\begin{align*}
				\Phi (\begin{pmatrix} a & c \\ b & d \\ \end{pmatrix}
				\begin{pmatrix} e & g \\ f & h \\ \end{pmatrix})(x) &= 
				\Phi (\begin{pmatrix} ae + cf & ag + ch \\ be + df & bg + dh \\ \end{pmatrix})(x) \\
				&= \frac{(ae+cf)x + be+df}{(ag+ch)x+bg+dh} (x)\\
				&= \frac{e(ax+b) + (cx+d)f}{g(ax+b)+ (cx+d)h} \\
				&= \frac{[e(ax+b) + (cx+d)f](cx+d)}{[g(ax+b)+ (cx+d)h](cx+d)} \\
				&= \frac{\frac{e(ax+b) + (cx+d)f}{(cx+d)}}{\frac{g(ax+b) + (cx+d)h}{(cx+d)}} \\
				&= \frac{\frac{e(ax+b)}{(cx+d)}+f}{\frac{g(ax+b)}{(cx+d)}+h} \\
				&= \frac{e\sigma_A(x)+f}{g\sigma_A(x)+h} \\
				&= \sigma_A(\frac{ex+f}{gx+h})(x) \\
				&= \sigma_A \circ \sigma_B (x) \\
				&= \Phi(\begin{pmatrix} a & c \\ b & d \\ \end{pmatrix}) \Phi(\begin{pmatrix} e & g \\ f & h \\ \end{pmatrix}) (x)
				\end{align*}
			\noindent Hence $\Phi$ is a group homomorphism. Clearly
			$\Phi(\begin{pmatrix} a & 0 \\ 0 & a \\ \end{pmatrix})(x) = \frac{ax + 0}{0x +a} = x = i(x)$. 
			So $\begin{pmatrix} a & 0 \\ 0 & a \\ \end{pmatrix} \in ker(\Phi)$. 
			Let $A = \begin{pmatrix} a & c \\ b & d \\ \end{pmatrix} \in ker(\Phi)$, 
			then $\frac{ax+b}{cx + d} = x \Rightarrow 0 = cx^2 + (d-a)x - b \Rightarrow a = d, b = c = 0$. 
			Notice that $a=d = 0$ implies $ad - bc = 0$, which is not allowed here. 
			Hence $ker(\Phi) = \begin{pmatrix} a & 0 \\ 0 & a \\ \end{pmatrix} $
			where $a \in K^{\times}$. Let $\sigma \in Aut(K(x)/K)$, 
			then by part \eqref{existsabcd}, there exists $a,b,c,d \in K$ such that 
			$\sigma(x) = \frac{ax+b}{cx+d}$, that is there exists 
			$A = \begin{pmatrix} a & 0 \\ 0 & a \\ \end{pmatrix} \in GL_2(K)$ such that 
			$\Phi(A) = \sigma$. Hence $\Phi$ is surjective and $Aut(K(x)/K) \cong GL_2(K)/\mathfrak{K}$,
			where $\mathfrak{K} = \begin{pmatrix} a & 0 \\ 0 & a \\ \end{pmatrix}$ for $a \in K^{\times}$.
	\end{enumerate}
\end{solution}
    

\begin{exercise}(Stichtenoth, Exercise 1.4)
	Let $K(x)$ be the rational function field over $K$. Find 
	bases for the following Riemann-Roch spaces: 
	\begin{enumerate}[(i)]
		\item \label{infinity} 		$\mathscr{L}(rP_{\infty})$
		\item \label{linear}   		$\mathscr{L}(rP_{\alpha})$
		\item \label{irreducible11} 	$\mathscr{L}(P_{p(x)})$
	\end{enumerate}
	where $R \geq 0$, and the places $P_{\infty}, P_{\alpha}$ and $P_{p(x)}$
	are as in section $1.2$ of Stichtenoth. 
\end{exercise}

\begin{solution}
	\eqref{infinity}: Let $r \geq 0$. $\mathscr{L}(rP_{\infty})$: Write $z \in K(x)^*$ as $z = f/g$ 
	where $f,g$ are relatively prime in $K[x]$. Then $z \in \mathscr{L}(rP_\infty)$ 
	if and only if $v_P(z) + v_P(rP_\infty) \geq 0 $ for all 
	$P \in \mathbb{P}_{K(x)}$ if and only if $v_P(z) + 0 \geq 0 $ 
	for all $P\in \mathbb{P}_{K(x)} \backslash \lbrace P_\infty \rbrace $ 
	and $v_\infty(z) + r \geq 0$ if and only if $z \in K[x]$ and 
	$r \geq \deg(f)$. Hence $\mathscr{L}(r P_\infty)$ is the vector space 
	of all polynomial in $K[x]$ with degree less than or equal to 
	$r$. Transcedance of $x$ over $K$ implies $1,x,x^2,...,x^r$ 
	are linearly independent over $K$ and clearly they span 
	$\mathscr{L}(rP_\infty)$. Hence $\ell(rP_\infty) = r+1$. 

	\eqref{linear}: Since $P_a$ corresponds to the linear polynomial
	$p = x - a$, so $\deg P = 1$. By proposition \ref{0923d091sd0192s}, there 
	exists $t \in K(x)$ such that $\OV_{P_a} = K[t^{-1}]_{(t^{-1})}$. Then
	from the solution to \eqref{infinity}, $\lbrace 1,t,t^2,...,t^r \rbrace$ is a $K$-basis
	for $\mathscr{L}(rP_{\alpha})$ with dimension $r + 1$.

	\eqref{irreducible11}: Let $n = \deg p(x)$. Then since 
	$p(x)$ is irriducible over $K$, all other places of 
	the form $P_{q(x)}$ for some irreducible polynomial $q(x) \in K[x]$
	will have valuation $0$. Thus by theorem \ref{Princ0}, $v_\infty(p) = -n$.
	Hence if $z \in \LL(P_{p(x)})$, then $v_\infty(z) \geq -n$. 
	Similarly, $v_P(z) \geq 0$ for all $P \in \PF \backslash \lbrace P_{p(x)},P_\infty \rbrace$.
	So $z \in K[x]$. This implies that $z$ is a polynomial in $K[x]$ with degree 
	less than or equal to $n$. Then clearly $\LL(P_{p(x)})$ has $K$-basis $\lbrace 1,x,...,x^{n} \rbrace$,
	which implies $\ell(P_{p(x)}) = n+1$.
\end{solution}
  
\begin{exercise}(Stichtenoth, Exercise 1.5) \label{PForPartial}
    
(\textit{Representation of rational funcitons by partial fractions})
    
\begin{enumerate}[(i)]
	\item Show that every $z \in K(x)$ can be written as 
	$$z = \sum^r_{i = 1} \sum^{k_i}_{j=1} \frac{c_{ij}(x)}{p_i(x)^j} + h(x)$$ where 

	\begin{enumerate}
		\item $p_1(x),...,p_r(x)$ are distinc monic irreducible polynomials in $K[x]$,
		\item $k_1,...,k_r \geq 1$,
		\item $c_{ij}(x) \in K[x]$ and $deg(c_{ij}(x)) < deg(p_i(x))$,
		\item $c_{ik_i}(x) \neq 0$ for $1 \leq i \leq r$,
		\item $h(x) \in K[x]$
	\end{enumerate}
	\item Show that the above representation of $z$ is unique.
\end{enumerate}
\end{exercise}

\begin{solution}
	\begin{enumerate}[(i)] 
		\item Let $z = f(x)/g(x) \in K(x)$. If $\deg(f) \geq \deg(g)$, 
		then we may write $f = q(x)g(x) + r(x)$ with $q(x),r(x) \in K[x]$ and
		$\deg(r(x)) < \deg(g(x))$. Then $f(x)/g(x) = q(x) + r(x)/g(x)$ and we may use 
		the following argument for $r(x)$ instead of $f(x)$.
		Hence it suffices to consider the case 
		where $\deg(g(x)) > \deg(f(x))$. Write $g(x) = p_1(x)^{e_1}p_2(x)^{e_2}...p_s(x)^{e_s}$ 
		where $e_1,e_2,...,e_s \in \Integ^{+}$ and $p_1,p_2,...,p_s$ are distinct 
		prime factors of $g(x)$. Suppose $deg(f) < deg(g)$. Then 
		$$\text{gcd}(p_1(x)^{e_1},p_2(x)^{e_2}...p_s(x)^{e_s}) = 1$$ hence there exists 
		$h_1(x),h_2(x) \in K[x]$ such that $$1 = h_1(x)p_1(x)^{e_1} + h_2(x)p_2(x)^{e_2}...p_s(x)^{e_s}$$ 
		Thus $$f(x) = f(x)h_1(x)p_1(x)^{e_1} + f(x)h_2(x)p_2(x)^{e_2}...p_s(x)^{e_s}$$ 
		We may divide and write $h_1(x)f(x) = p_1(x)^{e_1}q(x) + r_1(x)$ 
		for some $q(x),r_1(x) \in K[x]$ with $deg(r(x)) < deg(q(x))$. 
		Let $$r_2(x) = h_2(x)p_1(x)^{e_1}p_2(x)^{e_2}...p_s(x)^{e_s} + q(x)p_2(x)^{e_2}...p_s(x)^{e_s}$$ 

		\begin{align*} 
			f(x) &= p_1(x)^{e_1}r_2(x) - p_1(x)^{e_1}p_2(x)^{e_2}...p_s(x)^{e_s}q(x) \\
			&+ p_1(x)^{e_1}p_2(x)^{e_2}...p_s(x)^{e_s}f(x) + p_2(x)^{e_2}...p_s(x)^{e_s}r_1(x) \\
			&= p_1(x)^{e_1}r_2(x) + p_2(x)^{e_2}...p_s(x)^{e_s}r_1(x)
		\end{align*}
		Hence $\frac{f(x)}{g(x)} = \frac{r_1(x)}{p_1(x)^{e_1}} + \frac{r_2(x)}{p_2(x)^{e_2}...p_s(x)^{e_s}} $. 
		Claim: $deg(r_2(x)) < deg(p_2(x)^{e_2}...p_s(x)^{e_s})$: Suppose the opposite, 
		then $deg(p_1(x)^{e_1}r_2(x)) \geq deg(p_1(x)^{e_1}p_2(x)^{e_2}...p_s(x)^{e_s})$ 
		but we also have 
		$$deg(p_1(x)^{e_1}p_2(x)^{e_2}...p_s(x)^{e_s}) >  
		deg(deg(p_1(x)^{e_1}p_2(x)^{e_2}...p_s(x)^{e_s}r_1(x))$$
		Then
		    
		\begin{align*}
			deg(f(x)) &= deg(p_1(x)^{e_1}r_2(x) + p_2(x)^{e_2}...p_s(x)^{e_s}r_1(x)) \\
			&= deg(p_1(x)^{e_1}p_2(x)^{e_2}...p_s(x)^{e_s}r_2(x)) \\
			&\geq deg(p_1(x)^{e_1}p_2(x)^{e_2}...p_s(x)^{e_s}) \\
			&= deg(p_1(x)^{e_1}) + deg(p_2(x)^{e_2}...p_s(x)^{e_s}) \\
			&< deg(f(x))
		\end{align*}

		This is a contradiction, so the claim is verified. 

		Hence we may 
		repeat this processess $s-1$ times to obtain the expression
		$$\frac{f(x)}{g(x)} = \sum^s_{i=1}\frac{r_i(x)}{p_i(x)^{e_i}}$$ 
		We now need to expand the powers of $p_i(x)$ for $i=1,...,s$. 
		For $i=1,...,s$, let $r_{i0}(x) = p_i(x)$  and for $j=1,...,e_s$ 
		we can use the division algorithm to find $q_{ij}(x),r_{ij}(x) \in K[x]$ 
		such that $$q_{i(j-1)}(x) = q_ij(x)p_i(x) + r_{ij}(x)$$ where $\deg(r_{ij}(x)) < \deg(q_{ij}(x))$.
		Using back substitution, we find 
		\begin{align*}
			r_i(x)_i(x) &= q_{i0}(x) \\
			&= q_{i1}(x)p_i(x) + r_{i1}(x) \\
			&= (q_{i2}(x)p_i(x)+r_{i2}(x))p_i(x) + r_{i1}(x) \\
			&= ... \\
			&= q_{ie_{s}}(x)p_i(x)^{e_s} + r_{ie_{s}}(x)p_i(x)^{e_s -1} \\
			&+ r_{i(e_s -1)}(x)p_i(x)^{e_s -2} + ... + r_{i2}(x)p_i(x) + r_{i1}(x) \\
			\Longrightarrow \frac{r_i(x)}{p_i(x)^{e_s}} &= q_{ie_s}(x) + 
			\frac{r_{ie_s}(x)}{p_i(x)}+ \frac{r_{i(e_s -1)}(x)}{p_i(x)^2} + 
			...+ \frac{r_{i2}(x)}{p_i(x)^{e_s - 1}} + \frac{r_{i1}(x)}{p_i(x)^{e_s}}
		\end{align*}
		Hence $$\frac{f(x)}{g(x)} = \sum^s_{i=1}(q_{ie_s}(x) + \sum^i_{j=1}  \frac{r_{ij}(x)}{p_i(x)^j})$$
		Let $h(x) = \sum^s_{i=1}q_{ie_s}(x)$, $c_{ij}(x)=r_{if(x)}$, $e=k$ and $r = s$. Then 
		$$z = \sum^r_{i = 1} \sum^{k_i}_{j=1} \frac{c_{ij}(x)}{p_i(x)^j} + h(x)$$ as required.
	\end{enumerate}
\end{solution}

\begin{exercise}(Stichtenoth, Exercise 1.7)
		A valuation ring of  field $L$ is a subring $\OV \subsetneq L$ 
		such that for all $z \in L$ one has $z \in \OV$ or $z^{-1} \in \OV$.
		\begin{enumerate}[(i)]
			\item Show that every valuation ring is a local ring 
			(i.e., it has a unique maximal ideal)
			\item Now we consider the field $L = \Rat$. Show that for every 
			prime number $p\in \Integ$, the set 
			$\Integ_{(p)} := \lbrace a/b \in \Rat \mid a,b \in \Integ , b \notin (p) \rbrace $ 
			is a valuation ring of $\Rat$. What is the maximal ideal of $\Integ_{(p)}$?  
			\item Let $\OV$ be a valuation ring of $\Rat$. 
			Show that $\OV = \Integ_{(p)}$ for some prime number $p$.
	\end{enumerate}
\end{exercise}

\begin{solution}
	We use the definition of a ``valuation ring'' as given above. 
	\begin{enumerate}[(i)]
		\item Let $\OV$ be a valuation ring of a field $L$. 
		We claim that $\m = \OV \backslash \OV^*$ is the only 
		maximal ideal of $\OV$. Let $x \in \m$ and $y \in \OV$. 
		Notice $xy \in \OV^* \Longrightarrow x \in \OV^*$, which 
		is impossible since $x \in \m = \OV \backslash \OV^*$. 
		Hence $xy \in \m$. Let both $x,y \in \m$. Since $\OV$ 
		is a valuation ring, either $xy^{-1} \in \OV$ or $x^{-1}y \in \OV$. 
		Assume $xy^{-1} \in \OV$, then $1 + xy^{-1} \in \OV$. 
		Hence $y + x = y(1+xy^{-1}) \in \m$, since $y \in \m$. 
		So $\m$ is an ideal of $\OV$. Suppose $I$ is another 
		ideal of $\OV$ with $\m \subsetneq I \subset \OV$. 
		Then $I$ must contain a 
		unit of $\OV$, which would imply $I = \OV$. lastly 
		any other maximal ideal $M$ of $\OV$ would be properly 
		contained in $\m$. So $\m$ the only maximal ideal of the 
		local ring $\OV$.

		\item Let $p$ be a prime number and $z = a/b \in \raggedbottom$
		with $gcd(a,b)=1$. If $p \nmid b$, then $z \in \Integ_{(p)}$. 
		If $p \mid b $ but $p \nmid a$, then $z^{-1} = b/a \in \Integ_{(p)}$. 
		If $p \mid a$ and $p \mid b$, then $gcd(a,b) \neq 1$, 
		which contradicts our assumption that $gcd(a,b) =1$. 
		Hence $\Integ_{(p)}$ is a valutation ring of $\raggedbottom$. We claim 
		that the maximal ideal of $\Integ_{(p)}$ is 
		$(p)\Integ_{(p)} := \lbrace a/b \in \raggedbottom 
		\mid a,b \in \Integ, b \notin (p), a \in (p) \rbrace $. 
		We want to show that $(p)\Integ_{(p)} = \Integ_{(p)} \backslash \Integ_{(p)}^*$. 
		Let $z = a/b \in \OV^*$ such that $gcd(a,b)=1$, then 
		$z^{-1} = b/a \in \OV$. That is $p \nmid a$ and $p \nmid b$. 

		\item Let $P$ be the maximal ideal of $\OV$, by theorem
		\ref{primeValuation} there exists $t \in P$ such that 
		$(t) = P$. Write $t = p/q$ for $p,q \in \Integ \backslash \lbrace  0  \rbrace$. 
		Suppose $q \neq 1$. Then $(t) \subsetneqq (p) \subsetneqq \Rat$. 
		Which would contradict the minimality of $P$. Hence we may assume 
		that $t = p$. Suppose $p$ is not prime, then there exists 
		$n,m \in \Integ $ with $n,m >1$ such that $p=nm$. 
		But then $(p) \subsetneqq (n) \subsetneqq \Rat$, again a 
		contradiction. Thus $p$ must be prime.  
	\end{enumerate}
\end{solution}
