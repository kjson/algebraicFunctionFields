	

\begin{definition} \label{Divisor}
	Let $F/k $ be a function field over a 
	field $k$. The divisor group of $F/k$ is defined 
	as the free abelian group which is generated by 
	the places of $F/k$, denoted $\text{Div}(F)$. The 
	elements of $\text{Div}(F)$ are called divisors of $F/k$. 
	In other words; a divisor is a formal sum;
	$$ D = \sum_{P \in \PF } n_P P $$ with $n_P \in \Z$, 
	almost all $n_P = 0$. Two divisors $D = \sum n_P P$ and
	$D' = \sum n'_P P $ are added coefficientwise;
	$$ D + D' = \sum_{p \in \PF } (n_p + n'_p) P$$ The 
	zero element of the divisor group $\text{Div}(F)$ is the divisor;
	$$ 0 := \sum_{p \in \PF} r_P P$$ where all $r_P =  0$. 
	For all $Q \in \PF $ and $D \in \text{Div} (F)$ we define
	$v_Q(D) := n_Q$.
\end{definition}

\begin{example}
	Consider the rational function field $\Comp(x)/\Comp$. Since 
	$\Comp $ is algebraically closed, we may identify the places
	of $\Comp(x)/\Comp$ with $\Comp \cup \lbrace \infty \rbrace $
	as follows: Let $P$ be a place of $\Comp(x)/\Comp$.
	By theorem \ref{noPlaceLikeHome}, if $P$ is not the infinity 
	place, then it can be indentified with 
	a irreducible polynomial $p$ in $\C[x]$. 
	Since $\C$ is algebraically closed,
	$p = x - a$ for some $a \in \Comp$. Through this, 
	all non-infinity places maybe be identified with some 
	$a \in \Comp$. We identify $P_\infty$ with $\infty$.
	In this view, $3(i) + \sqrt{3}i(\infty) \in \text{Div}(\Comp(x))$
\end{example}

\begin{example}
	Let $F/k$ be a function field and $D = \sum_Pn_PP $ where 
	$n_P = 1 $ for all $P\in \PF$. Then $D  \notin \text{Div}(F)$
	since it has infinitely many nonzero coefficients.  
\end{example}

\begin{lemma} \label{finZeroPole}
	Let $F/k$ be a function field. Every $z \in F$ has finitely
	many zeros and finitely many poles.
\end{lemma}

\begin{proof}
	See Stichtenoth corollary 1.3.4. for proof. 
\end{proof}

\begin{definition} \label{divValuation}
	Let $F/k $ be a function field. 
	A partial ordering on $\text{Div}(F)$ is defined by;
	$$ D_1 \leq D_2 \Leftrightarrow v_P(D_1) \leq v_P(D_2) \text{ for all } P\in \PF $$ 
	A divisor $D \geq 0 $ is called possitive (or effective). 
	The degree of a divisor is defined as;
	$$ \text{deg}(D) := \sum _{P \in \PF}v_P (D) \cdot \deg P  $$ 
	which is a homomorphism $\text{Div}(F) \stackrel{\text{deg}}{\longrightarrow} \Z $. 
	By Lemma \ref{finZeroPole} any nonezero element $x \in F$ 
	has finitely many zeros and poles in $\PF$. Thus this 
	definition makes sense.
\end{definition}

\begin{definition} \label{princeDiv}
	Let $F/k $ be a function field. 
	Let $ 0 \neq x \in F$ and denote by $Z$ (respectively $N$) 
	the set of all zeros (repectively poles) of $x$ in $\PF$. 
	Then we define
	$$ (x)_0 := \sum _{P \in Z} v_p(x) P$$ $$ (x)_{\infty} :=  
	\sum_{P\in N} (-v_P(x))P $$ $$ (x) := (x)_0  - (x)_{\infty} $$ 
	where $(x)_0,(x)_{\infty} $, and $(x)$ are called the 
	zero divisor of $x$, the pole divisor of $x$ and the 
	principal divisor of $x$ respectively. 
\end{definition}

\begin{example}
	Let $F = \mathbb{C}(x)/\mathbb{C}$ be the rational function field
	and consider the polynomial $f = x^3(x+1) \in F$. Recall from example
	\ref{firstencounterwithpolesanddivisors} that the
	valutions of $f$ at places $P_x$,$P_{x+1}$ and $P_\infty$ were $3,1$ 
	and $-4$ respectively. Let $p$ be a monic irreducible polynomial in
	$\mathbb{C}[x]$ other than $x$ and $x + 1$. Consider the place $P= P_p$. 
	Suppose $p \mid f $, then $f = ph$ for some monic $h \in \mathbb{C}[x]$. 
	So $p = x^n(x+1)^m$ and $h=x^r(x+1)^s$ for some $n,m,r,s \in \mathbb{Z}^+$ 
	such that $n+r = 3$ and $m+s = 1$. Since $p \neq x$, $p \neq x+1$ and $p$ is 
	irreducible, it follows that $n = m = 0$ and $h =f$. Therefore
	$f = p^0x^3(x+1)$ and $p \nmid f$, so $(x^3(x+1))^{-1} \in \mathcal{O}_P$. 
	Furthermore $x^3(x+1) \in \mathcal{O}_P$, since $p \nmid 1$. 
	Which implies $ x^3(x+1) \in \mathcal{O}_P^*$ and $v_P(f) = 0$. This imples, by theorem
	\ref{noPlaceLikeHome}, that the only zeros of $f$ in $\Comp(x)/ \Comp$
	are the places $P_{x}$ 
	and $P_{x+1}$, while the only pole of $f$ in $\Comp(x)/ \Comp$
	is $P_\infty$. Hence for 
	the element $f \in \Comp(x) / \Comp$ we obtain divisors; 
	$$(f)_0 = 3P_x + P_{x+1}$$
	$$(f)_{\infty} = 4P_{1/x}$$ 
	$$(f) = 3P_x + P_{x+1} - 4P_{1/x}$$
\end{example}

\begin{definition} \label{divPrinc}
	Let $F/k $ be a function field. 
	The set of divisors; $$ Princ(F) : = \lbrace (x) \mid 0 \neq x \in F \rbrace  $$ 
	is called the group of principal divisors of $F/k$.
\end{definition}

\begin{example} 
	Again, consider the rational function field $ F = \Comp(x)/\Comp $. 
	Let $f \in \mathbb{C}[x] \backslash \lbrace 0 \rbrace  $ 
	and suppose we know the prime factorization of
	$f = a p_1^{e_1}p_2^{e_2}...p_n^{e_n}$ for
	$e_1,e_2,...,e_n \in \mathbb{N} \text{, } a\in \mathbb{C}$ and $p_1,p_2,...,p_n$ 
	distinct monic irreducible polynomials in
	$\mathbb{C}[x]$. Denote the place of $\mathbb{C}(x)$ at prime
	$p_i$ by $P_i = P_{p_i}$ for $i = 1,2,...n$. Then at places
	$P_1,P_2,...,P_n \in \mathbb{P}_F$, $f$ has
	valuation $v_{P_i}(f) = e_i $ for $i=1,2,...,n$. To verify 
	this claim it suffices
	to show that $fp_i^{-e_i} \in \mathcal{O}_{P_i}^*$ for 
	all $i = 1,2,...n$. Since $p_1,p_2,...,p_n$ are distinct
	irreducible polynomials in $\mathbb{C}[x]$, it follows that
	$p_i \nmid p_j $ for all $j \neq i$, thus
	$v_{p_i}(fp_i^{-e_{i}}) \leq 0$ for all $i = 1,2,...,n$.
	Since $e_1,e_2,...,e_n \geq 0$, it follows that
	$v_{p_i}(fp_i^{-e_{i}}) \geq 0$, and thus
	$v_{p_i}(fp_i^{-e_{i}}) = 0$ for all $i=1,2,...,n$. 
	By Theorem \ref{primeValuation} part \ref{valutionringbyvaluations},
	$fp_i^{-e_i} \in \mathcal{O}_{P_i}^*$ 
	for all $i=1,2,...,n$. From theorem \ref{noPlaceLikeHome}, 
	we get that besides the infinity place, 
	these are the only zeros of $f$: for any other 
	zero would have to be at a place corresponding to a 
	irreducible polynomial not in the representation of $f$ 
	and thus would have valuation $0$. Hence the $f$ has zero 
	divisor $$ (f)_0 = \sum^n_{i=1} e_i P_i $$ We calculate 
	the valutation at the infinity place $P_{\infty}$;
	$$ v_{\infty}(f) = \deg (1) - \deg(a p_1^{e_1}p_2^{e_2}...p_n^{e_n}) = 0 - \sum^n_{i=1} e_i \cdot \deg(p_i)$$ 
	Since $p_1,p_2,...,p_n \in \mathbb{C}[x]$, 
	they all have degree 1,  $v_{\infty}(f) = \sum^n_{i =1 } e_i$. 
	So
	$$ (f)_{\infty} = (\sum^n_{i=1} e_i )P_{\infty}$$
	$$(f)= \sum^n_{i=1} e_i P_i -   (\sum^n_{i=1} e_i) P_{\infty} $$ 
	To calculate the degree of $(f)_0,(f)_{\infty},(f)$, we need to find the
	degrees of the places $P_1,P_2,...,P_n$ and $P_{\infty}$. That is,
	calculate $\deg P_i = [F_{P_i} : \mathbb{C}] = [\mathcal{O}_{P_i}/ P_i] : \mathbb{C}]$ 
	for $i = 1,2,...,n $ and $\deg P_{\infty}$. 
	By proposition \ref{propaboutrationals} part \eqref{isoofresidueraitonal},
	$F_{P_i} = \mathcal{O} _ {P_i} / P_i \cong \mathbb{C}[x]/(p_i) $ for all
	$i = 1,2,...,n$. Since each $p_i$ is linear,
	$[\mathbb{C}[x]/(p_i) : \mathbb{C}] = 1$ for all
	$i = 1,2,...n $. Part \eqref{intinitydegree1} of proposition
	\ref{propaboutrationals} states
	that $degP_{\infty} = 1$, hence;
	$$\deg (f)_0 = \sum^n_{i=1} e_ i \cdot \deg P_i =  
	\sum^n_{i=1} e_i = \deg _{\mathbb{C}[x]}(f)$$
	$$\deg (f)_{\infty} = (\sum^n_{i=1} e_i ) \cdot P_{\infty} = 
	\sum^n_{i=1} e_i = \deg _{\mathbb{C}[x]}(f) $$
	$$ \deg (f) =  \sum^n_{i=1} e_i - \sum^n_{i=1} e_i  = 0$$
	So the degree of every principal divisor of a polynomial in $\Comp[x]$ has
	degree 0. Theorem \ref{Princ0} will generalize this result.
\end{example} 

\begin{definition} \label{RRdef}
	Let $F/k $ be a function field. 
	For a divisor $A \in Div(F)$ we define the Riemann-Roch space 
	associated to A by
	$$ \LL(A) := \lbrace x \in F \mid (x) \geq - A \rbrace  $$
\end{definition}

\begin{lemma} \label{RRisVectorSpace}
	Let $F/k $ be a function field.
	Let $A\in Div(F)$. Then 
	$\LL(A)$ is a vector space over $k$.
\end{lemma}

\begin{proof}
	Let $x,y \in \LL(A)$. Then $v_P(x) \geq - v_P(A) $ and 
	$v_P(y) \geq - v_P(y)  $ for all $P\in \PF$. Suppse $v_P(x) < v_P(y)$
	for all $P \in \PF$. Then
	$v_P(x+y) = min \lbrace v_P(x),v_P(y) \rbrace  = v_P(x) \geq v_P(A)$. 
	So $x + y \in \LL(A)$. Let $a\in k$. 
	Then $v_P(ax) = v_P(a) + v_P(x) = 0 + v_P(x) = v_P(x) \geq - v_P(A)$
	for all $P\in \PF$.
\end{proof}

\begin{definition}
	Let $F/k $ be a function field. 
	For a divisor $A \in Div(F)$ the integer $ \ell (A) := dim \LL (A) $ 
	is called the dimension of the divisor $A$.
\end{definition}

\begin{example} \label{ellipticRR}
	Let $p = y^2 + x^3 - x$ and consider the integral domain
	$A$ as in example
	\ref{elliptic}. Let $\pi : \Comp[x,y] \to A$ be the 
	canonical homomorphism of the quotient ring.
	For simplicity, define $x := \pi(x)$ and $y := \pi(y)$. 
	Then $A = \Comp [x,y]$ where $x,y \in A$ satisfy $y^2+x^3-x=0$.
	Exercise \ref{irriduciblePoverC} shows that 
	$A$ is a free $\Comp[x]$-module, with basis $\{ 1, y \}$. 
	So each element of $A$ has a unique expression of the form
	$p(x)y + q(x)$ where $p(x), q(x)$ are polynomials in $x$. 
	From exercise \ref{irriduciblePoverC} we also know that $F/\Comp$ 
	is a function field over $\Comp$. Let $\VVV = \VVV( F/ \Comp )$ 
	be the set of valuation rings of $F/ \Comp$ 
	and $\PPP = \PPP( F/ \Comp )$ the set of places. 
	If we make the following assumptions;
	\begin{itemize}
		\item There is exactly one element $\Oeul \in \VVV$ such that
		$A \nsubseteq \Oeul$. Denote it by $\Oeul_\infty$,
		let $P_\infty$ be its maximal ideal, and let
		$v_\infty : F^* \to \Integ$ be its valuation.
		\item $v_\infty(x) = -2$
		\item \label{Aisintersection}$A$ is equal to the intersection
		 of all rings $\Oeul \in \VVV \setminus \{ \Oeul_\infty \}$.
	\end{itemize}
	Then we have the following;
	\begin{enumerate}[(a)]
		\item If $f \in F^*$ then $v_P(f) \ge 0$ for all
		 $P \in \PPP \setminus \{ P_\infty \}$ $\iff$ $f \in A$.
		\item $v_\infty(y)=-3$
		\item For any $f = p(x)y + q(x) \in A$, let $m=\deg_x( p(x) )$ and $n = \deg_x( q(x) )$. 
		\begin{enumerate}[(i)]
			\item $v_\infty( p(x)y ) = -2n - 3$
			\item $v_\infty( q(x) ) = -2m$
			\item $v_\infty( p(x)y ) \neq  v_\infty( q(x) )$
			\item $v_\infty(f) = - \max \lbrace 2n + 3, 2m \rbrace $
		\end{enumerate}
		\item Let $N \ge 2$. Then $\Leul( 2N P_\infty )$ has basis 
		$\lbrace y, xy, x^2y,...,x^{N-2}y,1,x,x^2,...,x^N \rbrace $
		and dimension $\ell( 2N P_\infty ) = 2N$.
	\end{enumerate}
\end{example}

\begin{proof}
\begin{enumerate}[(a)]
	\item \label{inAiff} Let $f\in F^*$. $v_P(f) \geq 0 $ 
	for all $P\in \mathbb{P}_F \backslash \lbrace P_\infty \rbrace $ 
	if and only if
	$f \in O_P$ for all
	$P\in \mathbb{P}_F \backslash \lbrace P_\infty \rbrace $ 
	if and only if $f \in A$ by assumption \ref{Aisintersection}.

	\item Recall lemma \ref{strongtriange}. Then 

	\begin{align*}
		2v_\infty(y) &= v_\infty(y^2) \\
		&= v_\infty(x - x^3) \\
		&= v_\infty(x) + v_\infty(1+x) + v_\infty (1-x) \\
		&= v_\infty(x) + \min \lbrace v_\infty(1), v_\infty(-x) \rbrace + \min \lbrace v_\infty(1) , v_\infty(x) \rbrace \\
		&= (-2) + (-2) + (-2) \\
		&= -6
	\end{align*}

	Hence $v_\infty(y) = -3 $

	\item Since $p,q$ are polynomials in $\Comp [x]$, we may 
	write them as $p_1^{e_1}p_2^{e_2}...p_s^{e_s}$ and
	$q_1^{f_1}q_2^{f_2}...q_t^{f_t}$ respectively, 
	where $p_1,...,p_s,q_1,...q_t$ are linear polynomials 
	in $\Comp[x]$ and $e_1,...,e_s,f_1,...,f_t \in \Integ$. Hence 

	\begin{align*}
		v_\infty(py) &=v_\infty(p_1^{e_1}p_2^{e_2}...p_s^{e_s}y) \\
		&= e_1v_\infty(p_1) + ... + e_sv_\infty(p_s) + v_\infty (y)  \\
		&= e_1(-2) + ... + e_s(-2) - 3 \\
		&= -2n - 3
	\end{align*}

	\begin{align*}
		v_\infty(q) &=v_\infty(q_1^{f_1}q_2^{f_2}...q_t^{f_t}) \\
		&= f_1v_\infty(q_1) + ... + f_tv_\infty(p_t)  \\
		&= f_1(-2) + ... + f_t(-2) \\
		&= -2m
	\end{align*}

	Notice that 

	\begin{align*}
		v_\infty(py)=v_\infty(q) & \Longrightarrow -2n - 3 =  -2m  \\
		& \Longrightarrow m = n + 3/2
	\end{align*}

	Which is impossible since $q \in \Comp [x]$. 
	Hence $v_\infty(py) \neq v_\infty (q) $. Therefore 


	\begin{align*}
		v_\infty (f) &= v_\infty( p(x)y + q(x) ) \\
		&= \min \lbrace  v_\infty( p(x)y),  v_\infty( q(x) ) \rbrace \\
		&= \min \lbrace -2n - 3, -2m \rbrace  \\
		&= - \max \lbrace 2n + 3, 2m \rbrace 
	\end{align*}

	\item Let $N\geq 2$ and write $f = py + q$. $f \in \Leul (2N P_\infty ) $ 
	if and only if $v_P(f) + v_P(2N P_\infty) \geq 0 $ 
	for all $P\in \mathbb{P}_F$ by definition. 
	By part \eqref{inAiff}, we know that $v_P(f) \geq 0 $ 
	for all $P\in \mathbb{P}_F \backslash P_\infty$ if and 
	only if $f \in A$. Since $v_P(2NP_\infty) = 0 $ for 
	all $P \in \mathbb{P}_F \backslash P_\infty$, we 
	require $v_P(f) \geq 0 $ for 
	all $P \in \mathbb{P}_F \backslash P_\infty$. 
	Thus $f \in A$. Hence

	\begin{align*} 
		\Leul (2N P_\infty ) &= \lbrace f \in A \mid \max \lbrace 2\deg(q), 3 + 2\deg(p) \rbrace \leq 2N \rbrace \\
		 &= \lbrace  py + q \in A \mid \deg(p) \leq N-3/2, \deg(q) \leq N \rbrace \\
		 &= \lbrace  py + q \in A \mid \deg(p) \leq N - 2, \deg(q) \leq N \rbrace 
	\end{align*}

	\noindent Hence $\lbrace y, xy, x^2y,...,x^{N-2}y,1,x,x^2,...,x^N \rbrace $ 
	is a basis for $\Leul (2N P_\infty )$ 
	over $\Comp$ and $\ell( 2N P_\infty )= (N - 1) + N + 1 = 2N$. 

\end{enumerate}
\end{proof}

\begin{example} \label{RRis0}
	Let $F/k$ be a function field over $k$ and $A \in \text{Div}(F)$.
	We have $\LL(0) = k$ and if $A < 0$ then $\LL(A) = \lbrace 0 \rbrace$.
\end{example}


\begin{proof}
	To show the first assertion, let $0 \neq x \in k$, then $(x)=0$. So 
	$x \in \LL(0)$. Let $0 \neq x \in \LL(0)$.
	Then $(x) \geq 0$, but then $x$ has no pole, so by corollary \ref{everyonesgotapole},
	so $x \in k$. Suppose $A < 0$ and let $ 0 \neq x \in \LL(A)$. Then $(x) \geq - A > 0$, 
	but then $x $ has at least one zero and no pole. This is impossible. Hence $x = 0$.
\end{proof}


\begin{proposition} \label{Dimofthequotient}
	Let $A,B$ be two divisors of $F/k$ with $A \leq B$. 
	Then we have $\LL(A) \subseteq \LL(B)$ and $\text{dim}(\LL(B)/\LL(B)) \leq \deg B - \deg A$. 
\end{proposition}

\begin{proof}
	Assume that $A,B$ be two divisors of $F/k$ with $A \leq B$.  
	We show $\LL(A) \subseteq \LL(B)$. Let $x \in \LL(A)$, then 
	$v_P(x) + v_P(x) \geq v_P(x) + v_P(B) \geq 0 $, so $x \in \LL(B)$.
	Hence $\LL(A) \subseteq \LL(B)$. 
	To vertify the second claim, assume that 
	$B = A + P $ for some $P\in \PF$. This is possible since 
	the general case follows by induction. Let $t \in F $ such that 
	$v_P(t) = v_P(B) = v_P(A) + 1$. For $x \in \LL (B)$ we have 
	$v_P(x ) \geq - v_P(B) = - v_P(t)$, so $xt\in \OV_P$. So we obtain
	a $k-linear$ map $ \phi : \LL(B) \Longrightarrow F_P$, $x \mapsto xt + P$. 
	An element $x$ is in the the kernel of $\phi $ if and only if 
	$v_P(xt) > 0$, that is $v_P(x) \geq - v_P(A)$. So $\text{ker} \phi = \LL(A)$.
	Thus $\phi $ induces an \textit{injective} $k$-linear map from $\LL(B)/ \LL(A)$ to $\FP$. 
	Therefore $\text{dim} \LL(B)/ \LL(A) \leq \text{dim} F_P = \deg B - \deg A$.
\end{proof}

\begin{lemma} \label{lessthanextensiondeg}
	Let $F/k$ be a function field and let $P_1,...,P_r$ be 
	zeros of the element $x \in F$. Then $ \sum^r_{i=1} v_{P_i}(x) \leq [F:k(x)]$.
\end{lemma}

\begin{proof}
	See Stichtenoth Proposition 1.3.3.
\end{proof}

\begin{theorem} \label{Princ0}
	All principal divisors have degree zero. More precisely, 
	let $x \in F \backslash k$ and $(a)_0$ resp. $(a)_{\infty}$ 
	denote the zero resp. pole divisor of $x$. 
	Then $$deg(x)_0=deg(x)_{\infty}=[F:k(x)]$$. 
\end{theorem}


\begin{proof}
	Let $n := [F:k(x)]$. Then $\deg (x)_\infty  \leq n$ by 
	\ref{lessthanextensiondeg}, we have 
	$\sum^r_{i=1 }v_{P_i}(x) \leq [F:k(x)]$.
	Thus it remains to show that $n \geq \deg (x)_\infty $. 
	Let $v_1,...v_n$ as a basis for $F/k(x)$. Let $A \geq 0$ 
	be a divisor such that $(v_i) \geq A$ for $i = 1,..,n$.
	Then we have $\Leul(r(x)_\infty + A ) \geq n(r +1 )$ for 
	all $r \geq 0$, since $x^iv_j \in \LL (r (x)_\infty + 1)$ 
	for $0 \leq i \leq r$, $1 \leq j \leq n$. Letting $c: = \deg A$
	we get $n(r + 1 ) \leq \Leul (r(x)_\infty +A) \leq r \cdot (x)_\infty + c + 1 $. 
	Thus $r(\deg (x)_\infty - n )  \geq n - c - 1 $ 
	for all $r \in \Nat$. Hence $\deg (x)_\infty \geq n$.
\end{proof}

\begin{proposition}
	There is a constant $\gamma \in \Z $ such that for all 
	divisors $A \in Div(F)$ the following holds:
	$$ degA - l(A) \leq \gamma $$
\end{proposition}

\begin{proof}
	First observe that by proposition \ref{Dimofthequotient}, 
	$A_1 \leq A_2  \Rightarrow \deg A_1 - \Leul(A_1) \leq \deg A_2 - \Leul(A_2)$.
	Let $x \in F \backslash k$ and consider the divisor $(x)_\infty$. 
	There exists a divisor $C \geq 0 $ such that
	$\Leul (r(x)_\infty + C ) \geq (r + 1) \cdot \deg (x)_\infty$ for all 
	$r \geq 0$. We also have $\Leul (r(x)_\infty + C) \leq \Leul(r(x)_\infty) + \deg C$
	from proposition \ref{Dimofthequotient}. 
	Hence $\Leul(r(x)_\infty) \geq (r+1) \cdot \deg (x)_\infty 
	- \deg C = \deg (r(x)_\infty ) + ([F : k(x)] - \deg C) $. 
	Hence $\deg (r(x)_\infty ) - \Leul (r(x)_\infty ) \leq \gamma$ for all $r > 0$
	with some $\gamma \in \Integ$. 
	\textit{Claim}: For all $A \in \text{Div}(F)$, there exists divisors 
	$A_1,D$ and a integer $r \geq 0$ such that $A \leq A_1$, $A = D + P$ 
	for some $P \in \PF$ and $D \leq r(x)_\infty$. \textit{Proof of claim:}
	Let $A_1 \geq A $ such that $A_1 > 0 $. 
	Then $\Leul (r (x)_\infty - A_1) \geq \Leul (r(x)_\infty ) - 
	\deg A_1 \geq \deg (r(x)_\infty ) - \gamma - \deg A_1 > 0 $
	for sufficiently large $r$. Thus there exists some nonzero element 
	$z \in \LL ( r(x)_\infty - A_1)$. Letting $D:= A_1 - (z)$, we 
	obtain $A = D + P $ where $P + -(z)$ and
	$D \leq A_1 - (A_1 - r(x)_\infty ) = r(x)_\infty$ as desired.
	Thus the claim is verified. 
	From this, observe that
	$\deg A - \Leul (A) \leq \deg A_1 - \Leul (A_1) = \deg D 
	- \Leul (D) \leq \deg(r(x)_\infty ) - \Leul (r(x)_\infty ) \leq \gamma $.
\end{proof}



\begin{definition}
	Let $F/k $ be a function field.
	The genus of $g$ of $F/k$ is defined 
	by $$ g := max \lbrace degA - l(A) + 1 \mid A \in Div(F) \rbrace $$
\end{definition}


\begin{theorem}[Riemann's Theorem] \label{Riemannstheorem}
	Let $F/k$ be a function field of genus $g$. 
	Then there exists an integer $c$, depending only on the
	function field $F/k$, such that $l(A) = degA + 1 - g $ whenever $degA \geq c$.
\end{theorem}


\begin{proof}
	Let $A_0$ such that $g =	 \deg A_0 \Leul (A_0) + 1$ and 
	set $c = \deg A_0 + g$. If $\deg A \geq c$ then 
	$\Leul (A-A_0 ) \geq \deg (A-A_0) + 1  - g \geq c - \deg A_0 + 1 - g = 1$.
	So there is an element $0 \neq z \in \LL(A - A_0)$. 
	Consider the divisor $A^{\prime} =  A + (z) \geq A_0$. 
	We have
	$\deg A^{\prime } - \Leul (A^{\prime}) \geq \deg A_0 - \Leul (A_0) = g -1$.
	Hence $ \Leul (A) \leq \deg A + 1 - g$.
\end{proof}


\begin{example}
	Recall the setup of example \ref{ellipticRR}. Let $N > 0$ be 
	arbitrarly large. We know that 
	$\Leul (2NP_\infty) = 2N$. Assume $\deg P_\infty = 1$.
	Then by Riemann's Theorem, $g =\deg(2NP_\infty)  - \Leul (2NP_\infty) + 1 = 1$. 
	Then we may conclude that $F/\Comp$ has genus $1$, 
	that is the curve $p = y^2 + x^3 - x$ has genus $1$. Similary, 
	from exercise \ref{infinity}, we know that $\Leul (NP_\infty) = N + 1$, 
	where $P_\infty$ denotest the infinity place of the function field $k(x)/k$.
	From proposition \ref{propaboutrationals}, $\deg P)\infty = 1$. Thus by 
	Riemann's theorem $g = \deg(NP_\infty ) - \Leul (NP_\infty) + 1 =N - N-1  + 1= 0$
	Hence the rational function field has genus $0$.
\end{example}


