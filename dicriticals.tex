


\begin{lemma} \label {i9283hd01bxdo912ep01}
Let $E/k$ be a field extension and $u,v \in E$.  
Then we have field extensions:
$$
\xymatrix@R=0pt@C=6pt{
& E  \\
\\
& k(u,v) \ar @{-} [dl] \ar @{-} [dr] \ar @{-} [uu] \\
k(u) && k(v) \\
& k \ar @{-} [ul] \ar @{-} [ur]
}
$$
Assume that each of $u,v$ is transcendental over $k$. 
Then the following are equivalent:
\begin{itemize}

\item \label{con1} $(u,v)$ is algebraically dependent over $k$;
\item \label{con2} $v$ is algebraic over $k(u)$;
\item \label{con3} $u$ is algebraic over $k(v)$.

\end{itemize}
\end{lemma}

\begin{proof}
$\eqref{con1}  \Longrightarrow \eqref{con2} $ Assume $(u,v)$ 
are algebraically dependent over $k$. Then there 
exists $$ f(X,Y) = \sum_{i,j \in \N} a_{ij} X^iY^j \in 
k[X,Y] \backslash \lbrace 0 \rbrace $$ such that
$f(u,v) = \sum_{i,j \in \N} a_{ij} u^iv^j = 0 $. 
Consider $$g(Y) = \sum_{i,j \in \N} a_{ij} u^iY^j 
\in k[u][Y] \backslash \lbrace 0 \rbrace $$ We still 
have $g(v) = \sum_{i,j} a_{ij} u^iv^j = 0 $. 
Hence $v$ is algebraic over $k[u] \subset k(u)$. 


$\eqref{con2} \Longrightarrow \eqref{con3} $; Assume $v$ is 
algebraic over $k(u)$. Then there exists
$$ f(Y) = \sum_{i\in \N} a_i(u)Y^i \in k(u)[Y] \backslash \lbrace 0 \rbrace $$ 
such that $f(v) = 0$. Since $a_i(u) \in k(u)$ for 
all $i$, $a_i(u) = \frac{f_i(u)}{g_i(u)} $ 
where $f_i(u),g_i(u) \in k[u]$ and $g_i(u) \neq 0 $. 
Let $$a(u) := \prod_i g_i(u) \in k[u] \backslash \lbrace 0 \rbrace $$ 
Then consider $$ g(Y) := a(u)f(Y) = a(u)\sum_{i\in \N}
a_i(u)Y^i = \sum_{i \in \N}a(u) a_i(u)Y^i $$ 
Note that $h_i(u):=a(u)a_i(u) \in k[u] $ for all $i$ 
and we still have $g(v) = 0 $. We can rewrite $g(Y)$ 
as $$ g(X) = \sum_{i \in \N} h_i(X)v^i \in k[v][X] \backslash \lbrace 0 \rbrace $$ 
and we still have  $g(u) = \sum_{i \in \N} h_i(u) v^i = 0 $. 
Hence $u$ is algebraic over $k[v] \subset k(v)$ 

$\eqref{con3} \Longrightarrow \eqref{con1}$; 
Assume $u$ is algebraic over $k(v)$. Then there 
exists $$ f(X) = \sum_{i \in \N } a_i(v)X^i \in k(v)[X] 
\backslash \lbrace 0 \rbrace $$ such that $f(u) = 0$. 
Since $a_i(v) \in k(v) $ for all $i$, $a_i(v)= \frac{f_i(v)}{g_i(v)}$ 
where $f_i(v),g_i(v) \in k[Y]$ and $g_i(v) \neq 0 $. 
Let $$a(v) := \prod_i g_i(v) \in k[v] \backslash \lbrace 0 \rbrace  $$ 
Then consider $$ g(X) = a(v)f(X) = a(v)\sum_{i \in \N} a_i(v)X^i 
= \sum_{i \in \N} a(v)a_i(v)X^i $$ Note that
$h_i(v):=a(v)a_i(v) \in k[v] $ for all $i$ and we still 
have $g(u) = 0 $. We can rewrite $g(Y)$ as
$$ g(X,Y) = \sum_{i \in \N} h_i(Y)X^i \in k[X,Y] 
\backslash \lbrace 0 \rbrace  $$ We still have
$g(X,Y) = \sum_{i \in \N} h_i(v) u^i = 0 $. 
Hence $(u,v)$ is algebraically independent over $k$.
\end{proof}

\begin{lemma} \label {z0cb9182wbxi81f6}
Let $k$ be a field and let $k(x_1, \dots, x_n)$ be 
the field of fractions of the
polynomial ring  $k[x_1, \dots, x_n]$ in $n$ variables 
over $k$ (where $n \ge 1$).
Then $k$ is algebraically closed in $k(x_1, \dots, x_n)$.
\end{lemma}

\begin{proof}
Induction on $n$. When $n= 1$, then we are considering 
the rational function field in one variable over $k $.
Let $P$ be a place of $k(x)/k$ of degree $1$. That is
$P:=P_{x-a}$ for $a \in k$. The algebraic closure of 
$k$, denoted $\tilde{k}$, embeds into $k(x)_P$, since 
$P \cap \tilde{k} = \lbrace 0 \rbrace$. Then we have 
$k \subseteq \tilde{k} \subseteq k(x)_P = k$. 
Hence $k$ is algebraically closed in $k(x)$. 
Assume that $n>1$.
Inductive hypothesis: $k$ is algebraically closed 
in $k(x_1, \dots, x_{n-1})$.
To prove that $k$ is algebraically closed in $k(x_1, \dots, x_n)$, we consider
an element $w$ of $k(x_1, \dots, x_n)$ that is algebraic 
over $k$. We have to show that $w \in k$.
Observe that $w$ is algebraic over $k(x_1, \dots, x_{n-1})$.
Write $F = k(x_1, \dots, x_{n})$ and $K = k(x_1, \dots, x_{n-1})$.
Then $F/K$ is the rational function field of one variable, 
so $K$ is algebraically closed in $F$, so $w \in K$.
As $w \in k(x_1, \dots, x_{n-1})$ is algebraic over $k$, 
the inductive hypothesis implies that $w \in k$.
\end{proof}



For this section, let $ k$ be a field, $A =  k[x,y]$ 
the polynomial ring in two variables over $ k$,
and $L= \Frac A =  k(x,y)$, the field of rational functions in two variables.
The objects $ k$, $A$ and $L$ are fixed throughout.
For each choice of $F \in A \setminus  k$, 
we may consider the subfield $K= k(F)$ of $L$
(since $F$ is an element of the field $L =  k(x,y)$, 
it follows that $ k(F)$ is a subfield of $L$).
We have $ k \subset K \subset L$ where $ k$ 
and $L$ are always the same but $K$ depends on the choice of $F$.
We are particularly interested in the field extension $L/K$.
Our first objective is to show that $L/K$ is a function field 
of one variable.
There are several steps in the proof of this.

\begin{remark}
By lemma~\ref{z0cb9182wbxi81f6}, $ k$ is algebraically closed in $L$.
However, whether or not $K$ is algebraically closed in $L$ depends on the choice of $F$.
For instance, if $F=x$ then $L/K$ is the rational function field of one variable, so $K$ is algebraically closed 
in $L$ in this case.  But if $F = x^2$ then $x$ is an element of $L$ that is algebraic over $K$ but that does not 
belong to $K$, so $K$ is not algebraically closed in $L$ in this case.
\end{remark}

\begin{proposition} \label {90cdk2938db129}
Show that the following are equivalent:
\begin{enumerate}[(i)]
\item \label{fact1} $(F,x)$ is algebraically dependent over $ k$;
\item \label{fact2} $x$ is algebraic over $K$;
\item \label{fact3} $F$ is algebraic over $ k(x)$;
\item \label{fact4} $F \in  k(x)$;
\item \label{fact5} $F \in  k[x]$.
\end{enumerate}
\end{proposition}

\begin{proof}
Since $A$ is a polynomial ring in two variables $x,y$, by definition $(x,y)$ 
is algebraically independent over $ k$, so we have that $x$ is transcendental 
over $ k$. Suppose $F$ is algebraic over $ k$, 
then $F \in  k$ by \ref{z0cb9182wbxi81f6}, 
and this contradicts the hypothesis $F \in A \setminus k$.
Therefore we may use lemma \ref{i9283hd01bxdo912ep01} 
to show that \ref{fact1}, \ref{fact2} and \ref{fact3} are all equivalent.
(\eqref{fact3} $\Rightarrow $ \eqref{fact4}): 
Since $L/ k(x)$ is the rational function field 
in one variable, it follows that $ k(x)$ is algebraically closed
in $L$.
(\eqref{fact4} $\Rightarrow $ \eqref{fact5}): 
$ F \in  k[x,y] \backslash  k$ by definition. 
Hence if $F\in  k(x) $. We want to show that 
$ k(x) \cap  k[x,y] =  k[x]$. 
It is clear that $ k[x] \subseteq k(x) \cap  k[x,y]$.
Consider an element $\xi \in  k(x) \cap  k[x,y]$.
Since $\xi \in  k(x)$, we may write $\xi = fg^{-1}$ for
$f,g\in  k[x]$ and $g \neq 0$. Thus $g \in  k[x,y]$. 
So $g$ must belong to $\Comp$. Hence $ k(x) \cap  k[x,y] =  k[x]$.



(\eqref{fact5} $\Rightarrow $ \eqref{fact3}):
If $F \in  k[x]$, then $F$ is algebraic over $ k[x] \subset  k(x)$.
\end{proof}


\begin{remark}
The result of proposition \ref{90cdk2938db129} remains valid 
if one replaces all `$x$' by `$y$' in the statement.
In particular, if $y$ is algebraic over $K$ then $F \in  k[y]$.
\end{remark}

\begin{corollary} \label {9fo13d912o9o0}
At least one of $x,y$ is transcendental over $K$.
\end{corollary}

\begin{proof}
Let $F \in A \backslash  k$. Then if $F \notin  k(x)$, then $x$ is 
transcendental over $K$ by proposition \ref{90cdk2938db129}. Similarly, 
if $F\notin  k(y)$, then $y$ is transcendental over $K$. 
\end{proof}

\begin{proposition}
For some $t \in \{x,y\}$, $L/K(t)$ is a finite extension and therefore
$L/K$ is a function field of one variable.
Furthermore, $F \in  k[x]$.
\end{proposition}

\begin{proof}
Suppose that both $L/K(x)$ and $L/K(y)$ are not finite. Then we get the following
chain of inclusions $ k(x) \subseteq K(x) \subset L$ and $ k(y) \subseteq K(y) \subset L$. 
Observe that both $L/ k(x)$ and $L/ k(y)$ are function fields in one 
variable. So the fact that both $L/K(x)$ and $l/K(y)$ are not finite implies 
that $K(x)/ k(x)$ and $K(y)/ k(y)$ are both algebraic. In particular,
$F$ is algebraic over $ k(x)$ and $ k(y)$. Hence by proposition \ref{90cdk2938db129}
$F\in  k[x]$ and $F\in  k[y]$, which is impossible. So for some $t \in \{x,y\}$, $L/K(t)$ is a finite extension.
\end{proof}

\begin{remark}
So, for any choice of $F \in A \setminus  k$, $L/K$ is a function field of one variable
(it is important that $F \notin  k$ here; 
if $F \in  k$, then $ k(F)= k$ and the field extension $L/ k$ has transcendental degree 2).
The properties of the function field $L/K$ depend on the choice of $F$:
whether or not $K$ is algebraically closed in $L$ depends on the choice of $F$;
whether or not $L/K$ is the rational function field depends on the choice of $F$.
\end{remark}

\begin{example} \label {ue0982rhr329r23jew}
Let $k = \Comp$. Then for each of
the following values of $F$, the function 
field $L/K$ is the rational 
function field.

\begin{enumerate}
	\item $F = xy^2$
	\item $F = x^2y^3$
	\item $F = x(y+x^3)$ 
	\item $F = y^2 + x^2 - 1$
\end{enumerate}
To prove this, it suffices to find $G \in L$ such that $L = K(G)$.
\begin{enumerate}
	\item $K(y)=\Comp (xy^2,y) = \Comp (xy^2y^{-2},y)=\Comp (x,y) = L$
	\item $K(xy)=\Comp(x^2y^3,xy)=\Comp(x^2y^3x^{-2}y^{-2},xy)=\Comp(y,xy)=\Comp(x,y)=L$
	\item $K(x)= \Comp(x(y+x^3),x)=\Comp(y+x^3,x)=\Comp(y+x^3 - x^3,x)=\Comp(x,y)=L$ 
	\item Let $u = x+iy$ and $v=x-iy$ of $L=\Comp(x,y)$. 
	Notice that $uv=x^2+y^2 = F + 1$.  
	So $K = \Comp(x^2+y^2-1) = \Comp(x^2+y^2) = \Comp(uv)$ and
	consequently $K(v) = \Comp(uv,v) = \Comp(u,v) = L$.
\end{enumerate} 
\end{example}

\begin{notation}
Let $F \in A \setminus  k$.
\begin{enumerate}[(a)]
\item Let $\VV(F)$ be the set of all valuation rings of the function field $L/K$.
The notation `$\VV(F)$' reminds us that this set of rings depends on the choice of $F$.

\item Let $\PP(F)$ be the set of places of $L/K$.

\item Let $\VV^\infty(F) = \setspec{ R \in \VV(F) }{ A \nsubseteq R }$.

\end{enumerate}
Note that $\VV^\infty(F) = \setspec{ R \in \VV(F) }{ \{x,y\} \nsubseteq R }$.
\end{notation}


\begin{proposition} \label {3o9jdb9182so}
Let $F \in A \setminus  k$.
\begin{enumerate}[(a)]
\item \label{notmempty} $\VV^\infty(F)$ is a nonempty set.
\item \label{finite} $\VV^\infty(F)$ is a finite set.
\end{enumerate}
\end{proposition}

\begin{proof}
\eqref{notmempty}: By \ref{9fo13d912o9o0}, 
we know that at least one of $x,y$ is transcendental 
over $K$. Assume $x$ is transcendental over $K$, then by 
corollary \ref{everyonesgotapole}, $x$ has at least one pole. 
That is, there exists 
a place $P$ such that $v_P(x)<0$. Let $\OV_P$ be 
the valuation ring of $L$ corresponding to $P$. Hence $V^{\infty}(F) \neq \emptyset$. 
\eqref{finite}:
\end{proof}

\begin{definition}
Let $F \in A \setminus  k$.
The elements of the nonempty finite set $\VV^\infty(F)$ are called the {\it dicriticals\/} of $F$.
We define the {\it degree\/} of a dicritical $R$ of $F$ to be $\deg P$, where
$P$ is the place of $R$.
\end{definition}

\begin{example} \label {982b8d12vxowe9f}
Let $F=x$. In this case, we have $K= k(x)$; 
so $L=K(y)$ is the rational function field.
The place at infinity of $K(y)$ is $R = K[ z ]_{(z)}$ where $z = y^{-1}$;
since $y \notin R$, we have $R \in \VV^\infty(F)$.  
The degree of this dicritical is $1$,
because we know that the place at infinity of $K(y)$ has degree $1$.
If $R'$ is any valuation ring of $L/K$ other than $R$
then $R' = K[y]_{(p)}$ for some irreducible polynomial $p \in K[y]$;
then $x \in K \subseteq K[y]_{(p)}$ and $y \in K[y]_{(p)}$, 
so $\{x,y\} \subseteq R'$ and $R' \notin  \VV^\infty(F)$.
Hence  $\VV^\infty(x) = \lbrace K[z]_{(z)} \rbrace$. Since 
$\deg K[z]_{(z)} = 1$. We say that $F$ has only 1 rational 
dicritical. 
\end{example}

\begin{example} \label {0ndvx54120cmrnh}
The polynomial $F=xy$ has two dicriticals, $k[z_1]_{(z_1)}$ 
and $K[z_2]_{(z_2)}$, where $z_1=x^{-1}$ and $z_2=y^{-1}$. 
\end{example}


\begin{definition}
Use square brackets to represent unordered lists of positive integers.
For instance, $[1,1,2] = [1,2,1] = [2,1,1] \neq [1,2,2]$. Let $F \in A \setminus  k$.
Let $R_1, \dots, R_s$ be the distinct dicriticals of $F$, where $R_i$ is a dicritical of degree $d_i$.
Then we write $\Delta(F) = [ d_1, \dots, d_s ]$.
\end{definition}

\begin{examples}
From \ref{982b8d12vxowe9f}, $\Delta(x) = \Delta(y) = [1]$.
By \ref{0ndvx54120cmrnh}, $\Delta(xy)=[1,1]$.
\end{examples}
